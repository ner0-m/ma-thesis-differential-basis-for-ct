%%%%%%%%%%%%%%%%%%%%%%%%%%%%%%%%%%%%%%%%%%%%%%%%%%%%%%%%%%%%%%%%%%%%%%%%
\chapter*{Abstract}
%%%%%%%%%%%%%%%%%%%%%%%%%%%%%%%%%%%%%%%%%%%%%%%%%%%%%%%%%%%%%%%%%%%%%%%%

\noindent%
Numerous different applications in engineering can be mathematically described by so called
\textit{inverse problems}. All applications, where an initial quantity is recovered only from a set
of observations, can be described as such. One large application is \textit{tomographic
	reconstruction}. There, the observed measurements are always a finite set of projections.
Tomographic reconstruction covers important imaging modalities such as X-ray attenuation computed
tomography (CT). CT scanners based on X-ray attenuation have been a major innovation in medical
diagnostics. It provides a way to non-intrusively obtain images from the inside of objects. However,
it is known that X-ray attenuation based imaging modalities lack soft-tissue contrast
\cite{pfeiffer_phase_2006}. Other imaging modalities can overcome this issue, on example is
differential phase-contrast CT.

No matter the exact imaging modalities, in order to find a solution to the inverse problem in the
field of tomographic reconstruction, the problem domain has to be discretized. A common approach is
the \textit{series expansion}. There the original signal, is described by a linear combination of a
coefficient vector with a set of basis functions. Arguably, the most common choice of basis function
is the pixel (or voxel) basis function. Other alternatives have been proposed, such as
\textit{spherically symmetric} (blob)~\cite{lewitt_multidimensional_1990} or \textit{B-Spline}
~\cite{unser_fast_1991} basis function. Especially, spherically symmetric basis function have been
popular in tomographic reconstruction. Both basis function provide a higher accuracy compared to
voxel basis functions. But importantly, they are continuously differentiable (at least under certain
conditions). This is an important property, especially in the context of differential phase-contrast
CT\@.

Another common ground for tomographic reconstruction, apart from the reconstruction from
projections, is the mathematical description of the, so called, \textit{forward model}. The forward
model is the description of the process of acquisition of the projections. For all mentioned method
throughout this thesis, this model is based on the line integral. I.e.\ the projections are acquired
from a (possibly infinite) set of infinitely thin lines tracing through the desired problem domain.

As with the problem domain itself, the line integral also has to be discretized. Special software,
often referred to as projectors, handles the computation of the line integral. Most commonly used
projection software inherently assume the voxel basis function as a discretization model of the
problem domain. However, this creates issues for imaging modalities that assume a differentiable
signal in the problem domain, such as differential phase-contrast CT.

In the scope of this thesis, two such projectors are implemented for the tomographic reconstruction
framework \textit{elsa}. One projector is based on spherically symmetric and the other on B-Spline
basis functions. Both projectors significantly outperform proven projector methods such as proposed
by \citeauthor*{siddon_fast_1985}~\cite{siddon_fast_1985} or
\citeauthor*{joseph_improved_1982}~\cite{joseph_improved_1982} in terms of accuracy. However, due
the overlapping nature of the basis functions, the computational burden is increased. This lays the
foundation for first-class support of differential phase-contrast CT in elsa.
