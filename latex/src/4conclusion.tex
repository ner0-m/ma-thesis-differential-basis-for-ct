\chapter{Summary}\label{chap:summary}

This thesis provides an overview of the three important aspects of tomographic reconstruction, the
forward model, the discretization and finding the solution.

The forward model of X-ray attenuation CT is discussed, both a practical and engineering focused
view is presented followed by a mathematical centered discussion. Similarly, but with a little less
detail, this was discussed for differential phase-contrast CT\@. Further, analytical and iterative
reconstruction algorithm are discussed. Many important steps for tomographic reconstruction are
covered.

The discretization is discussed in depth. Specifically, the discretization via the series expansion
is explained along side 3 different basis functions: the voxel, the family of spherically-symmetric
(blob) and B-Spline basis functions. The blob basis function as proposed by
\citeauthor*{lewitt_multidimensional_1990} and B-Spline basis functions provide a better
discretization of a given signal compared to the voxel basis function. Blob basis functions are
quite well studied and well understood in the context of tomographic reconstruction. Notably, they
provide a closed form solution for the Radon Transform and are differentiable. However, they also
are parameterized using \(3\) parameters, which can become rather complex to optimize.

On the other hand, cubic B-Splines are relatively simple. They are also differentiable, but they do
not have a analytic solution for the X-ray transform and need approximation. Interestingly though,
they still have comparable accuracy with blobs. And together with their simple nature, they are an
interesting choice of basis function for tomographic reconstruction.

The practical part of this thesis implements two projectors and contributes them to the C++
framework elsa. One is based on the blob basis function and the other on cubic B-Splines. Hence,
both use differential basis function. This should enable the usage of these projectors in the
context of differential phase-contrast CT\@.

The forward projection of both projectors is qualitatively better than voxel based methods such as
Siddon's or Joseph's method. In simple to medium complex synthetic phantoms, the new projectors do
not exhibit of high-frequency artifacts in the reconstruction as well. Not only that, but they also
better reconstruction in cases where the measurements were noisy.  The B-Spline projector does show
artifacts in certain scenarios and the blob based projector has a drop in the error metrics for the
most complex synthetic phantom. Still, they both are superior choices, if accuracy is of high
importance.

From a computational standpoint, both projectors have considerably longer runtime than both the
Siddon's and Joseph's method. This is especially true for the three-dimensional case. In the
two-dimensional case there is around a six times performance penalty to use the new projectors. As,
instead of \(1\) or \(2\) voxels (for the Siddon's or Joseph's respectively) \(5\) voxels are
visited for both basis functions, this is what is to be expected. However, in the three-dimensional
case, the penalty riced to around \(22\) times. As now around \(25\) voxels are visited for each
voxel Siddon's method visits.

\chapter{Future Work}\label{chap:future_work}

The elephant in the room for future work --- at least for me --- is support for differential
phase-contrast in elsa. Much of this work is dedicated to emphasize the importance of differential
phase-contrast, but due to the lacking support, no experiments could be run using the newly
implemented methods.

With this thesis, a part of the forward model is already done. There are now projectors based on
differential basis functions. However, at least in elsa, the exact analytical formulations for the
derivatives is not implemented yet. This most likely is the largest piece left. A new projector can
easily be build around that with the code already present. This would provide a basic support. Other
aspects of the reconstructions might need special handling or special care for the reconstruction of
differential phase-contrast images. This might include specific reconstruction algorithms or pre-
and post-processing.

Another aspect, which this thesis did not touch on, is the signal extraction from grating-based
imaging setups. As described in \autoref{sec:phasecontrast_ct}, grating-based setups can measure
attenuation, phase-shift and scattering. There, it is described that the attenuation is measured as
a drop in average intensity during the stepping. The phase-shift as a shift of the signal, and
scattering as a decreased amplitude. However, in a real measurement these three are intermixed and
not easily split apart. Hence, reading a raw measurement from a grating-based system requires
processing until the differential phase-contrast signal is extracted and can be reconstructed. To
provide first-class support for differential phase-contrast, this needs to be dealt with in some
way.

A topic only briefly touched on in this thesis, is (anisotropic) X-ray dark-field tomography. A
similar processing step as for differential phase-contrast is necessary to retrieve the dark-field
signal from the raw measurements. It is a extremely interesting topic and deserves much attention.
Hopefully elsa will have proper support for it sooner rather than later.

Another aspects, which concerns itself more with details of the projectors, is the exact
implementation of the new projectors. The current implementation is ray-driven. Both
\citeauthor*{kohler_iterative_2011} in~\cite{kohler_iterative_2011} and
\citeauthor*{momey_spline_2015} in~\cite{momey_spline_2015} propose footprint based methods for blob
and B-Spline basis functions respectively. Based on the improvements of footprint based methods over
ray-driven methods using the voxel-basis functions provides (c.f.\ \cite{long_3d_2010}), the
footprints based method could increase the accuracy and/or improve performance of the projectors.
Another interesting investigation would be the blob based distance driven method, as it is proposed
by \citeauthor*{levakhina_distance-driven_2010}~\cite{levakhina_distance-driven_2010}.

As one can imagine the runtime performance of the newly implemented projectors can be improved. For
one, the current code allocates memory each iteration, which is a high cost and might even make
reduce the runtime by around \(30\%\) to \(50\%\). Please note, that these numbers are only based on
some preliminary profiling conducted. Another lacking feature is GPU support. Tomographic
reconstruction is a prime example for the performance GPU's bring to the table. And so the
projectors implemented here would also benefit of it.

\begin{flushright}
	With that said, there is much left to do, let's get to work!
\end{flushright}
