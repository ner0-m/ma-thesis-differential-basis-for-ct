\chapter{Summary}\label{chap:summary}

This thesis provides an overview of the three important aspects of tomographic reconstruction, the
forward model, the discretization and finding the solution.

The forward model of X-ray attenuation CT was discussed, both a more practical or engineering
focused view is presented and a more mathematical discussion is presented. Similarly, but with a
little less detail this was discussed for differential phase-contrast CT\@. Also analytical and
iterative reconstruciton algorithm are discussed. Many important steps for tomographic
reconstruction are covered.

The discretization is discussed in depth. The family of spherically-symmetric basis functions --- so
called blobs --- proposed by \citeauthor*{lewitt_multidimensional_1990} and B-Spline basis functions
provide a better discretization of a given signal compared to the voxel basis function. Blob basis
functions are quite well studied and well understood in the context of tomographic reconstruciton.
Notably, they provide a closed form solution for the Radon Transform and are differentiable.
However, they also are parameterized using \(3\) parameters, which can become rather complex to
optimize.

On the other hand, cubic B-Splines are relatively simple. They are also differentiable, but they are
not closed under the X-ray transform and need approximation. Interestingly though, they have
comparable accuracy with blobs, even thou they use an approximation. With regard to the future work,
I'm exited to keep investigation this direction.

The practical part of this thesis presents two new projectors and contributes them to the C++
framework elsa. Both projectors use differential basis functions as a basis for the approximation of
the Radon Transform. This is advantages in settings where not only the Radon Transform, but also its
derivative is necessary. Such a application is differential phase-contrast CT\@.

The forward projection of both projector is qualitative better than voxel based methods such as
Siddon's or Joseph's method. In simple to medium complex synthetic phantoms, the new projectors do
not exhibit of ring like artefacts. Not only that, but they also better reconstruction in cases
where the measurements were noisy.

From a computational standpoint, both projectors fall behind both the Siddon's and Joseph's method.
This is especially true for the three-dimensional case. In the two-dimensional case there is around
a six times performance penalty to use the new projectors. As, instead of \(1\) or \(2\) voxels (for
the Siddon's or Joseph's respectively) \(5\) voxels are visited for both basis functions, this is
what is to be expected. However, in the three-dimensional case, the penalty riced to around \(22\)
times. As now around \(25\) voxels are visited for each voxel Siddon's method visits.

\chapter{Future Work}\label{chap:future_work}

The elephant in the room for future work --- at least for me --- is support for differential
phase-contrast in elsa. Much of this work is dedicated to emphasize the importance of differential
phase-contrast, but due to the lacking support, no experiments could be run using the newly
implemented methods.

With this thesis a part of the forward model is already done. There are now projectors based on
differential basis functions. However, at least in elsa, the exact analytical formulations for the
derivatives is not implemented yet. This most likely is the biggest amount of effort needed to get
basis support. A new projector can easily be build around that with the code already present. This
would provide a basic support. Other aspects of the reconstructions might need special handling or
special care for the reconstruction of differential phase-contrast images.

Another aspect, which this thesis did not touch on, is the signal extraction from grating-based
imaging setups. As described in \autoref{sec:phasecontrast_ct}, can measure attenuation, phase-shift
and scattering. There it is described that the attenuation is measured as a drop in average
intensity during the stepping. The phase-shift as a shift of the signal, and scattering as a
decreased amplitude. However, in a real measurement these three are mixed and are not easily split
apart. Hence, reading a raw measurement from a grating-based system requires processing until the
measured differential phase-contrast signal is extracted and can be reconstruction. To provide
first-class support for differential phase-contrast, this needs to be dealt with in some way.

A topic not deeply discussed in this thesis is (anisotropic) X-ray dark-field tomography. The
extraction of the phase-shift signal from the grating-based system is also important for
(anisotropic) X-ray dark-field imaging. However, it is a extremely interesting topic and deserves
much attention. Hopefully elsa will have proper support for it sooner rather than later.

Another aspects, which concerns itself more with details of the projectors, is the exact
implementation of the new projectors. The current implementation is ray-driven. Both
\citeauthor*{kohler_iterative_2011} in~\cite{kohler_iterative_2011} and
\citeauthor*{momey_spline_2015} in~\cite{momey_spline_2015} propose footprint based methods for blob
and B-Spline basis functions respectively. Based on the improvements of footprint based methods over
ray-driven methods using the voxel-basis functions provides (c.f.\ \cite{long_3d_2010}), the
footprints based method could increase the accuracy and/or improve performance of the projectors.
Another interesting investigation would be the blob based distance driven method, as it is proposed
by \citeauthor*{levakhina_distance-driven_2010}~\cite{levakhina_distance-driven_2010}.

As one can imagine the runtime performance of the newly implemented projectors can be improved. For
one, the current code allocates memory each iteration, which is a high cost and might even make
reduce the runtime by around \(30\%\) to \(50\%\). Please note, that these numbers are only based on
some preliminary profiling conducted. Another lacking feature is GPU support. Tomographic
reconstruction is a prime example for the performance GPU's bring to the table. And so the
projectors implemented here would also benefit of it.

\begin{flushright}
	With that said, there is much work left to do, let's get to work!
\end{flushright}
