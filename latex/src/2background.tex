\chapter{Notation and Terminology}\label{chap:notation}

This is my notation used thourgh the thesis

\chapter{Imaging Modalities}\label{chap:imaging_modalities}

A large number of different fields boil down to a very similar problem statement. Based on a given
measurement, how can one retrieve the original measured object, assuming the projection process is
known \cite{herman_basis_2015}. This is presicly, the definition of reconstruction used throughout
this thesis.

These fields include but are not limited to variation methods in imaging (e.g. image denoising,
impainting, super resolution and more \insertref{find reference for each}), X-ray attenuation
computed tomography, Phase-Contrast computed tomography, ligth-field tomography, electron
microscopic \insertref{find references}, and many more.

All of these problems are part of the class of so called \textit{inverse problems}. Inverse problems
in turn are again part of the so called \textit{ill-posed problems}. Following the definition of
Hadamard, \insertref{properly cite Hadamard, J. (1902) Sur les Problèmes aux Dérivées Partielles et
	Leur Signification Physique. Princeton University Bulletin, 13, 49-52.} a problem is well posed if
all of the following properties are fulfilled:

\begin{itemize}
	\item \textbf{Existence}: The problem has a solution.
	\item \textbf{Uniqueness}: The solution is the only solution.
	\item \textbf{Stability}: The solution depends continuously on the data.
\end{itemize}

A problem, which fulfills all of the above is referred to as \textit{well-posed}. If any doesn't
hold it's called \textit{ill-posed}. The classes of problems referred to above, are basically all
ill-posed.

\paragraph{Existence}

If a problem does not have any solution, a reformulation can often yields a desired result. Instead
of the original problem statement, the equivalent least squares problem can be solved to find a
unique solution. \todo{find a nice example, see \cite{Hansen_discrete_2010}}

\paragraph{Uniqueness}

If a problem has multiple - or infinitely many solutions - more requirements or preconditions can be
added, to narrow down the number of possible solutions. \todo{find a nice example, see
	\cite{Hansen_discrete_2010}}

\paragraph{Stability}

This condition is quite critical. As a violation of stability can basically mean, a small
perturbation or change in the data will result in arbitrarily large changes in the result. Again a
reformulation can help out. Usually, one adds some for of regularisation to stabilize the results.
\todo{find a nice example, see \cite{Hansen_discrete_2010}}

The specific field of tomographic reconstruction deals with the challenge to reconstruction an
image, from a finite set of projections. Specifically, most of the applications considered are
projections from X-ray sources. However, we will also look at Light field microscopy.

The common ground for all tomographic reconstruction is the Radon transform \insertref{Radon
	Transform}, or it's closely related transformations as the X-ray transform \insertref{X-ray
	transform}. The Radon transform will be properly introduced in \insertref{Def Radon transform}.

The remainder of this chapter will introduce a variety of different imaging modalities and their
applications. This will serve as a motivation for developments in the different fields. This should
also help to gain an intuition for the mathematical introduction and definitions in the following
chapters.

Physical details will not be discussed, as they are beyond the scope of this thesis. Rather, they
will be striped down and simplified to what is necessary and useful for the scope of this thesis.
However, as much as possible, resources for the interested reader are cited in the corresponding
sections.

\todo{Cite (medical) applications for each method}

\section{X-ray Attenuation CT}\label{sec:xray_attenuation_ct}

\todo{First X-ray CT stuff, Look into stuff without cutting it open}

Narrowing our observations to a single X-ray going through a 2 dimensional object, which we wish to
examinate from the inside. For now, consider the X-ray going in a straight line \(L\). The
attenuation of the X-ray along this path is based on the Beer-Lambert law \cite{buzug_computed_2008}:

\begin{equation}
	\label{eq:beer-Lambert-law}
	- \ln \frac{I}{I_0} = \int_L \mu (x) dx
\end{equation}

where the integral along the path \(L\) is referred to as the line integral.
\(\mu \colon \mathbb{R}^2 \to \mathbb{R}\) denotes the linear attenuation cofficients, \(I_0\)
the initial intensity of the beam (i.e. without any object between the source and the detector).

The challenge for X-ray attenuation CT is to reconstruct the function \(\mu\).

\section{Phase-Contrast CT}\label{sec:phasecontrast_ct}

Though, attenuation is the predominant X-ray based imaging method, other methods take other physical
properties of X-rays into account, such as diffraction, refraction and scattering.

As the name suggest, Phase-contrast X-ray CT considers the phase shift of the X-rays going through
an object.

Read and incorporate:
\begin{itemize}
	\item Differential phase-contrast X-ray computed tomography: From model discretization to image
	      reconstruction, Nilchan
	\item Tomographic reconstruction of three-dimensional objects from hard X-ray differential phase
	      contrast projection images
\end{itemize}


\section{Anisotropic X-ray Dark-field Tomography}\label{sec:axdt}

Once one takes also

\section{Light Field Microscopy}\label{sec:lightfield_microscopy}

Lastly, a field not based on X-rays.


\chapter{Radon Transform and its related transform}\label{chap:radon_transform_and_related}

In the previous chapter, the notion of the inverse problem was introduced. There, based on a given
measurement one tries to reconstruct the original quantity, based on a given model. This chapter is
mostly devoted to derive and analyze this model. Specifically, in the context of X-ray attenuation
CT, with a brief outlook into other areas needed for e.g. phase-contrast CT.

The physical model is the link between the unknown signal and it's measurements. Usually, the model
is referred to as forward model. Let us first state a couple of basic definitions and then we will
look into the derivation of the forward model for X-ray attenuation CT.

\begin{definition}[Image]\label{def:image}
	Let \(f\colon \mathbb{R}^n \to \mathbb{R}\) be a \(n\)-dimensional continuous function, whose support is
	bounded. It is referd to it as a \(n\)-dimensional image. And often it will be referred to as image,
	without the special mention of \(n\)-dimensional.
\end{definition}

\inlinetodo{define cartesion grid for images}

In the special case of \(n=2\), it is called an \textit{2-dimensional image} or just
\textit{2D image}. However, if from the context, the \(2\)-dimensional is obvious, it will still be
referred to as image. In case of \(n=3\), it is called a \textit{volume}.

\begin{definition}[Forward Model]\label{def:forward-model}
	To reconstruct an unknown image \(f\), a set of \(J\) scalar measurements is necessary.
	A single scalar valued measurement \(m_j\), with \(j \in \{1, \dots, J\}\) is defined in therms
	of the pshyical model:
	\[ m_j = \mathscr{M}_j(f)\]
	where
	\[ \mathscr{M}_j\colon (\Omega \to \mathbb{R}) \to \mathbb{R} \]
	and \(\Omega \subseteq \mathbb{R}^n\). The mapping is required to be linear, as this will play an
	important role later on.
\end{definition}

This is a very general definition. This is useful to mathematically model a wide variety of
different applications. In fact, all of the applications in the previous chapter, can be modeled
using this general definition. And there are many more, which can be modeled this way.

\section{Radon Transform}\label{sec:radon_transform}

Looking at the previously stated Beer Lambert law for attenuation CT \autoref{eq:beer-Lambert-law}.
One can see, that the measured values, are somehow related to a line integral. The mathematical
model, one can use was first defined by Johann Radon in 1917 \insertref{Radon transform}, and
later rediscovered by Allan M. Cormack \insertref{CT Allan M. Cormack}.

\begin{definition}[Radon Transform]
	Again, let \(\Omega \subset \mathbb{R}^n\) and an image \(f\colon \Omega \to \mathbb{R}\),
	which is assumed to sufficiently nice. Then the mapping \(\mathscr{R}f\colon (\mathbb{R}^n
	\to \mathbb{R}) \to (\mathbb{R} \times \mathbb{S}^{n-1} \to \mathbb{R})\) of \(f\), maps
	\(f\) into the set of its integrals over the affine hyperplanes of \(\mathbb{R}^(n-1)\).
	Specifically, given a hyperplane \(\mathbb{H}^{n-1}(s, u)\) with \(u \in \mathbb{S}^{n-1}\)
	a normal vector and \(s \in \mathbb{R}\) the offset to the origin, then the (n-dimensional)
	\textbf{Radon Transform} is given by:
	\[ \mathscr{R}f(s, u) = \int_{\mathbb{H}^{n-1}(s, u)} f(v) \, dv \]

	See \insertref{The Mathematics of computed tomography}
\end{definition}

Note that for \(n=2\) the hyperplanes are lines, and hence match the forward model for X-ray
imaging. For the 3-dimensional case, this does not fit anymore. For this case, the so called
X-ray transform \(\mathscr{X}\) was developed \insertref{find X-ray transform reference}, which is
very similar to the Radon transform, but for all dimensions only considers integral along lines.

The Radon transform is a linear operator, which is pseudo-commuting with convolution (, invariant to
translations. \todo{find more important properties of the radon transform} \todo{inverse radon
	transform}

Of further interest to us is the first derivative of the Radon transform.

\todo{fourier slice theorem and such should go here?}

\section{X-ray Transform}\label{sec:xray_transform}

\section{Abel Transform}\label{sec:abel_transform}

\chapter{Image Representation}\label{chap:image_representation}

Images as defined in \ref{def:image}, are continuous functions. However, one wishes to use computers
to solve the reconstruction tasks and computers are inherently discrete. Hence, one wishes to
represent an image in a discrete fashion.

\begin{definition}[Permissible representation]
	\label{def:permissible_representation}
	Let \(N \in \mathbb{N}\) be a positive integer and \(\varphi_n\) a set basis function for
	\(1 \leq n \leq N\), then the signal \(f\) can be approximated as a linear combinations
	of these basis functions and the coefficients \(c_n\):
	\[ \hat{f}(x) = \sum_{k=1}^{N} c_k \varphi_k(x) \]
\end{definition}

For our purposes, we assume the function lies on a regular spaced discrete grid. Then, let
\(\varphi\) be a zero centered symmetrical basis function, \(\symbfit{k} \in \mathbb{Z}^n\) be the
\(n\)-dimensional index of a grid cell, and \(x_{\symbfit{k}} \in \mathbb{R}^n\) the center coordinate
of the \(\symbfit{k}\)-th grid cell. Then, the previous equation can be reformulated:
\[ \hat{f}(x) = \sum_{\symbfit{k} \in \mathbb{Z}^n} c_{\symbfit{k}} \varphi(x - x_{\symbfit{k}}) \]
This definition follows the notation given in \cite{momey_new_2011}.


Now, if one applies the Radon transformation to the discretized image: \todo{generalize to all linear physical models}
\[ \mathscr{R}\hat{f}(x) = \mathscr{R}\left( \sum_{\symbfit{k} \in \mathbb{Z}^n} c_{\symbfit{k}} \varphi(x - x_{\symbfit{k}}) \right) \]
Due to the linearity of the Radon Transform this is equivalent to \[ \mathscr{R}\hat{f}(x) =  \sum_{\symbfit{k} \in \mathbb{Z}^n} c_{\symbfit{k}}\mathscr{R}\left( \varphi(x - x_{\symbfit{k}}) \right) \]
i.e. the Radon transformation of the image, only act upon the basis function. Hence, it is
sufficient to study, how the Radon transformation acts upon the individual basis function.

Note that this holds for any linear physical model. Notably, this holds for the X-ray transform and
the first derivative of the Radon transform. But it is also true for the physical models behind the
applications discussed in the previous chapter. Hence, it is sufficient to study, how a basis
function acts under the given transformation.

\subsection{Properties of Basis Function}

\todo{maybe remove this section}

From this it is evident, that the choice of basis function is crucial for the accuracy of the
discretized image. A suboptimal representation, will yield undesired results. In
\cite{nilchian_optimized_2015}, 4 properties a basis functions should satisfy are proposed. These are:
\begin{itemize}
	\item Riesz Basis
	\item Partition Of Unity
	\item Compact Support
	\item Isotropy
\end{itemize}
Similar properties are stated in \cite{hanson_local_1985}.

\paragraph{Riesz Basis}

\cite{hanson_local_1985} formulates a similar requirement, however states it less restrictive. It is
states as a requirement for strong linear independence, rather than a unique representation.

\todo{Explain Riesz Basis, find definition of it and cite it}

\paragraph{Partition of Unity}

Some function \(g\), fulfills the property partition-of-unity if
\begin{itemize}
	\item \(g: \mathbb{R}^n \to [0, 1]\), i.e. it \(g\) maps into the unit interval
	\item \(\sum_{\symbfit{k} \in \mathbb{Z}^n} g(x + k) = 1 \; \forall x \in \mathbb{R}^n\)
\end{itemize}

Given the basis fulfills this property, the error of approximation converges to zero, with sampling
step \(\Delta\) going to zero \cite{nilchian_optimized_2015}. Formally
\[ \lim_{\Delta \to 0} \norm{f - \hat{f}}_{L_2} = 0 \]

\todo{define and explain the constraints required here}

This requirement is close to the property power of approximation and fidelity of visual
appearance in \cite{hanson_local_1985}.

\paragraph{Compact support}

The compact support is a rather practical requirement. In order to reduce the computational cost,
the function should be compact. Phrased differently, the smaller the support, fewer evaluations for
a given point are required, i.e. the number of zero elements in \(c_{\symbfit{k}}\) grows with
shrinking support.

\insertref{assumption about gaussian}
On the other hand, given a function without compact support, such as the gaussian.

\paragraph{Isotropy}

Isotropy is again a practical requirement. If a basis function is isotropic, it is projections do
not depend on the direction or angle. This greatly simplifies the implementation and improves
efficiency.

Both compact supoort and isotropy are practical requirement to fulfill the requirements of efficient
computation of forward and backward projection and implementation of reconstruction constraints by
\cite{hanson_local_1985}.

A couple of choices for different basis functions are discussed in this section. First, the most
common choice in the pixel basis function is presented. Then a closer look at two alternatives in
spherically symmetric basis functions (from now on referred to \textit{blobs}), and B-Splines.

\section{Voxel Basis}\label{sec:voxel_basis}

The most likely most well known basis function in imaging is the pixel or voxel basis functions. The
voxel basis function is a piecewise linear function. The voxel basis function is most likely the
most widely used basis function. Most literature assumes the voxel basis function implicitly.

The centered voxel basis function of step width \(h\), is given by:
\begin{equation}\label{eq:voxel_basis_fn}
	\varphi^{\text{pixel}}(\symbfit{x}) =
	\begin{cases}
		1, \abs{\symbfit{x}} < \frac{h}{2} \\
		0, \text{otherwise}
	\end{cases}
\end{equation}
Here, the absolute value is coefficient wise, as soon as the absolute value of any coefficient of
the vector \(\symbfit{x} \in \mathbb{R}^n\) is larger half the step size, the function will return
\(0\).

An image approximated by the voxel basis function, in the series expansion method is equivalent to
the nearest neighbourhood interpolation.

The analytical formulation of the Radon transform of the pixel basis function
(compare~\insertref{Toft 1996}) is given by:
\begin{equation}\label{eq:radon_voxel_basis}
	\mathscr{R}\varphi^{\text{pixel}}(\rho, \theta) =
	\begin{cases}
		0                                                  & x_1 > 0                         \\
		\sqrt{4 + (x_1 - x_{-1})^2} = \frac{2}{\cos\theta} & x_1 < 1\;\text{and}\;x_{-1} < 1 \\
		\sqrt{(1 - x_1)^2 + (1 - x_{-1})^2}                & x_1 < 1\;\text{and}\;x_{-1} > 1
	\end{cases}
\end{equation}

\todo{understand this properly and explain this properly}

\section{Blob Basis}\label{sec:blob_basis}

First introduced by Lewitt in \insertref{Reference to 1990 Lewitt paper}, spherically symmetric
volume elements (often referred to as blobs) are an alternative to the pixel basis.
~\cite{lewitt_alternatives_1992} describes how blobs can be used in iterative reconstruction
algorithms as a basis instead of pixels.

Blob basis functions have been adopted in many different fields. Among others electron
microscopy~\cite{marabini_3d_1998, garduno_optimization_2001}, poistron emission tomography
(PET)~\cite{jacobs_comparative_1999, chlewicki_noise_2004}, single-photon emission tomography
(SPECT)~\cite{wang_3d_2004, yendiki_comparison_2004}, attenuation X-ray
CT~\cite{jacobs_iterative_1999, carvalho_helical_2003, isola_motion-compensated_2008},
phase-contrast CT~\cite{kohler_iterative_2011, xu_investigation_2012}, reconstruction of coronary
trees~\cite{zhou_blob-based_2008}, breast tomosynthesis~\cite{wu_breast_2010}, reduction of metal
artifacts~\cite{levakhina_two-step_2010} or computed laminography~\cite{trampert_spherically_2017}.

\inlinetodo{Read papers and assert what blobs brings to the table}

Generally, many fields report increased accuracy with a comparable performance. In other fields,
such as phase-contrast CT, blobs enable the usage of iterative reconstructions without an extra step
of numerical differentiations.

The generalized Kaiser-Bessel basis function as proposed by Lewitt, is defined as:
\begin{equation}\label{eq:blob_basis_fn}
	\varphi^{\text{blob}}_{m, \alpha, a}(r) =
	\begin{cases}
		\frac{I_m\left( \alpha \sqrt{1 - \left(\frac{r}{a}\right)^2} \right)} {I_m\left( \alpha \right)} \left( \sqrt{1 - \left(\frac{r}{a}\right)^2}\right)^m & 0 \le r \le a      \\
		0                                                                                                                                                      & \textit{otherwise}
	\end{cases}
\end{equation}
where \(I_m\) is the modified Kaiser-Bessel function of the first kind of order \(m\), \(r\) the
distance to the blob center, \(a\) the blob radius given in units of the grid, and \(\alpha\)
controlling the shape of the blob. \(m\) controls the continuity of the blob function.
\todo{Figures showing different parameters of blob}

The X-ray transform of the blob basis function is given by
(c.f.~\cite{lewitt_multidimensional_1990,lewitt_alternatives_1992})
\begin{align}\label{eq:radon_blob_basis}
	p(s) & = 2 \int_0^{(a^2-s^2)^{1 / 2}} \varphi^{\text{blob}}_{m, \alpha, a}\left(\left(s^2 - t^2\right)^{1/2}\right) \diff t                                                                                \\
	     & = \frac{a}{I_m(\alpha)} \left( \frac{2\pi}{\alpha}\right)^{1/2} \left( \sqrt{1 - \left(\frac{s}{a}\right)^2} \right)^{m + 1/2} I_{m+1/2}\left( \alpha \sqrt{1 - \left(\frac{s}{a}\right)^2} \right)
\end{align}
\(s\) is the distance from the X-ray to the blob center, and \(\sqrt{a^2 - s^2}\) is one half of the
intersection length btween the blob and the ray. The projected value only depends on the distance
from the X-ray to the blob center. This is a very nice property. This makes implementations quite
efficient.

\inlinetodo{figure for parameters for projected basis, figure for parameters for basis}

From an implementational standpoint, the half integer order of the modified Kaiser-Bessel function
of the first kind, can be quite nasty. Implementations do exist as it can be seen
in~\cite{temme_numerical_1975}. However, the floating point implementations are non-trivial. Plus,
for the case of C++, since C++17 the standard library provides mathematical special functions
~\cite{noauthor_c_nodate-3, noauthor_stdcyl_bessel_i_nodate}. But sadly, it is not yet entirely
cross platform, as it is only supported by libstdc++~\cite{noauthor_libstdc_nodate-1}, and not
libc++. However, for our cases it is sufficient to assume \(m \in \mathbb{N}\). Then the above
equation can be further simplified.

The recurrence formulation for the modified Kaiser-Bessel function of the first kind is
(c.f.~\cite[chapter 9]{abramowitz_handbook_1972}):
\begin{equation}\label{eq:kaiser_bessel_recurrence}
	I_{m+1}(x) = I_{m-1}(x) - \frac{2 m}{x}I_m(x)
\end{equation}
Further, the Kaiser-Bessel functions have representations with elementary functions. For the
modified Kaiser-Bessel function of the first kind, there are defined as (c.f.~\cite[chapter 10]{abramowitz_handbook_1972}):
\begin{align}\label{eq:kaiser_bessel_half_integer}
	I_{0.5}(x) & = \sqrt{\frac{2}{\pi x}} \sinh(x)                                                                               \\
	I_{1.5}(x) & = \sqrt{\frac{2}{\pi x}} \left( \cosh(x) \frac{\sinh(x)}{x} \right)                                             \\
	I_{2.5}(x) & = \sqrt{\frac{2}{\pi x}} \left(\left(\frac{3}{x^2} + \frac{1}{x}\right)\sinh(x) - \frac{3}{x^2} \cosh(x)\right)
\end{align}
Then \autoref{eq:radon_blob_basis} can be simplified to not include any non-integer evaluations of
the modified Kaiser-Bessel function of the first kind. For example assuming, \(m = 0\), and to keep
everything a little more concise, let \(w = \sqrt{1 - \left(\frac{r}{a}\right)^2}\):
\begin{align}\label{eq:radon_blob_basis_order_0_simplified}
	p(s) & = \frac{a}{I_0(\alpha)} \left(\frac{2\pi}{\alpha}\right)^{1/2} \left( w \right)^{1/2} I_{1/2}\left( \alpha w \right)                     \\
	     & = \frac{a}{I_0(\alpha)} \left(\frac{2\pi w}{\alpha}\right)^{1/2} I_{1/2}\left( \alpha w \right)                                          \\
	     & = \frac{a}{I_0(\alpha)} \left(\frac{2\pi w}{\alpha}\right)^{1/2} \left( \frac{2}{\pi \alpha w}\right)^{1/2} \sinh \left(\alpha w \right) \\
	     & = \frac{2 a}{\alpha I_0(\alpha)} \sinh \left(\alpha w \right)
\end{align}
In the last step, \(\pi\) and \(w\) cancel out, and both the \(2^2\) and \(\alpha^2\) ared moved out
of the square root, leaving it empty. Similar operations can be done for \(m = 1\) and \(m = 2\).

\section{B-Spline Basis}\label{sec:bspline_basis}

Splines are common in image and signal processing~\cite{unser_splines_1999}. Applications include
image interpolation, image transformations, image compressions or the calculation of the first and
second derivative. A common approach is the approximation of the function or image using Splines and
then working efficiently on the continuous representation of the splines. For B-Splines
specifically~\cite{unser_fast_1991} shows the continuous image representation using B-Splines.

This approach was adopted to tomographic reconstruction.~\cite{la_riviere_spline-based_1998}
proposed the calculation of the inverse 2D and 3D Radon transform based on B-Spline.
Similarly,~\cite{horbelt_discretization_2002} develops a B-Spline based filtered backprojection.
Apart from attenuation CT, other medical applications of B-Splines include electron
tomography~\cite{tran_robust_2013, tran_inverse_2014}, positron emission tomography
(PET)~\cite{nichols_spatiotemporal_2002, li_fast_2007, verhaeghe_investigation_2007} and
single-photon emission tomography (SPECT)~\cite{guedon_b-spline_1991, reutter_fully_2007}.

Using B-Splines as a basis function was first presented by~\cite{momey_new_2011,
momey_b-spline_2012, momey_spline_2015}. And a similar image representation was adapted for
phase-contrast CT in~\cite{nilchian_fast_2013, nilchian_differential_2012, nilchian_spline_2015}.
They differ in the approximation of the evaluiation of the X-ray transform. The former use a
footprint of the B-Splines, where the later rely on the first derivative of the B-Spline basis
function.

\begin{definition}[Recursive B-Spline]
	\label{def:bspline}
	The centered B-Spline of degree $d$ and of width $\Delta$ is given by
	\begin{align*}
		\beta_\Delta^0(x) & = \mu(x) =
		\begin{cases}
			\frac{1}{\Delta}, & \text{if } x \in [-\frac{\Delta}{2}, \frac{\Delta}{2}] \\
			0,                & \text{otherwise}
		\end{cases} \\
		\beta_\Delta^d(x) & = \beta_\Delta^0 * \beta_\Delta^{d-1}(x) =
		\underbrace{\beta_\Delta^0 * \dots * \beta_\Delta^0(x)}_{d+1 \text{concolution terms}}
	\end{align*}
	where $\mu(x)$ is the step function centered around 0. \todo[inline]{I'm not sure if this is sound}

	The zero-dimensional B-Spline, is the normalized unit impulse of width $\Delta$. And the
	$d$-dimensional B-Spline is just the $d+1$ fold convolution of the normalized unit impulses
	\cite{horbelt_discretization_2002}.

	Note that
	\[ \beta_\Delta^d(x) = \frac{1}{\Delta} \beta_1^d(\frac{x}{\Delta}) \]
	therefore, if no specific subscript is mentioned, the B-Spline with $\Delta = 1$ is implied.

	In \cite{unser_fast_1991}, a non-recursive definition of the unit width B-Spline is given as:
	\[ \beta^d = \sum_{j=0}^{n+1} \frac{(-1)^j}{n!} \binom{n+1}{j}(x - j)\mu(x - j) \]
\end{definition}

B-Splines have a couple of really attractive properties. B-Splines are the shortest and smoothest
scaling functions for a given order of approximation \cite{momey_b-spline_2012}. This
are close to a Gaussian function,
with a sufficiently large $d$ \cite{momey_b-spline_2012}, all while preserving compactness.

In \cite{horbelt_discretization_2002}, it was shown that the Radon transform of B-Splines are spline
bikernel. \cite{entezari_box_2012} shows how expliclity how (tensor product) B-Splines act under
the X-ray and Radon transform. \cite{nilchian_differential_2012} shows how B-Splines act under
the first derivative of the Radon transform.

\chapter{Tomographic Reconstruction}\label{chap:tomographic_reconstruction}

\section{Analytical Reconstruction}\label{sec:analytical_reconstruction}

\section{Towards the Matrix form}\label{sec:matrix_formulation}

\section{Iterative Reconstruction}\label{sec:iterative_reconstruction}

\subsection{ART}\label{subsec:algebraic_reconstruction_technique}

\begin{listing}
	\begin{minted}{cpp}
int main() {
    fmt::print("hello, world\n");
    return 0;
}
    \end{minted}
	\caption{"Some sampe C code"}
\end{listing}
\begin{listing}
	\begin{minted}{python}
import numpy as np

np.linspace(0, 1)
    \end{minted}
	\caption{"Some sampe python code"}
\end{listing}

ART and derivatives

\subsection{CG}\label{subsec:conjuage_gradient}

CG and such

\subsection{First-order methods}\label{subsec:first_order_methods}

First orther methods such as Gradient Descent and it's derivatives

\section{Regularization}\label{sec:regularization}

\subsection{Tikonov}\label{subsec:tikhonov_regularization}

\subsection{LASSO L1}\label{subsec:l1_regularization}

\subsection{TV Regularization}\label{subsec:tv_regularization}
