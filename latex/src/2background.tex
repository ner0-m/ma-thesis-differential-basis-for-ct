\chapter{Notation and Terminology}

This is my notation used thourgh the thesis

\chapter{Imaging Modalities}

A large number of different fields boil down to a very similar problem statement. Based on a given
measurement, how can one retrieve the original measured object, assuming the projection process is
known \cite{herman_basis_2015}. This is presicly, the definition of reconstruction used throughout
this thesis.

These fields include but are not limited to variation methods in imaging (e.g. image denoising,
impainting, super resolution and more \insertref{find reference for each}), X-ray attenuation
computed tomography, Phase-Contrast computed tomography, ligth-field tomography, electron
microscopic \insertref{find references}, and many more.

All of these problems are part of the class of so called \textit{inverse problems}. Inverse problems
in turn are again part of the so called \textit{ill-posed problems}. Following the definition of
Hadamard, \insertref{properly cite Hadamard, J. (1902) Sur les Problèmes aux Dérivées Partielles et
Leur Signification Physique. Princeton University Bulletin, 13, 49-52.} a problem is well posed if
all of the following properties are fulfilled:

\begin{itemize}
    \item \textbf{Existence}: The problem has a solution.
    \item \textbf{Uniqueness}: The solution is the only solution.
    \item \textbf{Stability}: The solution depends continuously on the data.
\end{itemize}

A problem, which fulfills all of the above is referred to as \textit{well-posed}. If any doesn't
hold it's called \textit{ill-posed}. The classes of problems referred to above, are basically all
ill-posed.

\paragraph{Existence}

If a problem does not have any solution, a reformulation can often yields a desired result. Instead
of the original problem statement, the equivalent least squares problem can be solved to find a
unique solution. \todo{find a nice example, see \cite{Hansen_discrete_2010}}

\paragraph{Uniqueness}

If a problem has multiple - or infinitely many solutions - more requirements or preconditions can be
added, to narrow down the number of possible solutions. \todo{find a nice example, see
\cite{Hansen_discrete_2010}}

\paragraph{Stability}

This condition is quite critical. As a violation of stability can basically mean, a small
perturbation or change in the data will result in arbitrarily large changes in the result. Again a
reformulation can help out. Usually, one adds some for of regularisation to stabilize the results.
\todo{find a nice example, see \cite{Hansen_discrete_2010}}

The specific field of tomographic reconstruction deals with the challenge to reconstruction an
image, from a finite set of projections. Specifically, most of the applications considered are
projections from X-ray sources. However, we will also look at Light field microscopy.

The common ground for all tomographic reconstruction is the Radon transform \insertref{Radon
Transform}, or it's closely related transformations as the X-ray transform \insertref{X-ray
transform}. The Radon transform will be properly introduced in \insertref{Def Radon transform}.

The remainder of this chapter will introduce a variety of different imaging modalities and their
applications. This will serve as a motivation for developments in the different fields. This should
also help to gain an intuition for the mathematical introduction and definitions in the following
chapters.

Physical details will not be discussed, as they are beyond the scope of this thesis. Rather, they
will be striped down and simplified to what is necessary and useful for the scope of this thesis.
However, as much as possible, resources for the interested reader are cited in the corresponding
sections.

\todo{Cite (medical) applications for each method}

\section{X-ray Attenuation CT}

\todo{First X-ray CT stuff, Look into stuff without cutting it open}

Narrowing our observations to a single X-ray going through a 2 dimensional object, which we wish to
examinate from the inside. For now, consider the X-ray going in a straight line \(L\). The
attenuation of the X-ray along this path is based on the Beer-Lambert law \cite{buzug_computed_2008}:

\begin{equation}
    \label{eq:beer-Lambert-law}
    - \ln \frac{I}{I_0} = \int_L \mu (x) dx
\end{equation}

where the integral along the path \(L\) is referred to as the line integral.
\(\mu \colon \mathbb{R}^2 \to \mathbb{R}\) denotes the linear attenuation cofficients, \(I_0\)
the initial intensity of the beam (i.e. without any object between the source and the detector).

The challenge for X-ray attenuation CT is to reconstruct the function \(\mu\).

\section{Phase-Contrast CT}

Though, attenuation is the predominant X-ray based imaging method, other methods take other physical
properties of X-rays into account, such as diffraction, refraction and scattering.

As the name suggest, Phase-contrast X-ray CT considers the phase shift of the X-rays going through
an object.

\section{Anisotropic X-ray Dark-field Tomography}

Once one takes also

\section{Light Field Microscopy}

Lastly, a field not based on X-rays.



\chapter{Tomographic Reconstruction}

This chapter will be devoted to lay a common notation for the remainder of this thesis. First,
a couple of basic definitions will be presented.

\begin{definition}[Image]
    \label{def:image}
    Let \(f\colon \mathbb{R}^n \to \mathbb{R}\) be a \(n\)-dimensional continuous function, whose support is
    bounded. It is referd to it as a \(n\)-dimensional image. And often it will be referred to as image,
    without the special mention of \(n\)-dimensional.
\end{definition}

In the special case of \(n=2\), it is called an \textit{2-dimensional image} or just
\textit{2D image}. However, if from the context, the \(2\)-dimensional is obvious, it will still be
referred to as image. In case of \(n=3\), it is called a \textit{volume}.

\section{Radon transform}

First talk about the radon transform, it's properties and it's related transformations,
such as the X-ray transform.

Then later, we can say that the Radon transform of an image only acts on the basis functions.

\section{Image Representation}

Images as defined in \ref{def:image}, are continuous functions. However, one wishes to use computers
to solve the reconstruction tasks and computers are inherently discrete. Hence, one wishes to
represent an image in a discrete fashion.

\begin{definition}[Permissible representation]
    \label{def:permissible_representation}
    Let \(N \in \mathbb{N}\) be a positive integer and \(\varphi_n\) a set basis function for
    \(1 \leq n \leq N\), then the signal \(f\) can be approximated as a linear combinations
    of these basis functions and the coefficients \(c_n\):
    \[ \hat{f}(x) = \sum_{k=1}^{N} c_k \varphi_k(x) \]
\end{definition}

For our purposes, we assume the function lies on a regular spaced discrete grid. Then, let
\(\varphi\) be a zero centered symmetrical basis function, \(\symbfit{k} \in \mathbb{Z}^n\) be the
\(n\)-dimensional index of a grid cell, and \(x_{\symbfit{k}} \in \mathbb{R}^n\) the center coordinate
of the \(\symbfit{k}\)-th grid cell. Then, the previous equation can be reformulated:
\[ \hat{f}(x) = \sum_{\symbfit{k} \in \mathbb{Z}^n} c_{\symbfit{k}} \varphi(x - x_{\symbfit{k}}) \]
This definition follows the notation given in \cite{momey_new_2011}.


Now, if one applies the Radon transformation to the discretized image: \todo{generalize to all
linear physical models}
\[ \mathscr{R}\hat{f}(x) = \mathscr{R}\left( \sum_{\symbfit{k} \in \mathbb{Z}^n} c_{\symbfit{k}} \varphi(x -
x_{\symbfit{k}}) \right) \]
Due to the linearity of the Radon Transform this is equivalent to
\[ \mathscr{R}\hat{f}(x) =  \sum_{\symbfit{k} \in \mathbb{Z}^n} c_{\symbfit{k}}\mathscr{R}\left( \varphi(x -
x_{\symbfit{k}}) \right) \]
i.e. the Radon transformation of the image, only act upon the basis function. Hence, it is
sufficient to study, how the Radon transformation acts upon the individual basis function.

Note that this holds for any linear physical model. Most notably, for the content of this
thesis, this holds for the X-ray transform. But it is also true for the physical models behind
the applications discussed in the previous chapter.

\subsection{Properties of Basis Function}

\todo{maybe remove this section}

From this it is evident, that the choice of basis function is crucial for the accuracy of the
discretized image. A suboptimal representation, will yield undesired results. In
\cite{nilchian_optimized_2015}, 4 properties a basis functions should satisfy are proposed. These are:
\begin{itemize}
    \item Riesz Basis
    \item Partition Of Unity
    \item Compact Support
    \item Isotropy
\end{itemize}
Similar properties are stated in \cite{hanson_local_1985}.

\paragraph{Riesz Basis}

\cite{hanson_local_1985} formulates a similar requirement, however states it less restrictive. It is
states as a requirement for strong linear independence, rather than a unique representation.

\todo{Explain Riesz Basis, find definition of it and cite it}

\paragraph{Partition of Unity}

Some function \(g\), fulfills the property partition-of-unity if
\begin{itemize}
    \item \(g: \mathbb{R}^n \to [0, 1]\), i.e. it \(g\) maps into the unit interval
    \item \(\sum_{\symbfit{k} \in \mathbb{Z}^n} g(x + k) = 1 \; \forall x \in \mathbb{R}^n\)
\end{itemize}

Given the basis fulfills this property, the error of approximation converges to zero, with sampling
step \(\Delta\) going to zero \cite{nilchian_optimized_2015}. Formally
\[ \lim_{\Delta \to 0} \norm{f - \hat{f}}_{L_2} = 0 \]

\todo{define and explain the constraints required here}

This requirement is close to the property power of approximation and fidelity of visual
appearance in \cite{hanson_local_1985}.

\paragraph{Compact support}

The compact support is a rather practical requirement. In order to reduce the computational cost,
the function should be compact. Phrased differently, the smaller the support, fewer evaluations for
a given point are required, i.e. the number of zero elements in \(c_{\symbfit{k}}\) grows with
shrinking support.

\insertref{assumption about gaussian}
On the other hand, given a function without compact support, such as the gaussian.

\paragraph{Isotropy}

Isotropy is again a practical requirement. If a basis function is isotropic, it is projections do
not depend on the direction or angle. This greatly simplifies the implementation and improves
efficiency.

Both compact supoort and isotropy are practical requirement to fulfill the requirements of efficient
computation of forward and backward projection and implementation of reconstruction constraints by
\cite{hanson_local_1985}.

A couple of choices for different basis functions are discussed in this section. First, the most
common choice in the pixel basis function is presented. Then a closer look at two alternatives in
spherically symmetric basis functions (from now on referred to \textit{blobs}), and B-Splines.

\subsection{Pixel Basis}



\subsection{Blob Basis}

What ware spherically semmetrical basis elements (blobs)

Very important literature
\cite{lewitt_alternatives_1992}
\cite{matej_practical_1996}

\subsection{B-Spline Basis}

B-Splines have been used in multiple fields. \insertref{Find some different fields}.

\begin{definition}[Recursive B-Spline]
    \label{def:bspline}
    The centered B-Spline of degree $d$ and of width $\Delta$ is given by
    \begin{align*}
        \beta_\Delta^0(x) &= \mu(x) =
            \begin{cases}
                \frac{1}{\Delta}, & \text{if } x \in [-\frac{\Delta}{2}, \frac{\Delta}{2}]\\
                0,           & \text{otherwise}
            \end{cases} \\
           \beta_\Delta^d(x) &= \beta_\Delta^0 * \beta_\Delta^{d-1}(x) =
               \underbrace{\beta_\Delta^0 * \dots * \beta_\Delta^0(x)}_{d+1 \text{concolution terms}}
    \end{align*}
    where $\mu(x)$ is the step function centered around 0. \todo[inline]{I'm not sure if this is sound}

    The zero-dimensional B-Spline, is the normalized unit impulse of width $\Delta$. And the
    $d$-dimensional B-Spline is just the $d+1$ fold convolution of the normalized unit impulses
    \cite{horbelt_discretization_2002}.

    Note that
    \[ \beta_\Delta^d(x) = \frac{1}{\Delta} \beta_1^d(\frac{x}{\Delta}) \]
    therefore, if no specific subscript is mentioned, the B-Spline with $\Delta = 1$ is implied.

    In \cite{unser_fast_1991}, a non-recursive definition of the unit width B-Spline is given as:
    \[ \beta^d = \sum_{j=0}^{n+1} \frac{(-1)^j}{n!} \binom{n+1}{j}(x - j)\mu(x - j) \]
\end{definition}

B-Splines have a couple of really attractive properties. B-Splines are the shortest and smoothest
scaling functions for a given order of approximation \cite{momey_b-spline_2012}. This
are close to a Gaussian function,
with a sufficiently large $d$ \cite{momey_b-spline_2012}, all while preserving compactness.

\cite{unser_b-spline_1993}
\cite{unser_b-spline_1993-1}
\cite{unser_fast_1991}
\cite{briand_theory_2018}

\subsection{Box splines}

What are box splines

\cite{entezari_box_2012}
\cite{de_boor_box_1993}

\section{Analytical Reconstruction}
something

\section{Iterative Reconstruction}

\subsection{ART}

\begin{listing}
    \begin{minted}{cpp}
int main() {
    fmt::print("hello, world\n");
    return 0;
}
    \end{minted}
    \caption{"Some sampe C code"}
\end{listing}
\begin{listing}
    \begin{minted}{python}
import numpy as np

np.linspace(0, 1)
    \end{minted}
    \caption{"Some sampe python code"}
\end{listing}

ART and derivatives

\subsection{CG}

CG and such

\subsection{First-order methods}

First orther methods such as Gradient Descent and it's derivatives

\section{Regularization}

\subsection{Tikonov}

\subsection{LASSO L1}

\subsection{TV Regularization}

\chapter{Basis}

Explain basis, and why they are useful

\subsection{Series expansion under the Radon transform}

\todo[inline]{Write up \cite{nilchian_optimized_2015} to get to matrix notation}

\section{Voxels}

\section{Blobs}


\section{B-Splines}
