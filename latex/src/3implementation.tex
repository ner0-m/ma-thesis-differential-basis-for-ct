\chapter{elsa}

So much implementation design

\chapter{Projector}\label{chap:projector}

As shown in previous chapters, calculation of the system matrix coefficients is one of the key
components of tomographic reconstruction. Thus, this has been an important part of research. The
routines or algorithms, which calculate the matrix are usually referred to as projectors. For the
forward projection, the contribution of each voxel to each detector pixel is calculated. For the
backward projection, the contribution of each detector pixel each a voxel is computed.

In the first part of this chapter, a detailed overview over existing research and implementations of
projectors is given. Next details on the projector used for this thesis is given. This is followed
by a detailed studied of accuracy and performance of the new projector compared to other projection
methods.

\section{Types of Projectors}\label{sec:projector_types}

One of the first mentions of projector routines can be found in the RECLBL library package
\insertref{RECLBL}. Namely, the two types of projectors introduced there are the
\textit{voxel-driven} (or \textit{pixel-driven} for the 2D case) and the \textit{ray driven}. Many
state of the art projectors still build on the ideas of these projectors.

The voxel driven approach is very simple. For the forward projection, each voxel is visited and the
voxel center is projected onto the detector. Finally, the contribution of the voxel to the detector,
is computed using some form of interpolation.~\cite{peters_algorithms_1981} used bilinear
interpolation between the two neighbouring detector pixels.~\cite{harauz_interpolation_1983}
improved on the approach by using bicubic spline interpolation. However, the approach is used rarely
due to the introduced artifacts (C.f.~\cite[Chapter 3.3]{levakhina_three-dimensional_2014}). I.e. if
the resolution of the detector is finer compared to the volume, detector pixels might never be
assigned a value. \todo{describe back projector properly as well}

Instead of projecting the voxel center and interpolating, one can project the complete voxel onto
the projector plane. This approach was taken by~\cite{long_3d_2010, long_3d_nodate} and is usually
referred to as separable footprint. They use trapezoidal functions to approximate the footprint both
accurately and efficiently. Hence, the contribution of voxels to the detector pixels is based on
these trapeziodial functions. It as also been ported to graphic processing units (GPUs) as shown in
~\cite{wu_gpu_2011, xie_effective_2015, chapdelaine_new_nodate}. To the best of my knowledge, for
voxel based approaches, this is the state of the art approach and outperforms other approaches.

This approach was translated to B-Splines in~\cite{momey_b-spline_2012, momey_spline_2015}. There
B-Splines are assumed to be the basis function at pixel centers, and then the B-Splines are
projected onto the projector plane. As shown in chapter about B-Spline basis functions
\insertref{B-Spline chapter}, the projection of \(n\)-dimensional B-Splines yield a
\(n-1\)-dimensional B-Spline, hence the projection is simple and
accurate.~\cite{momey_b-spline_2012} already incoorporated both parallel-beam and cone-beam
geometry.

Similarly,~\cite{ziegler_efficient_2006} proposed a footprint approach for blobs. And it was
improved and ported to the GPU by~\cite{bippus_projector_2011}.~\cite{kohler_iterative_2011}
describes a blob projector for phase-contrast CT\@.

A shared problem of the voxel-based approaches, is the challenge of parallel implementations. During
the forward projectiong, shared access to the projector pixels might be needed. Hence, mitigation
strategies must be developed.

A conceptually different approach compared to the voxel-based approach, is the \textit{ray-based}
approach. There, the key idea is to trace rays through the volume. Typically, the intersection
length is used as a contribution of a pixel to the detector pixel.

Note that this the forward projection of this approach, is trivially parallisable. Each ray can be
traced independently through the volume.

A classical ray-driven approach was proposed by~\cite{siddon_fast_1985}, often is is referred to as
Siddon's method. There, the exact path length of a ray traversing through a volume is calculated.
I.e.\, this is the exact calculation of the line integral of a single ray through the volume. More
recent work like~\cite{jacobs_fast_1998, christiaens_fast_1999, zhao_fast_2004, gao_fast_2012}
improved on the Siddon's method, especially in efficiency and
performance.~\cite{de_greef_accelerated_2009, xiao_efficient_2012} ported Siddon's methods to the
GPU\@.

Another classic is presented in~\cite{joseph_improved_1982}. It assumes a smooth image and
interpolates between neighbouring voxels along the ray path. It does so by a in slice-interpolation,
i.e.\ the voxels perpendicular to the main ray direction are visited.~\cite{graetz_high_2020}
proposed a branchless GPU version of this approach.

The backward projections of ray-driven approaches typically trace the ray from a detector pixel to
the source and update all visited voxels based on the weights calculated as in the forward
direction. And in the of Joseph's based methods, also voxels close by are updated. Note again here,
that parallelization is hindered by the possibility of shared write access to voxels.

Typically, iterative reconstruction algorithms expect that the forward and backward projectors are
adjoints of each other \insertref{forward and backward projectors are adjoints}, i.e.\ they are
\textit{matched} projector pair. This fails for basic pixel- and ray-driven approaches.
They either either introduce artifacts during the forward or backward projections.

An entirely different approach is the \textit{distance-driven} approach
~\cite{de_man_distance-driven_2002, de_man_distance-driven_2004}. There, the voxel boundaries and
detector pixel boundaries are projected onto a common axis, then the overlap is used as a weight.
This approach is still state of the art, however, is suffers from inaccuracys for projections, which
are close to 45°. A branchless GPU version was proposed in~\cite{liu_gpu-based_2017}. The
distance-driven approach doesn't suffer from any high-frequency artifacts.

Projectors based on blobs have been studied for quite some time. Ray-driven approaches are based on
algorithms proposed in~\cite{matej_practical_1996, popescu_ray_2004}. But rather than assuming a
infinitely thin ray, they assume a beam. Hence, they are somewhat similar to Joseph's projectors,
that they have to visit neighbouring voxels, as the support of blobs is larger than that of pixels.
~\cite{levakhina_distance-driven_2010} proposed a variant of the distance-driven approach based on
blobs.

Another area of projectors compute the intersection area of multiple rays. Such methods, as proposed
in~\cite{ha_study_2015, ha_efficient_2016, ha_look-up_2018}, calculate the intersection between rays
directed a detector boundaries.

\section{Implementation}


\chapter{Experiments}

How does it compare
