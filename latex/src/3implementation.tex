\chapter{elsa}

So much implementation design

\chapter{Projector}

There are a couple of basic principles on how projectors for X-ray attentuation CT work.
The most common ones are \textit{ray driven}, \textit{voxel driven}, \textit{distance driven},
or using \textit{separable footsprints}.


\section{Ray driven}
 
Siddon \cite{siddon_fast_1985}, Joseph \cite{joseph_improved_1982}, with improvments
by \cite{jacobs_fast_1998}, \cite{christiaens_fast_1999}, \cite{zhao_fast_2004} and \cite{gao_fast_2012}.

\section{Voxel driven}

\enquote{The pixel-driven forward projection is rarely used because it tends to introduce
high-frequency artifacts to the sinogram. If the detector element size is much smaller than the
pixel element size, there is a danger to have detector elements in which no value is written. In a
similar manner, in the pixel-driven backprojection, each pixel element is updated based on the value
which is obtained from the neighboring detector elements, typically using linear interpolation.}
\cite{levakhina_three-dimensional_2014}

Introduced by \cite{peters_algorithms_1981}, with improved interpolation by \cite{harauz_interpolation_1983}

\section{Distance driven}
 
Introduced by \cite{de_man_distance-driven_2002}, and extended to 3D by \cite{de_man_distance-driven_2004}

\section{Separable footprints}

Improved over distance driven projector, introducted by \cite{long_3d_2010} \cite{long_3d_nodate}.

\chapter{Experiments}

How does it compare

