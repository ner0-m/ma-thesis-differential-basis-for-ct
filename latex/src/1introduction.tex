\chapter{Introduction}\label{chap:introduction}

This chapter briefly introduces the context in which this thesis is places.

\section{Motivation}\label{chap:Motivation}

In \citeyear{rontgen_uber_1895} \citeauthor{rontgen_uber_1895} discovered X-rays. Since then
different methods based on X-ray have been revolutionizing medical treatment. For the first time it
was possible to inspect (among others) the human body, without cutting it open. It was used among
others to detect metal objects in the body such as bullets~\cite{haygood_skeletal_1996}. But also
new kinds of fractures could be detected and diagnosed. Bone diseases such as Osteopoikilosis could
be discovered, or it helped identify skeletal tuberculosis~\cite{haygood_skeletal_1996}, just to
name a few.

The next major step was the invention of X-ray attenuation computed tomography (CT), which is often
accounted to \citeauthor{hounsfield_computerized_1973}. At EMI laboratories the first prototypes
were build and tested. Finally, it was presented in a series of
publications~\cite{hounsfield_computerized_1973,ambrose_computerized_1973, perry_computerized_1973}.
Compared to simple X-ray images, X-ray attenuation CT can create cross-sectional images from the
interior of the scanned object. Whereas, in tradition X-ray imaging different objects superimpose
each other, X-ray attenuation CT can distinguish them.

However, X-ray attenuation does lack soft-tissue contrast. I.e.\ it is very easy to distinguish
bones from muscles, as the attenuation of these tissues is different enough. But its challenging to
distinguish different kinds of soft-tissues, e.g.\ tumor tissues of healthy tissue.

Phase-contrast X-ray imaging improves this specific issues~\cite{lewis_medical_2004} plus it can
reveal additional information~\cite{hahn_numerical_2012}. One way to measure phase-contrast for
X-rays are grating interferometer~\cite{pfeiffer_hard-x-ray_2008}. This information can be measured
using the refraction of X-rays, instead of the attenuation. Specifically, the phase gradient
perpendicular to the gratings can be measured, therefore, this method is often referred to as
\textit{differential phase-contrast imaging}.

Further, the same grating interferometer yields further information about small-angle scattering.
Imaging modalities based on this property are referred to as \textit{X-ray dark-field imaging}.
Compared to attenuation based imaging. Studies have been proposed for many different medical
applications, see~\cite[chap. 1.3.1]{wieczorek_anisotropic_2017} for an overview of the different
fields. Only recently, the grating interferometer setup was incorporated into a clinical CT
scanner~\cite{viermetz_dark-field_2022}.

This is the working context of the research group \textit{\gls{CIIP}} at the \gls{TUM} around PD
Dr.\ rer.\ nat.\ Tobias Lasser. Specifically, at the research group the open-source C++ framework
\textit{elsa} is developed for use in tomographic reconstruction. Currently, it only supports X-ray
attenuation CT\@. As the previous paragraphs should make clear, there exists high interest to
support imaging modalities based on X-ray differential phase-contrast and X-ray dark-field contrast.
This thesis is a step into this direction, specifically in the direction to support differential
phase-contrast in elsa.

\section{Scope}\label{chap:scope}

\inlinetodo{Maybe this isn't necessary anymore}

The scope of this work is twofold. First, a major part is the support of some kind of differentiable
basis functions in the elsa~\cite{lasser_elsa_2019} framework. Only recently, the team
around Prof.\ Dr.\ Franz Pfeiffer from the Technical University of Munich developed an advanced
Dark-Field imaging to the human scale~\cite{viermetz_dark-field_2022}. This is an exiting
development, which also has an impact on team around my supervisor PD Dr.\ rer.\ nat.\ Tobias
Lasser. As it stands today, elsa is not able to reconstruct either phase-contrast CT nor Dark-field
CT\@. This thesis does not implement full support for reconstruction of phase-contrast CT\@.
However, this thesis should lay the foundation for this support. In the scope of this thesis is a
projector, which handles different kinds of differentiable basis functions. As these kind basis
functions are the basis to support, phase-contrast CT\@.

Secondly, this thesis should be a guide for students trying to get into the field of tomographic
reconstruction. E.g.\ Students starting a project at the working group of PD Dr.\ rer.\ nat.\ Tobias
Lasser. This thesis tries to cover many areas, which are already or soon to be implemented in elsa.
And so, this should be of use to people trying to get into elsa, or parts of this will be worked
into the documentation of elsa.

I want the theoretical part to be of help for other students, but also for my future self as a
reference. Hence, I will try to investigate problems on different levels. One level is the
motivational and high level overview. This covers the larger picture, on what certain methods
provide benefits over other methods and how they relate. Plus certain similarities and differences.
Another level is an intuitive approach. Personally, throughout my studies, I found it immensely
useful to gain an intuition for a method. Together, with the motivational aspects and classification,
I could dig deeper into further details. Sadly, I will not be able to dig deep into all different
aspects. As best I can, I'll provide further reading material for those interested.

\section{Outline}\label{sec:outline}

To find a solution in tomographic reconstruction, there are a couple of important steps involved.
These include discretization of the problem, describing the model behind the reconstruction and
finally actually solving the problem.

The first part of the thesis, \autoref{chap:imaging_modalities}, covers different imaging
modalities. It covers their use cases, a bit of history, and a basic intuition on how they
work. Afterwards in \autoref{chap:radon_transform_and_related}, the mathematical abstraction for
these different physical models is laid down. Then \autoref{chap:image_representation} is all about
discretization of the problem domain. There different basis functions, that can be used to discrete
the image domain. With a specific focus on differentiable basis functions. In
\autoref{chap:tomographic_reconstruction} reconstruction techniques are presented. It is all about
recovering an image from projections. First, a couple of analytical solutions are presented, then a
deeper dive into iterative reconstruction algorithms is provided.

In the practical section, \autoref{chap:elsa} provides a light introduction into the C++ framework
\textit{elsa} developed at the \gls{CIIP} research group at the \gls{TUM}. The
next \autoref{chap:projector} provides details into implementations of the physical forward and
backward model. And a detailed discussion of
implementation choices for the projector developed as part of this thesis. In
\autoref{chap:experiments} a detailed evaluation of the projector is conducted.

The thesis ends with the conclusion in \autoref{chap:conclusion}.

\section{Disclaimer}\label{sec:disclaimer}

I want my research to be as open, reproducible and comprehensible as possible. Hence, the
development of the source code I've written is open source, discussions (though they were few) can be
found in the \href{https://gitlab.lrz.de/IP/elsa}{elsa GitLab repository}, the \LaTeX{} source code
of thesis can be found on my
\href{https://github.com/ner0-m/ma-thesis-differential-basis-for-ct}{personal GitHub page}. To the
best of my possibilities, I try to include the code I've used to generate all of my plots and
graphs, I try to state dependencies and make building and using my work as easy as I can. But of
course, I am aware, what is easy to me, might be cumbersome to the next, and impossible to the
other.

\inlinetodo{disclaimer gender-neutral language}
