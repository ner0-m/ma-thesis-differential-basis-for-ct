\chapter{Introduction}\label{chap:introduction}

this should not be empty!!


\section{Motivation}\label{chap:Motivation}

\section{Scope}\label{chap:scope}

The scope of this work is two fold. First, a major part is the support of some kind of
differentiable basis functions in the \textit{elsa}~\cite{lasser_elsa_2019} framework. Only
recently, the team around Prof.\ Dr.\ Franz Pfizer from the Technical University of Munich developed
an advanced Dark-Field imaging to the human scale~\cite{viermetz_dark-field_2022}. This is an
exiting development, which also has an impact on team around my supervisor PD Dr.\ rer.\ nat.\
Tobias Lasser. As it stands today, elsa is not able to reconstruct either phase-contrast CT nor Dark-field
CT\@. This thesis does not implement full support for reconstruction of phase-contrast CT\@.
However, this thesis should lay the foundation for this support. In the scope of this thesis is a
projector, which handles different kinds of differentiable basis functions. As these kind basis
functions are the basis to support, phase-contrast CT\@.

Secondly, this thesis should be a guide for students trying to get into the field of tomographic
reconstruction. E.g.\ Students starting a project at the working group of PD Dr.\ rer.\ nat.\ Tobias
Lasser. This thesis tries to cover many areas, which are already or soon to be implemented in elsa.
And so, this should be of use to people trying to get into elsa, or parts of this will be worked
into the documentation of elsa.

I want the theoretical part to be of help for other students, but also for my future self as a
reference. Hence, I will try to investigate problems on different levels. One level is the
motivational and high level overview. This covers the larger picture, on what certain methods
provide benefits over other methods and how they relate. Plus certain similarities and differences.
Another level is an intuitive approach. Personally, throughout my studies, I found it immensely
useful to gain a intuition for a method. Together, with the motivational aspects and classification,
I could dig deeper into further details. Sadly, I will not be able to dig deep into all different
aspects. As best I can, I'll provide further reading material for those interested.

\section{Outline}\label{sec:outline}

For problems in tomographic reconstruction, there are a couple of important steps involved. These
include discretization of the problem, describing the model behind the reonconstruction and finally
actually solving the problem.

The first part of the thesis, \autoref{chap:imaging_modalities}, covers different imaging
modalities. It covers their use cases, a little bit of history, and a basic intuition on how they
work. Afterwards in \autoref{chap:radon_transform_and_related}, the mathematical abstraction for
these different physical models is laied down. Then \autoref{chap:image_representation} is all about
discretization of the problem domain. There different basis functions, that can be used to discrete
the image domain. With a specific focus on differentiable basis functions. In
\autoref{chap:tomographic_reconstruction} reconstruction techniques are presented. Its all about
recovering an image from projections. First, a couple of analytical solutions are presented, then a
deeper dive into iterative reconstruction algorithms is provided.

In the practical section, \autoref{chap:elsa} provides a light introduction into the C++ framework
\textit{elsa} developed at the \textit{Computational Imaging and Inverse Problems} research group at
the Technichal University of Munich. The next \autoref{chap:projector} provides details into
implementations of the physical forward and backward model. And a detailed discussion of
implementation choices for the projector developed as part of this thesis. In
\autoref{chap:experiments} a detailed evaluation of the projector is conducted.

The thesis ends with the conclusion in \autoref{chap:conclusion}.

\section{Disclaimer}\label{sec:disclaimer}

I want my research to be as open, reproducible and comprehensible as possible. Hence, the
development of the source code I've wrote is open source, discussions (though they were few) can be
found in the \href{https://gitlab.lrz.de/IP/elsa}{elsa GitLab repository}, the \LaTeX{} source code
of thesis can be found on my
\href{https://github.com/ner0-m/ma-thesis-differential-basis-for-ct}{personal GitHub page}. To the
best of my possibilities, I try to include the code I've used to generate all of my plots and
graphs, I try to state dependencies and make building and using my work as easy as I can. But of
course, I am aware, what is easy to me, might be cumberstone to the next, and impossible to the
other.

\inlinetodo{disclaimer gender-neutral language}
