\documentclass{mimosis}

\usepackage{metalogo}

%%%%%%%%%%%%%%%%%%%%%%%%%%%%%%%%%%%%%%%%%%%%%%%%%%%%%%%%%%%%%%%%%%%%%%%%
% Some of my favourite personal adjustments
%%%%%%%%%%%%%%%%%%%%%%%%%%%%%%%%%%%%%%%%%%%%%%%%%%%%%%%%%%%%%%%%%%%%%%%%
%
% These are the adjustments that I consider necessary for typesetting
% a nice thesis. However, they are *not* included in the template, as
% I do not want to force you to use them.

% This ensures that I am able to typeset bold font in table while still aligning the numbers
% correctly.
\usepackage{etoolbox}
\usepackage{csvsimple}

% use for stackinset
\usepackage[usestackEOL]{stackengine}
\usepackage{caption}
\usepackage{subcaption}

% Use svg
\usepackage{svg}

% Render incscape pdf's with equation
\usepackage{import}
\usepackage{xifthen}
\usepackage{pdfpages}
\usepackage{transparent}

% https://tex.stackexchange.com/a/282
\usepackage[section]{placeins}

% Scientific notation
\usepackage{siunitx}

%%%%%%%%%%%%%%%%%%%%%%%%%%%%%%%%%%%%%%%%%%%%%%%%%%%%%%%%%%%%%%%%%%%%%%%%
% Colors
%%%%%%%%%%%%%%%%%%%%%%%%%%%%%%%%%%%%%%%%%%%%%%%%%%%%%%%%%%%%%%%%%%%%%%%%
\definecolor{black}               {RGB}{  0,   0,  0}
\definecolor{kelly_green}         {RGB}{ 76, 187, 23}
\definecolor{international_orange}{RGB}{255,  79,  0}
\definecolor{venetian_red}        {RGB}{200,   8, 21}

\definecolor{amber}     {RGB}{255,191,  0}
\definecolor{burgundy}  {RGB}{144,  0, 32}
\definecolor{cardinal}     {RGB}{247, 118, 142}
\definecolor{deep-azure}{RGB}{ 10, 50, 94}
\definecolor{mahogany}  {RGB}{192, 64,  0}
\definecolor{malory-red}{RGB}{192, 46, 16}
\definecolor{sky}       {RGB}{ 62, 93,240}
\definecolor{teal}      {RGB}{  0,128,128}
\definecolor{tyrian}    {RGB}{102,  2, 60}
\definecolor{yale}      {RGB}{ 70,130,180}

\definecolor{quality-high}  {RGB}{ 70,130,180}
\definecolor{quality-medium}{RGB}{255,191,  0}
\definecolor{quality-low}   {RGB}{196, 30, 58}
%
% Colour for Part numbers
%
\renewcommand*{\partformat}{%
	\textcolor{cardinal}{Part \thepart}\autodot\enskip%
}

%
% Colour for section numbers
%
\renewcommand*{\sectionformat}{%
	\textcolor{cardinal}{\thesection}\autodot\enskip%
}

%
% Colour for subsection numbers
%
\renewcommand*{\subsectionformat}{%
	\textcolor{cardinal}{\thesubsection}\autodot\enskip%
}

%
% Colour for subsubsection numbers
%
\renewcommand*{\subsubsectionformat}{%
	\textcolor{cardinal}{\thesubsubsection}\autodot\enskip%
}

%ornamet
\newcommand{\ornament}{%
	\begin{center}
		\large\textcolor{cardinal}{\ding{70}}
	\end{center}
}

% \pagecolor[RGB]{26,27,38}
% \color[RGB]{192, 202, 245}

% \pagecolor[RGB]{213, 214, 219}
% \color[RGB]{52, 59, 88}

%%%%%%%%%%%%%%%%%%%%%%%%%%%%%%%%%%%%%%%%%%%%%%%%%%%%%%%%%%%%%%%%%%%%%%%%
% Figure commands
%%%%%%%%%%%%%%%%%%%%%%%%%%%%%%%%%%%%%%%%%%%%%%%%%%%%%%%%%%%%%%%%%%%%%%%%

% Then I can import matplotlib stuff
\usepackage{pgf}

% Usage of includegraphicsmaybe
% \begin{figure}[t]
%         \centering
%         \includegraphicsmaybe[width=0.24\textwidth]{asdasdfigures/experiments/forward_projection/2dphantom.png}
%         \caption{Test maybe image\inlinetodo{Put figure here}}\label{fig:test_maybe}
% \end{figure}
\newcommand{\includegraphicsmaybe}[2][]{\IfFileExists{#2}{\includegraphics[#1]{#2}}{\includegraphics{example-image}}}

% \newcommand{\incfig}[2]{%
% 	\def\svgwidth{\columnwidth}
% 	\import{#1}{#2.pdf_tex}
% }
\newcommand{\incfig}[1]{%
	\def\svgwidth{\columnwidth}
	\input{#1.pdf_tex}
}
\newcommand{\incfigmaybe}[1]{\IfFileExists{#1.pdf_tex}{\def\svgwidth{\columnwidth}\input{#1.pdf_tex}}{\includegraphics{example-image}}}

% This avoid nasty errors, but I also need symbolic links from subfolder to figures folder
\graphicspath{{figures/}}
% \newcommand{\incfigmaybe}[1]{\IfFileExists{#1}{\incfig{#1}}{\includegraphics{example-image-a}}}
%%%%%%%%%%%%%%%%%%%%%%%%%%%%%%%%%%%%%%%%%%%%%%%%%%%%%%%%%%%%%%%%%%%%%%%%
% Sectioning of thesis
%%%%%%%%%%%%%%%%%%%%%%%%%%%%%%%%%%%%%%%%%%%%%%%%%%%%%%%%%%%%%%%%%%%%%%%%

% I want subsubsections to have numbering as well
\setcounter{secnumdepth}{\subsubsectionnumdepth}

%%%%%%%%%%%%%%%%%%%%%%%%%%%%%%%%%%%%%%%%%%%%%%%%%%%%%%%%%%%%%%%%%%%%%%%%
% Code in latex
%%%%%%%%%%%%%%%%%%%%%%%%%%%%%%%%%%%%%%%%%%%%%%%%%%%%%%%%%%%%%%%%%%%%%%%%

\usepackage[chapter]{minted}
\usemintedstyle{friendly}

%%%%%%%%%%%%%%%%%%%%%%%%%%%%%%%%%%%%%%%%%%%%%%%%%%%%%%%%%%%%%%%%%%%%%%%%
% Hyperlinks & bookmarks
%%%%%%%%%%%%%%%%%%%%%%%%%%%%%%%%%%%%%%%%%%%%%%%%%%%%%%%%%%%%%%%%%%%%%%%%

\usepackage[%
	colorlinks = true,
	citecolor  = cardinal,
	linkcolor  = cardinal,
	urlcolor   = cardinal,
	unicode = true,
]{hyperref}

\usepackage{bookmark}

%%%%%%%%%%%%%%%%%%%%%%%%%%%%%%%%%%%%%%%%%%%%%%%%%%%%%%%%%%%%%%%%%%%%%%%%
% TODO notes
%%%%%%%%%%%%%%%%%%%%%%%%%%%%%%%%%%%%%%%%%%%%%%%%%%%%%%%%%%%%%%%%%%%%%%%%

\usepackage[colorinlistoftodos,prependcaption,textsize=tiny]{todonotes}
\setuptodonotes{noshadow, color=amber}

% If a ref is missing
\newcommand{\insertref}[1]{\todo[color=green!40]{MISSING REF: #1}}
\newcommand{\inlinetodo}[1]{\todo[inline]{#1}}
\newcommand{\missingfig}[1]{\missingfigure[figcolor=white]{#1}}

%%%%%%%%%%%%%%%%%%%%%%%%%%%%%%%%%%%%%%%%%%%%%%%%%%%%%%%%%%%%%%%%%%%%%%%%
% Math relatex settings
%%%%%%%%%%%%%%%%%%%%%%%%%%%%%%%%%%%%%%%%%%%%%%%%%%%%%%%%%%%%%%%%%%%%%%%%


\usepackage{thmtools}

\declaretheoremstyle[
	headpunct={},
	headfont=\color{cardinal}\normalfont\bfseries,
	notefont=\itshape,
	bodyfont=\normalfont,
]{definitioncolor}

\declaretheorem[
	style=definitioncolor,
	name=Definition,
	numberwithin=chapter
]{definition}

% \usepackage{bm}
\usepackage{mathtools}

% Define \abs
\DeclarePairedDelimiter\abs{\lvert}{\rvert}%

% Define \norm
\DeclarePairedDelimiter\norm{\lVert}{\rVert}%

% Define \sign
\DeclareMathOperator{\sign}{sign}

% Define Radon Transform
\newcommand{\radon}{\mathscr{R}}
% Define X-ray Transform
\newcommand{\xray}{\mathscr{X}}
% definitions for vectors
\newcommand{\mvec}[1]{\symbfit{#1}}
% approx something
\newcommand{\near}[1]{\hat{#1}}
%%%%%%%%%%%%%%%%%%%%%%%%%%%%%%%%%%%%%%%%%%%%%%%%%%%%%%%%%%%%%%%%%%%%%%%%
% Bibliography
%%%%%%%%%%%%%%%%%%%%%%%%%%%%%%%%%%%%%%%%%%%%%%%%%%%%%%%%%%%%%%%%%%%%%%%%
%
% I like the bibliography to be extremely plain, showing only a numeric
% identifier and citing everything in simple brackets. The first names,
% if present, will be initialized. DOIs and URLs will be preserved.

\usepackage[%
	autocite     = plain,
	backend      = biber,
	doi          = true,
	url          = true,
	giveninits   = true,
	hyperref     = true,
	maxbibnames  = 99,
	maxcitenames = 99,
	sortcites    = true,
	style        = numeric,
]{biblatex}

\input{bibliography-mimosis}
\addbibresource{bibliography.bib}

%%%%%%%%%%%%%%%%%%%%%%%%%%%%%%%%%%%%%%%%%%%%%%%%%%%%%%%%%%%%%%%%%%%%%%%%
% Fonts
%%%%%%%%%%%%%%%%%%%%%%%%%%%%%%%%%%%%%%%%%%%%%%%%%%%%%%%%%%%%%%%%%%%%%%%%

\ifxetexorluatex%
	\usepackage{amssymb}
	% \usepackage{mathrsfs}
	\let\mathbbalt\mathbb
	\usepackage{unicode-math}
	\let\mathbb\mathbbalt% UNIVERSAL RESET TO ORIGINAL \mathbb

	\setmainfont{IBM Plex Serif}[Scale=MatchLowercase]
	\setsansfont{IBM Plex Sans}[Scale=MatchLowercase]
	\setmonofont{IBM Plex Mono}[Scale=MatchLowercase]

	% \setmathfont[range={\mathscr,\mathbfscr}]{XITS Math}
	% Try again after restart
	% \setmathfont[range={\mathscr,\mathbfscr}]{XITS Math}
\else
	\usepackage[lf]{ebgaramond}
	\usepackage[oldstyle,scale=0.7]{sourcecodepro}
	\singlespacing%
\fi

\newacronym[description={Principal component analysis}]{PCA}{PCA}{principal component analysis}
\newacronym                                            {SNF}{SNF}{Smith normal form}
\newacronym[description={Topological data analysis}]   {TDA}{TDA}{topological data analysis}
\newacronym[description={Computational Imaging and Inverse Problems research group}]{CIIP}{CIIP}{Computational Imaging and Inverse Problems}
\newacronym{TUM}{TUM}{Technical University of Munich}
\newacronym{CT}{CT}{computed tomography}
\newacronym{ART}{ART}{Algebraic Reconstruction Technique}
\newacronym{SIRT}{SIRT}{Simultaneous Iterative Reconstruction Technique}
\newacronym{ISTA}{ISTA}{Iterative Shrinkage-Thresholding Algorithms}
\newacronym{FISTA}{FISTA}{Fast Iterative Shrinkage-Thresholding Algorithm}
\newacronym{FBP}{FBP}{Filtered Back-projection}
\newacronym{CPU}{CPU}{Central Processing Unit}
\newacronym{GPU}{GPU}{Graphics Processing Unit}
\newacronym{LUT}{LUT}{Look-up Table}
\newacronym{MSE}{MSE}{Mean Squared Error}
\newacronym{RMSE}{RMSE}{Root Mean Square Error}
\newacronym{NRMSE}{NRMSE}{Normalized Root Mean Square Error}
\newacronym{PSNR}{PSNR}{Peak Signal to Noise Ration}
\newacronym{SSIM}{SSIM}{Structural Similarity Index Measure}
\newacronym{BSpline}{B-Spline}{Basic-Spline}
\newacronym{AXDT}{AXDT}{Anisotropic X-ray Dark-field Tomography}

\makeindex%
\makeglossaries%

%%%%%%%%%%%%%%%%%%%%%%%%%%%%%%%%%%%%%%%%%%%%%%%%%%%%%%%%%%%%%%%%%%%%%%%%
% Nomenclature
%%%%%%%%%%%%%%%%%%%%%%%%%%%%%%%%%%%%%%%%%%%%%%%%%%%%%%%%%%%%%%%%%%%%%%%%
% \usepackage{nomencl}
% \renewcommand{\nomname}{List of Symbols}

% \makenomenclature%


%%%%%%%%%%%%%%%%%%%%%%%%%%%%%%%%%%%%%%%%%%%%%%%%%%%%%%%%%%%%%%%%%%%%%%%%
% Ordinals
%%%%%%%%%%%%%%%%%%%%%%%%%%%%%%%%%%%%%%%%%%%%%%%%%%%%%%%%%%%%%%%%%%%%%%%%

\makeatletter
\@ifundefined{st}{%
	\newcommand{\st}{\textsuperscript{\textup{st}}\xspace}
}{}
\@ifundefined{rd}{%
	\newcommand{\rd}{\textsuperscript{\textup{rd}}\xspace}
}{}
\@ifundefined{nd}{%
	\newcommand{\nd}{\textsuperscript{\textup{nd}}\xspace}
}{}
\makeatother

\renewcommand{\th}{\textsuperscript{\textup{th}}\xspace}

%%%%%%%%%%%%%%%%%%%%%%%%%%%%%%%%%%%%%%%%%%%%%%%%%%%%%%%%%%%%%%%%%%%%%%%%
% Incipit
%%%%%%%%%%%%%%%%%%%%%%%%%%%%%%%%%%%%%%%%%%%%%%%%%%%%%%%%%%%%%%%%%%%%%%%%

%%%%%%%%%%%%%%%%%%%%%%%%%%%%%%%%%%%%%%%%%%%%%%%%%%%%%%%%%%%%%%%%%%%%%%%%
% General Information about my thesis
%%%%%%%%%%%%%%%%%%%%%%%%%%%%%%%%%%%%%%%%%%%%%%%%%%%%%%%%%%%%%%%%%%%%%%%%
\newcommand*{\getUniversity}{Technische Universität München}
\newcommand*{\getTitleEng}{Differentiable projection operations for X-ray computed tomography}
\newcommand*{\getTitleGer}{Differenzierbare Projektionsoperatoren für Röntgen-Computertomographie}
\newcommand*{\getFaculty}{Department of Informatics}
\newcommand*{\getDoctype}{Master's Thesis in Data Engineering and Analytics}
\newcommand*{\getSupervisor}{PD Dr.\ rer.\ nat.\ Tobias Lasser}
\newcommand*{\getAdvisor}{Advisor}
\newcommand*{\getAuthor}{David Frank}
\newcommand*{\getSubmissionDate}{15.06.2021}
\newcommand*{\getSubmissionLocation}{Munich}


\title{\texttt{\getTitleEng}}
\subtitle{\texttt{\getTitleGer}}

\author{\getAuthor{}}

\begin{document}

\frontmatter
\begin{titlepage}
    \makeatletter
    \begin{center}
        % Let's put the TUM logo here
        \includegraphics[height=20mm]{figures/logos/tum_logo.png}\\[0.5cm]
        % Department
        \begin{Huge}
            \MakeUppercase{\getFaculty}
        \end{Huge}\\[0.5cm]
        % University
        \begin{large}
            \MakeUppercase{\getUniversity}
        \end{large}\\[2cm]
        % Type of thesis
        \begin{Large}
            \getDoctype%
        \end{Large}\\[2cm]
        %
        \begin{Huge}
            \@title\par
        \end{Huge}
        \vspace{5mm}
        %
        \begin{Large}
            \@subtitle
        \end{Large}\\
        %
        \emph{by}\\
        \getAuthor
        % Let's inmclude the Informatics logo as well
        \vfill
        \includegraphics[height=20mm]{figures/logos/info_logo.png}\\
    \end{center}
    % Directly append the title page, to avoid the empty page...
    \newpage
    \thispagestyle{empty}
    \begin{center}
        % Let's put the TUM logo here
        \includegraphics[height=20mm]{figures/logos/tum_logo.png}\\[0.5cm]
        % Department
        \begin{Huge}
            \MakeUppercase{\getFaculty}
        \end{Huge}\\[0.5cm]
        % University
        \begin{large}
            \MakeUppercase{\getUniversity}
        \end{large}\\[2cm]
        % Type of thesis
        \begin{Large}
            \getDoctype%
        \end{Large}\\[1.5cm]
        %
        \begin{Huge}
            \@title\par
        \end{Huge}
        \vspace{10mm}
        %
        \begin{Huge}
            \texttt{\getTitleGer{}}
        \end{Huge}
        \vspace{10mm}
        %
        \begin{tabular}{l l}
            Author:          & \getAuthor{} \\
            Supervisor:      & \getSupervisor{} \\
            Submission Date: & \getSubmissionDate{} \\
        \end{tabular}
        % Let's inmclude the Informatics logo as well
        \vfill
        \includegraphics[height=20mm]{figures/logos/info_logo.png}\\
    \end{center}
    \makeatother
\end{titlepage}

\input{latex/pages/disclaimer}
\thispagestyle{empty}
\chapter*{Acknowledgements}
% \begin{center}
%   \textsc{Acknowledgements}
% \end{center}

\noindent
%

Firstly, of course, I have to thank my supervisor Tobias Lasser! We have been working together now
for more than 5 years. It all started with a bachelor Seminar, turned into my bachelor's thesis,
followed by many years of research assistant positions.

During the last years of my studies, I found myself struggling with the pressure and stress.
Tobias always supported me in many different ways, writing letters of recommendation, providing me
an amazing opportunity to do an intership in Denmark, providing me access to facilities or just
show interest in weird my intererst.

I sincerely thank Tobias, for the freedome he offered me. During all the different projects we worked on,
I never had the feeling of doing something, I was forced to and did not like myself.
Raterh, it was the other way around. I could come up with esoteric ideas and given enough reason,
I could spend time on it and explore. This is and was a great way to grow for me personally. And
I can't thank him enough for that.

Then I also want to thank my family. I want to thank my parents. They supported me in many different
ways thorughout my studies. Of course, they provided me with the chances to my education and
supported me financially.

Lastly, I want to thank - in no particular order - Magi, Luzi, Mira, Eli, Anna, Phillip, Alex, Alaya,
Magda for everything. For all the discussions, late night talks, walks, breakfest talks, coffee breaks.
For all the emotional support, for putting up with me in all my different moods, for providing
distraction from my moods.

Without all of you, I would have not made it.

\begin{flushleft}
	Thank you so so much.
\end{flushleft}

%%%%%%%%%%%%%%%%%%%%%%%%%%%%%%%%%%%%%%%%%%%%%%%%%%%%%%%%%%%%%%%%%%%%%%%%
\chapter*{Abstract}
%%%%%%%%%%%%%%%%%%%%%%%%%%%%%%%%%%%%%%%%%%%%%%%%%%%%%%%%%%%%%%%%%%%%%%%%

\noindent%

Something Something such a nice abstract


\tableofcontents

\mainmatter
%%%%%%%%%%%%%%%%%%%%%%%%%%%%%%%%%%%%%%%%%%%%%%%%%%%%%%%%%%%%%%%%%%%%%%%%
% List of TODOs
%%%%%%%%%%%%%%%%%%%%%%%%%%%%%%%%%%%%%%%%%%%%%%%%%%%%%%%%%%%%%%%%%%%%%%%%
% \todototoc
% \listoftodos

%%%%%%%%%%%%%%%%%%%%%%%%%%%%%%%%%%%%%%%%%%%%%%%%%%%%%%%%%%%%%%%%%%%%%%%%
% Introduction
%%%%%%%%%%%%%%%%%%%%%%%%%%%%%%%%%%%%%%%%%%%%%%%%%%%%%%%%%%%%%%%%%%%%%%%%
\part[Introduction]{%
	Introduction\\
	%
	\vspace{1cm}
	%
	% \begin{minipage}[l]{\textwidth}
	% 	%
	% 	\textnormal{%
	% 		\normalsize
	% 		%
	% 		\begin{singlespace*}
	% 			\onehalfspacing
	% 			%
	% 			You can also use parts in order to partition your great work
	% 			into larger `chunks'. This involves some manual adjustments in
	% 			terms of the layout, though.
	% 		\end{singlespace*}
	% 	}
	% \end{minipage}
}\label{part:introduction}

\chapter{Introduction}

this shoult not be empty!!

\chapter{Scope}

This is the scope of my thesis let's keep writing because this is not very
important, but let's write it a little longer so that I see how it works.


%%%%%%%%%%%%%%%%%%%%%%%%%%%%%%%%%%%%%%%%%%%%%%%%%%%%%%%%%%%%%%%%%%%%%%%%
% Background
%%%%%%%%%%%%%%%%%%%%%%%%%%%%%%%%%%%%%%%%%%%%%%%%%%%%%%%%%%%%%%%%%%%%%%%%
\part[Foundation]{%
	Foundation\\
	%
	\vspace{1cm}
	%
	% \begin{minipage}[l]{\textwidth}
	% 	%
	% 	\textnormal{%
	% 		\normalsize
	% 		%
	% 		\begin{singlespace*}
	% 			\onehalfspacing%
	% 		\end{singlespace*}
	% 	}
	% \end{minipage}
}\label{part:foundation}

\chapter{Notation}

This is my notation used thourgh the thesis

\chapter{Inverse Problems}

\section{Applications}
 
\subsection{X-ray Attenuation CT}

How does X-ray Attenuation CT work

\subsection{Phase Contrast CT}

How does phase contrast ct work

\section{Solving Inverse Problems}
 
\subsection{ART}

ART and derivatives

\subsection{CG}

CG and such
 
\subsection{First-order methods}

First orther methods such as Gradient Descent and it's derivatives

\section{Regularization}

\subsection{Tikonov}

How does tikonov reg wwork

\subsection{TV Regularization}

This should also work

\chapter{Basis}

Explain basis

\section{Series expansion}

Before we looks closely at different basis functions, we need a couple of definitions to concentrate
on the different basis functions. So far, the inverse problems discussed are treated mostly in a
continuous fashion, i.e. at no point we look at problems arising from discretization.

\begin{definition}[Signal]
    \label{def:signal}
    Let $f: \mathbb{R}^n \to \mathbb{R}$ be a $n$-dimensional continuous function, whose support is 
    bounded. We'll refer to it as a $n$-dimensional signal. And often it will be refered to as signal, 
    without the special mention of $n$-dimensional.

    In the special case of $n=2$, it is called an \textit{image} (following \cite{herman_basis_2015})
    and in case of $n=3$, it is called a $volume$.
\end{definition}
\todo{Move to other section, it doesn't fit here quite right}

Remember the functions, we wish to reconstruct are of the from as defined in \ref{def:signal}.
However, these are continuous and therefore not representable by any digital computer.
We seek to find a permissible representation $\hat{f}$ of the signal $f$ \cite{herman_basis_2015}.

\begin{definition}[Permissible representation]
    \label{def:permissible_representation}
    Let $N \in \mathbb{N}$ be a positive integer and $\varphi_n$ a set basis function for 
    $1 \leq n \leq N$, then the signal $f$ can be approximated as a linear combinations
    of these basis functions and the coefficients $c_n$:
    \[ \hat{f}(x) = \sum_{k=1}^{N} c_k \varphi_k(x) \] 

    For our purposes, we assume the function lies on a regular spaced discrete grid. Then, let
    $\varphi$ be a zero centered symmetrical basis function, $\bm{k} \in \mathbb{Z}^n$ be the
    $n$-dimensional index a grid cell, and $x_{\bm{k}} \in \mathbb{R}^n$ the coordinate of the 
    $\bm{k}$-th grid cell. Then, the previous equation can be reformulated:
    \[ \hat{f}(x) = \sum_{\bm{k} \in \mathbb{Z}^n} c_{\bm{k}} \varphi(x - x_{\bm{k}}) \] 
    This definition follows the notation given in \cite{momey_new_2011}.
\end{definition}

From this, it is quite obvious how important the choise of basis functions are. 
A suboptimal representation, will yield undesired results. In \cite{nilchian_optimized_2015}, 4 
properties a basis functions should satisfy are proposed. These are:
\begin{itemize}
    \item Riesz Basis
    \item Partition Of Unity
    \item Compact Support
    \item Isotropy
\end{itemize}

Similar properties are stated in \cite{hanson_local_1985}.

\subsubsection{Riesz Basis}

\todo[inline]{Explain Riesz Basis, find definition of it and cite it}

\cite{hanson_local_1985} formulates a similar requierment, however states it less restrictive. It is
states as a requierment for strong linear independence, rather than a unique representation.
 
\paragraph{Partition of Unity}
 
Some function $g$, fulfills the property partition-of-unity if
\begin{itemize}
    \item $g: \mathbb{R}^n \to [0, 1]$, i.e. it $g$ maps into the unit interval
    \item $\sum_{\bm{k} \in \mathbb{Z}^n} g(x + k) = 1 \; \forall x \in \mathbb{R}^n$
\end{itemize}
Given the basis fulfills this property, the error of approximation converges to zero, with sampling 
step $\Delta$ going to zero \cite{nilchian_optimized_2015}. Formally
\begin{equation}
    \lim_{\Delta \to 0} \norm{f - \hat{f}}_{L_2} = 0
\end{equation}
\todo[inline]{define and explain the constraints required here}

This requirement is close to the property power of approximation and fidelity of visual
appearance in \cite{hanson_local_1985}.

\paragraph{Compact support}

The compact support is a rather practical requierment. In order to reduce the computational cost,
the function should be compact. Phrased differently, the smaller the support, fewer evaluations
for a given point are required, i.e. the number of zero elements in $c_{\bm{k}}$ grows with shrinking
support.

On the other hand, given a function without compact support, such as the gaussian.
\insertref{assumption about gaussian}

\paragraph{Isotropy}

Isotropy is again a practical requierment. If a basis function is isotropic, it is projections
do not depend on the direction or angle. This greatly simplifies the implementation and improves efficiency.

Both compact supoort and isotropy are practical requierments to fulfill the requierments of
efficient computation of forward and backward projection and implementation of reconstruction constraints
by \cite{hanson_local_1985}.

\subsection{Series expansion under the Radon transform}

\todo[inline]{Write up \cite{nilchian_optimized_2015} to get to matrix notation}

\section{Voxels}

\section{Blobs}

What ware spherically semmetrical basis elements (blobs)

Very important literature
\cite{lewitt_alternatives_1992}
\cite{matej_practical_1996}
 
\section{B-Splines}

B-Splines have been used in multiple fields. \insertref{Find some different fields}.

\begin{definition}[Recusrive B-Spline]
    \label{def:bspline}
    The centered B-Spline of degree $d$ and of width $\Delta$ is given by
    \begin{align*}
        \beta_\Delta^0(x) &= \mu(x) = 
            \begin{cases}
                \frac{1}{\Delta}, & \text{if } x \in [-\frac{\Delta}{2}, \frac{\Delta}{2}]\\
                0,           & \text{otherwise}
            \end{cases} \\
           \beta_\Delta^d(x) &= \beta_\Delta^0 * \beta_\Delta^{d-1}(x) = 
               \underbrace{\beta_\Delta^0 * \dots * \beta_\Delta^0(x)}_{d+1 \text{concolution terms}}
    \end{align*}
    where $\mu(x)$ is the step function centered around 0. \todo[inline]{I'm not sure if this is sound}

    The zero-dimensional B-Spline, is the normalized unit impulse of width $\Delta$. And the 
    $d$-dimensional B-Spline is just the $d+1$ fold convolution of the normalized unit impulses
    \cite{horbelt_discretization_2002}.

    Note that
    \[ \beta_\Delta^d(x) = \frac{1}{\Delta} \beta_1^d(\frac{x}{\Delta}) \]
    therefore, if no specific subscript is mentioned, the B-Spline with $\Delta = 1$ is implied.

    In \cite{unser_fast_1991}, a non-recursive definition of the unit width B-Spline is given as:
    \[ \beta^d = \sum_{j=0}^{n+1} \frac{(-1)^j}{n!} \binom{n+1}{j}(x - j)\mu(x - j) \] 
\end{definition}

B-Splines have a couple of really attractive properties. B-Splines are the shortest and smoothest
scaling functions for a given order of approximation \cite{momey_b-spline_2012}. This
are close to a Gaussian function,
with a sufficiently large $d$ \cite{momey_b-spline_2012}, all while preserving compactness.


\cite{unser_b-spline_1993}
\cite{unser_b-spline_1993-1}
\cite{unser_fast_1991}
\cite{briand_theory_2018}
 
\section{Box splines}

What are box splines

\cite{entezari_box_2012}
\cite{de_boor_box_1993}


%%%%%%%%%%%%%%%%%%%%%%%%%%%%%%%%%%%%%%%%%%%%%%%%%%%%%%%%%%%%%%%%%%%%%%%%
% Implementation
%%%%%%%%%%%%%%%%%%%%%%%%%%%%%%%%%%%%%%%%%%%%%%%%%%%%%%%%%%%%%%%%%%%%%%%%
\part[Practical]{%
	Practical\\
	%
	% \vspace{1cm}
	% %
	% \begin{minipage}[l]{\textwidth}
	% 	%
	% 	\textnormal{%
	% 		\normalsize
	% 		%
	% 		\begin{singlespace*}
	% 			\onehalfspacing
	% 			%
	% 			After all the theory, let's focus on practical notions of previously stated
	% 			theory. Parts, which can be a little messier than the theory makes us believe.
	% 		\end{singlespace*}
	% 	}
	% \end{minipage}
}\label{part:practical}

\chapter{elsa}\label{chap:elsa}

So much implementation design

\chapter{Projector}\label{chap:projector}

As shown in previous chapters, calculation of the system matrix coefficients is one of the key
components of tomographic reconstruction. Thus, this has been an important part of research. The
routines or algorithms, which calculate the matrix are usually referred to as projectors. For the
forward projection, the contribution of each voxel to each detector pixel is calculated. For the
backward projection, the contribution of each detector pixel each a voxel is computed.

In the first part of this chapter, a detailed overview over existing research and implementations of
projectors is given. Next details on the projector used for this thesis is given. This is followed
by a detailed studied of accuracy and performance of the new projector compared to other projection
methods.

\section{Types of Projectors}\label{sec:projector_types}

One of the first mentions of projector routines can be found in the RECLBL library package
\insertref{RECLBL}. Namely, the two types of projectors introduced there are the
\textit{voxel-driven} (or \textit{pixel-driven} for the 2D case) and the \textit{ray driven}. Many
state of the art projectors still build on the ideas of these projectors.

The voxel driven approach is very simple. For the forward projection, each voxel is visited and the
voxel center is projected onto the detector. Finally, the contribution of the voxel to the detector,
is computed using some form of interpolation.~\cite{peters_algorithms_1981} used bilinear
interpolation between the two neighbouring detector pixels.~\cite{harauz_interpolation_1983}
improved on the approach by using bicubic spline interpolation. However, the approach is used rarely
due to the introduced artifacts (C.f.~\cite[Chapter 3.3]{levakhina_three-dimensional_2014}). I.e. if
the resolution of the detector is finer compared to the volume, detector pixels might never be
assigned a value. \todo{describe back projector properly as well}

Instead of projecting the voxel center and interpolating, one can project the complete voxel onto
the projector plane. This approach was taken by~\cite{long_3d_2010, long_3d_nodate} and is usually
referred to as separable footprint. They use trapezoidal functions to approximate the footprint both
accurately and efficiently. Hence, the contribution of voxels to the detector pixels is based on
these trapeziodial functions. It as also been ported to graphic processing units (GPUs) as shown in
~\cite{wu_gpu_2011, xie_effective_2015, chapdelaine_new_nodate}. To the best of my knowledge, for
voxel based approaches, this is the state of the art approach and outperforms other approaches.

This approach was translated to B-Splines in~\cite{momey_b-spline_2012, momey_spline_2015}. There
B-Splines are assumed to be the basis function at pixel centers, and then the B-Splines are
projected onto the projector plane. As shown in chapter about B-Spline basis functions
\insertref{B-Spline chapter}, the projection of \(n\)-dimensional B-Splines yield a
\(n-1\)-dimensional B-Spline, hence the projection is simple and
accurate.~\cite{momey_b-spline_2012} already incoorporated both parallel-beam and cone-beam
geometry.

Similarly,~\cite{ziegler_efficient_2006} proposed a footprint approach for blobs. And it was
improved and ported to the GPU by~\cite{bippus_projector_2011}.~\cite{kohler_iterative_2011}
describes a blob projector for phase-contrast CT\@.

A shared problem of the voxel-based approaches, is the challenge of parallel implementations. During
the forward projectiong, shared access to the projector pixels might be needed. Hence, mitigation
strategies must be developed.

A conceptually different approach compared to the voxel-based approach, is the \textit{ray-based}
approach. There, the key idea is to trace rays through the volume. Typically, the intersection
length is used as a contribution of a pixel to the detector pixel.

Note that this the forward projection of this approach, is trivially parallisable. Each ray can be
traced independently through the volume.

A classical ray-driven approach was proposed by~\cite{siddon_fast_1985}, often is is referred to as
Siddon's method. There, the exact path length of a ray traversing through a volume is calculated.
I.e.\, this is the exact calculation of the line integral of a single ray through the volume. More
recent work like~\cite{jacobs_fast_1998, christiaens_fast_1999, zhao_fast_2004, gao_fast_2012}
improved on the Siddon's method, especially in efficiency and
performance.~\cite{de_greef_accelerated_2009, xiao_efficient_2012} ported Siddon's methods to the
GPU\@.

Another classic is presented in~\cite{joseph_improved_1982}. It assumes a smooth image and
interpolates between neighbouring voxels along the ray path. It does so by a in slice-interpolation,
i.e.\ the voxels perpendicular to the main ray direction are visited.~\cite{graetz_high_2020}
proposed a branchless GPU version of this approach.

The backward projections of ray-driven approaches typically trace the ray from a detector pixel to
the source and update all visited voxels based on the weights calculated as in the forward
direction. And in the of Joseph's based methods, also voxels close by are updated. Note again here,
that parallelization is hindered by the possibility of shared write access to voxels.

Typically, iterative reconstruction algorithms expect that the forward and backward projectors are
adjoints of each other \insertref{forward and backward projectors are adjoints}, i.e.\ they are
\textit{matched} projector pair. This fails for basic pixel- and ray-driven approaches.
They either either introduce artifacts during the forward or backward projections.

An entirely different approach is the \textit{distance-driven} approach
~\cite{de_man_distance-driven_2002, de_man_distance-driven_2004}. There, the voxel boundaries and
detector pixel boundaries are projected onto a common axis, then the overlap is used as a weight.
This approach is still state of the art, however, is suffers from inaccuracys for projections, which
are close to 45°. A branchless GPU version was proposed in~\cite{liu_gpu-based_2017}. The
distance-driven approach doesn't suffer from any high-frequency artifacts.

Projectors based on blobs have been studied for quite some time. Ray-driven approaches are based on
algorithms proposed in~\cite{matej_practical_1996, popescu_ray_2004}. But rather than assuming a
infinitely thin ray, they assume a beam. Hence, they are somewhat similar to Joseph's projectors,
that they have to visit neighbouring voxels, as the support of blobs is larger than that of pixels.
~\cite{levakhina_distance-driven_2010} proposed a variant of the distance-driven approach based on
blobs.

Another area of projectors compute the intersection area of multiple rays. Such methods, as proposed
in~\cite{ha_study_2015, ha_efficient_2016, ha_look-up_2018}, calculate the intersection between rays
directed a detector boundaries.

\section{Implementation}\label{sec:implementation}


\chapter{Experiments}\label{chap:experiments}

\section{Reproducibility}\label{sec:experiments_repoduction}

I want this part to be as as reproducible as possible. Therefore, I ran as much as I can with
scripts, which can be found in the GitHub repository for this Thesis. You need to checkout elsa
\todo{add commit hash}, build it including the examples (please see elsa's documentation for that).
Then you'll be able to run all the scripts, with the last argument being the path to the
\textit{example\_argparse} binary, which you just build.

\section{Error measurements}\label{sec:error_measurements}

\section{Forward Projection}\label{sec:experiments_forward_projection}

\inlinetodo{Maybe you have time to do implement a analytic sinogram for blob phantoms}

As explained in details in the background part of thesis, the forward projection is in the case of
attenuation X-ray CT, the line integral through the object. It's quality if important for every
reconstruction task.

\inlinetodo{Rework with reformated figures}

In \autoref{fig:sinogram_shepp_logan}, one can see the sinogram of the Shepp-Logan phantom
\insertref{Reference to Shepp-Logan phantom} (see \autoref{fig:shepp_logan_phantom}). If one never
seen a sinogram, each row (in this case, someimtes it is turned 90°) represents the projection of a
certain angle. In this case, the first row is the projection with angle 0° and increments downwards
in equally spaced steps for a total arc of 360°. For this sonogram in total 1024 angles are
computed.

\begin{figure}[h]
	\centering
	\makebox[\textwidth]{ \makebox[1.3\textwidth]{%
			\begin{subfigure}[b]{0.3125\textwidth}
				% \centering
				\includegraphics[width=\textwidth]{./figures/experiments/forward_projection/2dsinogram_Blob.png}
				\caption{Blob based projector}\label{fig:sinogram_blob}
			\end{subfigure}
			\begin{subfigure}[b]{0.3125\textwidth}
				% \centering
				\includegraphics[width=\textwidth]{./figures/experiments/forward_projection/2dsinogram_BSpline.png}
				\caption{B-Spline based projector}\label{fig:sinogram_bspline}
			\end{subfigure}
			\begin{subfigure}[b]{0.3125\textwidth}
				% \centering
				\includegraphics[width=\textwidth]{./figures/experiments/forward_projection/2dsinogram_Siddon.png}
				\caption{Siddon's projector}\label{fig:sinogram_siddon}
			\end{subfigure}
			\begin{subfigure}[b]{0.3125\textwidth}
				% \centering
				\includegraphics[width=\textwidth]{./figures/experiments/forward_projection/2dsinogram_Joseph.png}
				\caption{Joseph's projector}\label{fig:sinogram_joseph}
			\end{subfigure}
		}} \\
	\begin{subfigure}[b]{0.3125\textwidth}
		% \centering
		\includegraphics[width=\textwidth]{./figures/experiments/forward_projection/2dphantom.png}
		\caption{Shepp-Logan Phantom}\label{fig:shepp_logan_phantom}
	\end{subfigure}
	\caption{(a)-(d) sinogram of Shepp-Logan phantom, (e) original shepp-logan phantom}%
	\label{fig:sinogram_shepp_logan}
\end{figure}

However, on this scale fine details will be lost. Therefore, a crop into the center left edge of the
sinogram is given in \autoref{fig:sinogram_shepp_logan_center_crop}. This is the region for the
projection angles around 180° degrees. The Siddon's projector in
\autoref{fig:sinogram_siddon_center_crop} exhibits the strongest artifacts. Similar artifacts are
present in regions around all multiples of 45° degrees (visible in \autoref{fig:sinogram_siddon}, if
zoomed in) Similar artifacts, These are the typical artifacts for ray-driven projectors. The
Joseph's projector in \autoref{fig:sinogram_joseph_center_crop} already reduces these artefacts
successfully.

\begin{figure}[h]
	\centering
	\makebox[\textwidth]{
		\makebox[1.3\textwidth]{%
			\begin{subfigure}[b]{0.3125\textwidth}
				% \centering
				\includegraphics[width=\textwidth]{./figures/experiments/forward_projection/2dsinogram_croped_Blob.png}
				\caption{Blob based projector}\label{fig:sinogram_blob_center_crop}
			\end{subfigure}
			\begin{subfigure}[b]{0.3125\textwidth}
				% \centering
				\includegraphics[width=\textwidth]{./figures/experiments/forward_projection/2dsinogram_croped_BSpline.png}
				\caption{B-Spline based projector}\label{fig:sinogram_bspline_center_crop}
			\end{subfigure}
			\begin{subfigure}[b]{0.3125\textwidth}
				% \centering
				\includegraphics[width=\textwidth]{./figures/experiments/forward_projection/2dsinogram_croped_Siddon.png}
				\caption{Siddon's projector}\label{fig:sinogram_siddon_center_crop}
			\end{subfigure}
			\begin{subfigure}[b]{0.3125\textwidth}
				% \centering
				\includegraphics[width=\textwidth]{./figures/experiments/forward_projection/2dsinogram_croped_Joseph.png}
				\caption{Joseph's projector}\label{fig:sinogram_joseph_center_crop}
			\end{subfigure}
		}}
	\caption{Crop of the left center of the sinogram to highlight artifacts}%
	\label{fig:sinogram_shepp_logan_center_crop}
\end{figure}

Both the Blob and B-Spline based projector don't exhibit these kind of artifacts as the Siddon's
projector. However, they also contain less artifacts than the Joseph's one. This is better visible
in the difference images, depicted in \autoref{fig:sino_differences}. Please note that these are
normalized absolute difference, i.e.\ the sinograms are normalized to the interval \(\mathopen[0,
	1\mathclose]\), and then the absolute value of the voxel-by-voxel difference is computed.

The first row of \autoref{fig:sino_differences} show the difference of the sinograms of the Siddon's
and Joseph's projector compared to the Blob based projector. The second row, the difference of
Siddon's and Joseph's to the B-Spline based on. For both projectors, the ring like artifacts from
the Siddon's projector are clearly visible again. However, now one can also see ring-like artifacts
in the difference to the Joseph's projector. They are clearly visible around each multiple of 45°
and 90° degree. Here, it is also clearly visible, that the Joseph's projector has the most artifacts
around the multiples of 45° degrees.

\autoref{fig:sino_diff_blob_bspline} shows the difference between the blob and B-Spline based
projectors. As expected they behave very much identical. The light outlines of the sinogram are
mostly due to the a small difference in brightness in the blob based projector. However, in the 45°
degree cases, they behave differently. \todo{They also different in something around 22.5° degrees,
	find out which exactly}

\begin{figure}[h]
	\centering
	\makebox[\textwidth]{
		\makebox[1.3\textwidth]{%
			\begin{subfigure}[b]{0.25\textwidth}
				\includegraphics[width=\textwidth]{./figures/experiments/forward_projection/2dsinodifference_Blob_Siddon.png}
				\caption{Blob to Siddon}\label{fig:sino_diff_blob_siddon}
			\end{subfigure}
			\begin{subfigure}[b]{0.25\textwidth}
				\includegraphics[width=\textwidth]{./figures/experiments/forward_projection/2dsinodifference_Blob_Joseph.png}
				\caption{Blob to Joseph}\label{fig:sino_diff_blob_joseph}
			\end{subfigure}
			\begin{subfigure}[b]{0.25\textwidth}
				\includegraphics[width=\textwidth]{./figures/experiments/forward_projection/2dsinodifference_BSpline_Siddon.png}
				\caption{B-Spline to Siddon}\label{fig:sino_diff_bspline_siddon}
			\end{subfigure}
			\begin{subfigure}[b]{0.25\textwidth}
				\includegraphics[width=\textwidth]{./figures/experiments/forward_projection/2dsinodifference_BSpline_Joseph.png}
				\caption{B-Spline to Joseph}\label{fig:sino_diff_bspline_joseph}
			\end{subfigure}
			\begin{subfigure}[b]{0.25\textwidth}
				\includegraphics[width=\textwidth]{./figures/experiments/forward_projection/2dsinodifference_Blob_BSpline.png}
				\caption{Blob to B-Spline}\label{fig:sino_diff_blob_bspline}
			\end{subfigure}
		}}
	\caption{Normalized absolute difference between different projectors}%
	\label{fig:sino_differences}
\end{figure}

Overall, one should note that the L2-norms of each sinogram is vastly differently. Especially, the
blob based projector is noticeable brighter. \todo{compute l2 norm and make table out of it, find
	explanation}

\inlinetodo{As of now, only the blob based images are normalized then compared, maybe stay
consistent}

\section{Reconstruciton of Synthetic Data}\label{sec:experiments_synthethic_projection}

In this chapter, many different reconstruction tasks are presentend to show different behaviour
under different conditions. All reconstructions are performed using the Fast Iterative
Shrinkage-Thresholding (FISTA) Algorithm \insertref{FISTA ref}. Usually, \(25\) iterations are
performed for the reconstruction unless mentioned otherwise. The regularization parameter is tuned
for each reconstruction to perform best visually. It is given in tables respective tables for the
reconstruction containing the error metrics~\ref{tab:error_metric_shepp_logan}.

\begin{figure}[h]
	\centering
	\makebox[\textwidth]{
		\makebox[1.3\textwidth]{%
			\begin{subfigure}[b]{0.25\textwidth}
				\includegraphics[width=\textwidth]{./figures/experiments/reconstruction_rectangle/2dphantom.png}
				\caption{Rectangular phantom}%
				\label{fig:rectangle_recon_phantom}
			\end{subfigure}
			\begin{subfigure}[b]{0.25\textwidth}
				\includegraphics[width=\textwidth]{./figures/experiments/reconstruction_rectangle/2dreconstruction_blob_croped.png}
				\caption{Blob based projector}%
				\label{fig:rectangle_recon_artifacts_blob}
			\end{subfigure}
			\begin{subfigure}[b]{0.25\textwidth}
				\includegraphics[width=\textwidth]{./figures/experiments/reconstruction_rectangle/2dreconstruction_bspline_croped.png}
				\caption{B-Spline based projector}%
				\label{fig:rectangle_recon_artifacts_bspline}
			\end{subfigure}
			\begin{subfigure}[b]{0.25\textwidth}
				\includegraphics[width=\textwidth]{./figures/experiments/reconstruction_rectangle/2dreconstruction_siddon_croped.png}
				\caption{Siddon's projector}%
				\label{fig:rectangle_recon_artifacts_siddon}
			\end{subfigure}
			\begin{subfigure}[b]{0.25\textwidth}
				\includegraphics[width=\textwidth]{./figures/experiments/reconstruction_rectangle/2dreconstruction_joseph_croped.png}
				\caption{Joseph's projector}%
				\label{fig:rectangle_recon_artifacts_joseph}
			\end{subfigure}
		}}
	\caption{(a) Rectangular phantom, (b) - (e) crop of top left corner of the reconstruction
		rectangle with different projectors}%
	\label{fig:rectangle_recon_artifacts}
\end{figure}

Looking at \autoref{fig:rectangle_recon_artifacts}, the original phantom is depicted at the bottom
(\autoref{fig:rectangle_recon_phantom}). It is a square, centered in the middle of the frame with a
constant value of 1. There, the first row show the blob and B-Spline based projectors, and the
second row the Siddon's and Joseph's projector. Clearly, in both the reconstruction of Siddon's and
Joseph's artifacts can be seen, which are not present in the blob and B-Spline based ones. These are
again due to the nature of the ray-driven approaches and the thin support of the rays.

\begin{figure}[h]
	\centering
	\makebox[\textwidth]{
		\makebox[1.3\textwidth]{%
			\begin{subfigure}[b]{0.25\textwidth}
				\includegraphics[width=\textwidth]{./figures/experiments/reconstruction_rectangle/2ddifference_Blob_croped.png}
				\caption{Blob to projector}%
				\label{fig:rectangle_recon_difference_blob}
			\end{subfigure}
			\begin{subfigure}[b]{0.25\textwidth}
				\includegraphics[width=\textwidth]{./figures/experiments/reconstruction_rectangle/2ddifference_BSpline_croped.png}
				\caption{B-Spline based projector}%
				\label{fig:rectangle_recon_difference_bspline}
			\end{subfigure}
			\begin{subfigure}[b]{0.25\textwidth}
				\includegraphics[width=\textwidth]{./figures/experiments/reconstruction_rectangle/2ddifference_Siddon_croped.png}
				\caption{Siddon's projector}%
				\label{fig:rectangle_recon_difference_siddon}
			\end{subfigure}
			\begin{subfigure}[b]{0.25\textwidth}
				\includegraphics[width=\textwidth]{./figures/experiments/reconstruction_rectangle/2ddifference_Joseph_croped.png}
				\caption{Joseph's projector}%
				\label{fig:rectangle_recon_difference_joseph}
			\end{subfigure}
			\begin{subfigure}[b]{0.25\textwidth}
				\includegraphics[width=\textwidth]{./figures/experiments/reconstruction_rectangle/2ddifference_Blob_BSpline_croped.png}
				\caption{Blob to B-Spline}%
				\label{fig:rectangle_recon_difference_blob_bspline}
			\end{subfigure}
		}}
	\caption{\subref{fig:rectangle_recon_difference_blob}--\subref{fig:rectangle_recon_difference_joseph}
        absolute difference image to phantom, \subref{fig:rectangle_recon_difference_blob_bspline}
        absolute difference between reconstruction using the blob and B-Spline projectors}%
	\label{fig:rectangle_recon_difference}

\end{figure}

Looking at the normalized absolute differences, shown in \autoref{fig:rectangle_recon_difference},
the artifacts already visible in the previous figure are again clearly visible. Interestingly, both
the blob based projector and the B-Spline based on are cleaner than the other approaches. They do
exhibit a touch of glow around the corner, but they are pretty consistent. Especially, compare to
Siddon's method, the glow around the corner is quite noisy. This is also true for the Siddon's based
one, but it's less severe. \todo{Add difference of blob to bspline one} No noticeable difference
between the blob and B-Spline based projectors can be notices.

\begin{figure}[h]
	\centering
	\makebox[\textwidth]{ \makebox[1.3\textwidth]{%
			\begin{subfigure}[b]{0.3125\textwidth}
				\includegraphics[width=\textwidth]{./figures/experiments/reconstruction_shepp_logan/2dreconstruction_blob.png}
				\caption{Blob based projector}%
				\label{fig:reconstruction_shepp_logan_blob}
			\end{subfigure}
			\begin{subfigure}[b]{0.3125\textwidth}
				\includegraphics[width=\textwidth]{./figures/experiments/reconstruction_shepp_logan/2dreconstruction_bspline.png}
				\caption{B-Spline based projector}%
				\label{fig:reconstruction_shepp_logan_bspline}
			\end{subfigure}
			\begin{subfigure}[b]{0.3125\textwidth}
				\includegraphics[width=\textwidth]{./figures/experiments/reconstruction_shepp_logan/2dreconstruction_siddon.png}
				\caption{Siddon's projector}%
				\label{fig:reconstruction_shepp_logan_siddon}
			\end{subfigure}
			\begin{subfigure}[b]{0.3125\textwidth}
				\includegraphics[width=\textwidth]{./figures/experiments/reconstruction_shepp_logan/2dreconstruction_joseph.png}
				\caption{Joseph's projector}%
				\label{fig:reconstruction_shepp_logan_joseph}
			\end{subfigure}
		}} \\
	\makebox[\textwidth]{
		\makebox[1.3\textwidth]{%
			\begin{subfigure}[b]{0.3125\textwidth}
				\includegraphics[width=\textwidth]{./figures/experiments/reconstruction_shepp_logan/2ddifference_blob.png}
				\caption{blob based projector}%
				\label{fig:reconstruction_shepp_logan_diff_blob}
			\end{subfigure}
			\begin{subfigure}[b]{0.3125\textwidth}
				\includegraphics[width=\textwidth]{./figures/experiments/reconstruction_shepp_logan/2ddifference_bspline.png}
				\caption{b-spline based projector}%
				\label{fig:reconstruction_shepp_logan_diff_bspline}
			\end{subfigure}
			\begin{subfigure}[b]{0.3125\textwidth}
				\includegraphics[width=\textwidth]{./figures/experiments/reconstruction_shepp_logan/2ddifference_siddon.png}
				\caption{siddon's projector}%
				\label{fig:reconstruction_shepp_logan_diff_siddon}
			\end{subfigure}
			\begin{subfigure}[b]{0.3125\textwidth}
				\includegraphics[width=\textwidth]{./figures/experiments/reconstruction_shepp_logan/2ddifference_joseph.png}
				\caption{joseph's projector}%
				\label{fig:reconstruction_shepp_logan_diff_joseph}
			\end{subfigure}
		}}
	\caption{Top row: Reconstruction of Shepp-Logan phantom using FISTA, Bottom row: differences
        to ground truth}%
	\label{fig:reconstruction_shepp_logan}
\end{figure}

Without trying to repeat the results to often, as a last synthetic data set, the reconstruction of
the Sheep-Logan phantom is shown the top row of \autoref{fig:reconstruction_shepp_logan}. Similarly,
to before both, the blob and B-Spline based projector exhibit no ring like artifacts and look quite
clean compared to both the Siddon and Joseph projectors. And again no noticeable difference, can be
seen between the blob and B-Spline projectors. \todo{also add difference between blob and B-Spline
based one}. This is again noticeable in the differences given in the bottom row of
\autoref{fig:reconstruction_shepp_logan}

\autoref{tab:error_metric_shepp_logan} shows the different error metrics for the reconstruction of
the Shepp-Logan phantom. They are all quite similar and all in the neighbourhood of \(0.00002\)
(MSR), \(0.001\) (RMSR), \(0.5\) (PSNR) or \(0.01\) (SSIM). \todo{validate via python/matlab
	scripts}

\begin{table}%
	\centering
	\csvreader[
		head to column names,
		separator=semicolon,
		tabular = cccccc,
		table head = \toprule \textbf{Projector} & \textbf{MSR} & \textbf{RMSR} & \textbf{MAE} & \textbf{PSNR} & \textbf{SSIM} \\\midrule,
		table foot = \bottomrule
	]{figures/experiments/reconstruction_shepp_logan/metrics.csv}{}{%
		\csvlinetotablerow%
	}
	\caption{Error metrics for the reconstruction of the Shepp-Logan phantom using FISTA}%
	\label{tab:error_metric_shepp_logan}
\end{table}

\subsection{Few Angle Reconstruction}\label{sec:experiments_few_angle}

\begin{table}%
	\centering
	\csvreader[
		head to column names,
		separator=semicolon,
		tabular = cccccc,
		table head = \toprule \textbf{Projector} & \textbf{MSR} & \textbf{RMSR} & \textbf{MAE} & \textbf{PSNR} & \textbf{SSIM} \\\midrule,
		table foot = \bottomrule
	]{figures/experiments/reconstruction_fewangles/metrics.csv}{}{%
		\csvlinetotablerow%
	}
	\caption{error metric for reconstruction of shepp-logan phantom using fista with few angles}%
	\label{tab:error_metric_shepp_logan_few_angles}
\end{table}

\begin{figure}[h]
	\centering
	\makebox[\textwidth]{
		\makebox[1.3\textwidth]{%
			\begin{subfigure}[b]{0.3125\textwidth}
				\includegraphics[width=\textwidth]{./figures/experiments/reconstruction_fewangles/2dreconstruction_blob.png}
				% \caption{blob based projector}%
				% \label{fig:reconstruction_fewangles_blob}
			\end{subfigure}
			\begin{subfigure}[b]{0.3125\textwidth}
				\includegraphics[width=\textwidth]{./figures/experiments/reconstruction_fewangles/2dreconstruction_bspline.png}
				% \caption{b-spline based projector}%
				% \label{fig:reconstruction_fewangles_bspline}
			\end{subfigure}
			\begin{subfigure}[b]{0.3125\textwidth}
				\includegraphics[width=\textwidth]{./figures/experiments/reconstruction_fewangles/2dreconstruction_siddon.png}
				% \caption{siddon's projector}%
				% \label{fig:reconstruction_fewangles_siddon}
			\end{subfigure}
			\begin{subfigure}[b]{0.3125\textwidth}
				\includegraphics[width=\textwidth]{./figures/experiments/reconstruction_fewangles/2dreconstruction_joseph.png}
				% \caption{joseph's projector}%
				% \label{fig:reconstruction_fewangles_joseph}
			\end{subfigure}
		}}
	\makebox[\textwidth]{
		\makebox[1.3\textwidth]{%
			\begin{subfigure}[b]{0.3125\textwidth}
				\includegraphics[width=\textwidth]{./figures/experiments/reconstruction_fewangles/2ddifference_blob.png}
				% \caption{blob based projector}%
				% \label{fig:reconstruction_diff_fewangles_blob}
			\end{subfigure}
			\begin{subfigure}[b]{0.3125\textwidth}
				\includegraphics[width=\textwidth]{./figures/experiments/reconstruction_fewangles/2ddifference_bspline.png}
				% \caption{b-spline based projector}%
				% \label{fig:reconstruction_diff_fewangles_bspline}
			\end{subfigure}
			\begin{subfigure}[b]{0.3125\textwidth}
				\includegraphics[width=\textwidth]{./figures/experiments/reconstruction_fewangles/2ddifference_siddon.png}
				% \caption{siddon's projector}%
				% \label{fig:reconstruction_diff_fewangles_siddon}
			\end{subfigure}
			\begin{subfigure}[b]{0.3125\textwidth}
				\includegraphics[width=\textwidth]{./figures/experiments/reconstruction_fewangles/2ddifference_joseph.png}
				% \caption{joseph's projector}%
				% \label{fig:reconstruction_diff_fewangles_joseph}
			\end{subfigure}
		}}
	\makebox[\textwidth]{
		\makebox[1.3\textwidth]{%
			\begin{subfigure}[b]{0.3125\textwidth}
				\includegraphics[width=\textwidth]{./figures/experiments/reconstruction_fewangles/2dreconstruction_blob_croped.png}
				% \caption{blob based projector}%
				% \label{fig:reconstruction_crop_fewangles_blob}
			\end{subfigure}
			\begin{subfigure}[b]{0.3125\textwidth}
				\includegraphics[width=\textwidth]{./figures/experiments/reconstruction_fewangles/2dreconstruction_bspline_croped.png}
				% \caption{b-spline based projector}%
				% \label{fig:reconstruction_crop_fewangles_bspline}
			\end{subfigure}
			\begin{subfigure}[b]{0.3125\textwidth}
				\includegraphics[width=\textwidth]{./figures/experiments/reconstruction_fewangles/2dreconstruction_siddon_croped.png}
				% \caption{siddon's projector}%
				% \label{fig:reconstruction_crop_fewangles_siddon}
			\end{subfigure}
			\begin{subfigure}[b]{0.3125\textwidth}
				\includegraphics[width=\textwidth]{./figures/experiments/reconstruction_fewangles/2dreconstruction_joseph_croped.png}
				% \caption{joseph's projector}%
				% \label{fig:reconstruction_crop_fewangles_joseph}
			\end{subfigure}
		}}
	\caption{top row: Reconstruction of Shepp-Logan phantom with few angles, middle row:
        differences to ground truth, bottom row: crop into portion of the reconstruction,from left
        to right: the blob based projector, the B-Spline based projector, Siddon's projector,
        Joseph's projector}%
	\label{fig:reconstruction_fewangles}
\end{figure}

\subsection{Behaviour in the Presence of Noise}

\inlinetodo{extent elsa example to be able to add noise}

\section{Reconstruction of Medical Data}\label{sec:experiments_medical_projection}

\inlinetodo{extent elsa example to be able to read phantom data in}

\section{Quality Overview}\label{sec:experiments_quality_projection}

\subsection{Reduced artifacts in Forward Projection}\label{sec:experiments_artifacts_forward}

\subsection{Reduced artifacts in Reconstruction}\label{sec:experiments_artifacts_reconstruction}

\section{Performance Overview}\label{sec:experiments_performance_projection}


%%%%%%%%%%%%%%%%%%%%%%%%%%%%%%%%%%%%%%%%%%%%%%%%%%%%%%%%%%%%%%%%%%%%%%%%
% Conclusion
%%%%%%%%%%%%%%%%%%%%%%%%%%%%%%%%%%%%%%%%%%%%%%%%%%%%%%%%%%%%%%%%%%%%%%%%
\part[Conclusion]{%
	Conclusion\\
	%
	% \vspace{1cm}
	% %
	% \begin{minipage}[l]{\textwidth}
	% 	%
	% 	\textnormal{%
	% 		\normalsize
	% 		%
	% 		\begin{singlespace*}
	% 			\onehalfspacing
	% 			%
	% 			Now we wrap it all up
	% 		\end{singlespace*}
	% 	}
	% \end{minipage}
}\label{part:conclusion}

\chapter{Summary}\label{chap:summary}

This thesis provides an overview of the three important aspects of tomographic reconstruction, the
forward model, the discretization and finding the solution.

The forward model of X-ray attenuation CT is discussed, both a practical and engineering focused
view is presented followed by a mathematical centered discussion. Similarly, but with a little less
detail, this was discussed for differential phase-contrast CT\@. Further, analytical and iterative
reconstruction algorithm are discussed. Many important steps for tomographic reconstruction are
covered.

The discretization is discussed in depth. Specifically, the discretization via the series expansion
is explained along side 3 different basis functions: the voxel, the family of spherically-symmetric
(blob) and B-Spline basis functions. The blob basis function as proposed by
\citeauthor*{lewitt_multidimensional_1990} and B-Spline basis functions provide a better
discretization of a given signal compared to the voxel basis function. Blob basis functions are
quite well studied and well understood in the context of tomographic reconstruction. Notably, they
provide a closed form solution for the Radon Transform and are differentiable. However, they also
are parameterized using \(3\) parameters, which can become rather complex to optimize.

On the other hand, cubic B-Splines are relatively simple. They are also differentiable, but they do
not have a analytic solution for the X-ray transform and need approximation. Interestingly though,
they still have comparable accuracy with blobs. And together with their simple nature, they are an
interesting choice of basis function for tomographic reconstruction.

The practical part of this thesis implements two projectors and contributes them to the C++
framework elsa. One is based on the blob basis function and the other on cubic B-Splines. Hence,
both use differential basis function. This should enable the usage of these projectors in the
context of differential phase-contrast CT\@.

The forward projection of both projectors is qualitatively better than voxel based methods such as
Siddon's or Joseph's method. In simple to medium complex synthetic phantoms, the new projectors do
not exhibit of high-frequency artifacts in the reconstruction as well. Not only that, but they also
better reconstruction in cases where the measurements were noisy.  The B-Spline projector does show
artifacts in certain scenarios and the blob based projector has a drop in the error metrics for the
most complex synthetic phantom. Still, they both are superior choices, if accuracy is of high
importance.

From a computational standpoint, both projectors have considerably longer runtime than both the
Siddon's and Joseph's method. This is especially true for the three-dimensional case. In the
two-dimensional case there is around a six times performance penalty to use the new projectors. As,
instead of \(1\) or \(2\) voxels (for the Siddon's or Joseph's respectively) \(5\) voxels are
visited for both basis functions, this is what is to be expected. However, in the three-dimensional
case, the penalty riced to around \(22\) times. As now around \(25\) voxels are visited for each
voxel Siddon's method visits.

\chapter{Future Work}\label{chap:future_work}

The elephant in the room for future work --- at least for me --- is support for differential
phase-contrast in elsa. Much of this work is dedicated to emphasize the importance of differential
phase-contrast, but due to the lacking support, no experiments could be run using the newly
implemented methods.

With this thesis, a part of the forward model is already done. There are now projectors based on
differential basis functions. However, at least in elsa, the exact analytical formulations for the
derivatives is not implemented yet. This most likely is the largest piece left. A new projector can
easily be build around that with the code already present. This would provide a basic support. Other
aspects of the reconstructions might need special handling or special care for the reconstruction of
differential phase-contrast images. This might include specific reconstruction algorithms or pre-
and post-processing.

Another aspect, which this thesis did not touch on, is the signal extraction from grating-based
imaging setups. As described in \autoref{sec:phasecontrast_ct}, grating-based setups can measure
attenuation, phase-shift and scattering. There, it is described that the attenuation is measured as
a drop in average intensity during the stepping. The phase-shift as a shift of the signal, and
scattering as a decreased amplitude. However, in a real measurement these three are intermixed and
not easily split apart. Hence, reading a raw measurement from a grating-based system requires
processing until the differential phase-contrast signal is extracted and can be reconstructed. To
provide first-class support for differential phase-contrast, this needs to be dealt with in some
way.

A topic only briefly touched on in this thesis, is (anisotropic) X-ray dark-field tomography. A
similar processing step as for differential phase-contrast is necessary to retrieve the dark-field
signal from the raw measurements. It is a extremely interesting topic and deserves much attention.
Hopefully elsa will have proper support for it sooner rather than later.

Another aspects, which concerns itself more with details of the projectors, is the exact
implementation of the new projectors. The current implementation is ray-driven. Both
\citeauthor*{kohler_iterative_2011} in~\cite{kohler_iterative_2011} and
\citeauthor*{momey_spline_2015} in~\cite{momey_spline_2015} propose footprint based methods for blob
and B-Spline basis functions respectively. Based on the improvements of footprint based methods over
ray-driven methods using the voxel-basis functions provides (c.f.\ \cite{long_3d_2010}), the
footprints based method could increase the accuracy and/or improve performance of the projectors.
Another interesting investigation would be the blob based distance driven method, as it is proposed
by \citeauthor*{levakhina_distance-driven_2010}~\cite{levakhina_distance-driven_2010}.

As one can imagine the runtime performance of the newly implemented projectors can be improved. For
one, the current code allocates memory each iteration, which is a high cost and might even make
reduce the runtime by around \(30\%\) to \(50\%\). Please note, that these numbers are only based on
some preliminary profiling conducted. Another lacking feature is GPU support. Tomographic
reconstruction is a prime example for the performance GPU's bring to the table. And so the
projectors implemented here would also benefit of it.

\begin{flushright}
	With that said, there is much left to do, let's get to work!
\end{flushright}

% This ensures that the subsequent sections are being included as root
% items in the bookmark structure of your PDF reader.
\bookmarksetup{startatroot}
\backmatter

\begingroup
\let\clearpage\relax
\glsaddall
\printglossary[type=\acronymtype]
\newpage
\endgroup

\printindex
\printbibliography

\end{document}
