\documentclass{mimosis}

\usepackage{metalogo}

%%%%%%%%%%%%%%%%%%%%%%%%%%%%%%%%%%%%%%%%%%%%%%%%%%%%%%%%%%%%%%%%%%%%%%%%
% Some of my favourite personal adjustments
%%%%%%%%%%%%%%%%%%%%%%%%%%%%%%%%%%%%%%%%%%%%%%%%%%%%%%%%%%%%%%%%%%%%%%%%
%
% These are the adjustments that I consider necessary for typesetting
% a nice thesis. However, they are *not* included in the template, as
% I do not want to force you to use them.

% This ensures that I am able to typeset bold font in table while still aligning the numbers
% correctly.
\usepackage{etoolbox}
\usepackage{csvsimple}

% use for stackinset
\usepackage[usestackEOL]{stackengine}
\usepackage{caption}
\usepackage{subcaption}

% Use svg
\usepackage{svg}

% Render incscape pdf's with equation
\usepackage{import}
\usepackage{xifthen}
\usepackage{pdfpages}
\usepackage{transparent}

% https://tex.stackexchange.com/a/282
\usepackage[section]{placeins}
%%%%%%%%%%%%%%%%%%%%%%%%%%%%%%%%%%%%%%%%%%%%%%%%%%%%%%%%%%%%%%%%%%%%%%%%
% Colors
%%%%%%%%%%%%%%%%%%%%%%%%%%%%%%%%%%%%%%%%%%%%%%%%%%%%%%%%%%%%%%%%%%%%%%%%
\definecolor{black}               {RGB}{  0,   0,  0}
\definecolor{kelly_green}         {RGB}{ 76, 187, 23}
\definecolor{international_orange}{RGB}{255,  79,  0}
\definecolor{venetian_red}        {RGB}{200,   8, 21}

\definecolor{amber}     {RGB}{255,191,  0}
\definecolor{burgundy}  {RGB}{144,  0, 32}
\definecolor{cardinal}     {RGB}{247, 118, 142}
\definecolor{deep-azure}{RGB}{ 10, 50, 94}
\definecolor{mahogany}  {RGB}{192, 64,  0}
\definecolor{malory-red}{RGB}{192, 46, 16}
\definecolor{sky}       {RGB}{ 62, 93,240}
\definecolor{teal}      {RGB}{  0,128,128}
\definecolor{tyrian}    {RGB}{102,  2, 60}
\definecolor{yale}      {RGB}{ 70,130,180}

\definecolor{quality-high}  {RGB}{ 70,130,180}
\definecolor{quality-medium}{RGB}{255,191,  0}
\definecolor{quality-low}   {RGB}{196, 30, 58}
%
% Colour for Part numbers
%
\renewcommand*{\partformat}{%
	\textcolor{cardinal}{Part \thepart}\autodot\enskip%
}

%
% Colour for section numbers
%
\renewcommand*{\sectionformat}{%
	\textcolor{cardinal}{\thesection}\autodot\enskip%
}

%
% Colour for subsection numbers
%
\renewcommand*{\subsectionformat}{%
	\textcolor{cardinal}{\thesubsection}\autodot\enskip%
}

%
% Colour for subsubsection numbers
%
\renewcommand*{\subsubsectionformat}{%
	\textcolor{cardinal}{\thesubsubsection}\autodot\enskip%
}

%ornamet
\newcommand{\ornament}{%
	\begin{center}
		\large\textcolor{cardinal}{\ding{70}}
	\end{center}
}

% \pagecolor[RGB]{26,27,38}
% \color[RGB]{192, 202, 245}

% \pagecolor[RGB]{213, 214, 219}
% \color[RGB]{52, 59, 88}

%%%%%%%%%%%%%%%%%%%%%%%%%%%%%%%%%%%%%%%%%%%%%%%%%%%%%%%%%%%%%%%%%%%%%%%%
% Figure commands
%%%%%%%%%%%%%%%%%%%%%%%%%%%%%%%%%%%%%%%%%%%%%%%%%%%%%%%%%%%%%%%%%%%%%%%%

% Then I can import matplotlib stuff
\usepackage{pgf}

% Usage of includegraphicsmaybe
% \begin{figure}[t]
%         \centering
%         \includegraphicsmaybe[width=0.24\textwidth]{asdasdfigures/experiments/forward_projection/2dphantom.png}
%         \caption{Test maybe image\inlinetodo{Put figure here}}\label{fig:test_maybe}
% \end{figure}
\newcommand{\includegraphicsmaybe}[2][]{\IfFileExists{#2}{\includegraphics[#1]{#2}}{\includegraphics{example-image}}}

% \newcommand{\incfig}[2]{%
% 	\def\svgwidth{\columnwidth}
% 	\import{#1}{#2.pdf_tex}
% }
\newcommand{\incfig}[1]{%
	\def\svgwidth{\columnwidth}
	\input{#1.pdf_tex}
}
\newcommand{\incfigmaybe}[1]{\IfFileExists{#1.pdf_tex}{\def\svgwidth{\columnwidth}\input{#1.pdf_tex}}{\includegraphics{example-image}}}

% This avoid nasty errors, but I also need symbolic links from subfolder to figures folder
\graphicspath{{figures/}}
% \newcommand{\incfigmaybe}[1]{\IfFileExists{#1}{\incfig{#1}}{\includegraphics{example-image-a}}}
%%%%%%%%%%%%%%%%%%%%%%%%%%%%%%%%%%%%%%%%%%%%%%%%%%%%%%%%%%%%%%%%%%%%%%%%
% Sectioning of thesis
%%%%%%%%%%%%%%%%%%%%%%%%%%%%%%%%%%%%%%%%%%%%%%%%%%%%%%%%%%%%%%%%%%%%%%%%

% I want subsubsections to have numbering as well
\setcounter{secnumdepth}{\subsubsectionnumdepth}

%%%%%%%%%%%%%%%%%%%%%%%%%%%%%%%%%%%%%%%%%%%%%%%%%%%%%%%%%%%%%%%%%%%%%%%%
% Code in latex
%%%%%%%%%%%%%%%%%%%%%%%%%%%%%%%%%%%%%%%%%%%%%%%%%%%%%%%%%%%%%%%%%%%%%%%%

\usepackage[outputdir=build,chapter]{minted}
\usemintedstyle{friendly}

%%%%%%%%%%%%%%%%%%%%%%%%%%%%%%%%%%%%%%%%%%%%%%%%%%%%%%%%%%%%%%%%%%%%%%%%
% Hyperlinks & bookmarks
%%%%%%%%%%%%%%%%%%%%%%%%%%%%%%%%%%%%%%%%%%%%%%%%%%%%%%%%%%%%%%%%%%%%%%%%

\usepackage[%
	colorlinks = true,
	citecolor  = cardinal,
	linkcolor  = cardinal,
	urlcolor   = cardinal,
	unicode = true,
]{hyperref}

\usepackage{bookmark}

%%%%%%%%%%%%%%%%%%%%%%%%%%%%%%%%%%%%%%%%%%%%%%%%%%%%%%%%%%%%%%%%%%%%%%%%
% TODO notes
%%%%%%%%%%%%%%%%%%%%%%%%%%%%%%%%%%%%%%%%%%%%%%%%%%%%%%%%%%%%%%%%%%%%%%%%

\usepackage[colorinlistoftodos,prependcaption,textsize=tiny]{todonotes}
\setuptodonotes{noshadow, color=amber}

% If a ref is missing
\newcommand{\insertref}[1]{\todo[color=green!40]{MISSING REF: #1}}
\newcommand{\inlinetodo}[1]{\todo[inline]{#1}}
\newcommand{\missingfig}[1]{\missingfigure[figcolor=white]{#1}}

%%%%%%%%%%%%%%%%%%%%%%%%%%%%%%%%%%%%%%%%%%%%%%%%%%%%%%%%%%%%%%%%%%%%%%%%
% Math relatex settings
%%%%%%%%%%%%%%%%%%%%%%%%%%%%%%%%%%%%%%%%%%%%%%%%%%%%%%%%%%%%%%%%%%%%%%%%


\usepackage{thmtools}

\declaretheoremstyle[
	headpunct={},
	headfont=\color{cardinal}\normalfont\bfseries,
	notefont=\itshape,
	bodyfont=\normalfont,
]{definitioncolor}

\declaretheorem[
	style=definitioncolor,
	name=Definition,
	numberwithin=chapter
]{definition}

\usepackage{mathtools}

% Define \abs
\DeclarePairedDelimiter\abs{\lvert}{\rvert}%

% Define \norm
\DeclarePairedDelimiter\norm{\lVert}{\rVert}%

% Define \sign
\DeclareMathOperator{\sign}{sign}

% Define Radon Transform
\newcommand{\radon}{\mathscr{R}}
% Define X-ray Transform
\newcommand{\xray}{\mathscr{X}}
% definitions for vectors
\newcommand{\mvec}[1]{\symbfit{#1}}
% approx something
\newcommand{\near}[1]{\hat{#1}}
%%%%%%%%%%%%%%%%%%%%%%%%%%%%%%%%%%%%%%%%%%%%%%%%%%%%%%%%%%%%%%%%%%%%%%%%
% Bibliography
%%%%%%%%%%%%%%%%%%%%%%%%%%%%%%%%%%%%%%%%%%%%%%%%%%%%%%%%%%%%%%%%%%%%%%%%
%
% I like the bibliography to be extremely plain, showing only a numeric
% identifier and citing everything in simple brackets. The first names,
% if present, will be initialized. DOIs and URLs will be preserved.

\usepackage[%
	autocite     = plain,
	backend      = biber,
	doi          = true,
	url          = true,
	giveninits   = true,
	hyperref     = true,
	maxbibnames  = 99,
	maxcitenames = 99,
	sortcites    = true,
	style        = numeric,
]{biblatex}

%%%%%%%%%%%%%%%%%%%%%%%%%%%%%%%%%%%%%%%%%%%%%%%%%%%%%%%%%%%%%%%%%%%%%%%%
% Some adjustments to make the bibliography more clean
%%%%%%%%%%%%%%%%%%%%%%%%%%%%%%%%%%%%%%%%%%%%%%%%%%%%%%%%%%%%%%%%%%%%%%%%
%
% The subsequent commands do the following:
%  - Removing the month field from the bibliography
%  - Fixing the Oxford commma
%  - Suppress the "in" for journal articles
%  - Remove the parentheses of the year in an article
%  - Delimit volume and issue of an article by a colon ":" instead of
%    a dot ""
%  - Use commas to separate the location of publishers from their name
%  - Remove the abbreviation for technical reports
%  - Display the label of bibliographic entries without brackets in the
%    bibliography
%  - Ensure that DOIs are followed by a non-breakable space
%  - Use hair spaces between initials of authors
%  - Make the font size of citations smaller
%  - Fixing ordinal numbers (1st, 2nd, 3rd, and so) on by using
%    superscripts

% Remove the month field from the bibliography. It does not serve a good
% purpose, I guess. And often, it cannot be used because the journals
% have some crazy issue policies.
\AtEveryBibitem{\clearfield{month}}
\AtEveryCitekey{\clearfield{month}}

% Fixing the Oxford comma. Not sure whether this is the proper solution.
% More information is available under [1] and [2].
%
% [1] http://tex.stackexchange.com/questions/97712/biblatex-apa-style-is-missing-a-comma-in-the-references-why
% [2] http://tex.stackexchange.com/questions/44048/use-et-al-in-biblatex-custom-style
%
\AtBeginBibliography{%
  \renewcommand*{\finalnamedelim}{%
    \ifthenelse{\value{listcount} > 2}{%
      \addcomma
      \addspace
      \bibstring{and}%
    }{%
      \addspace
      \bibstring{and}%
    }
  }
}

% Suppress "in" for journal articles. This is unnecessary in my opinion
% because the journal title is typeset in italics anyway.
\renewbibmacro{in:}{%
  \ifentrytype{article}
  {%
  }%
  % else
  {%
    \printtext{\bibstring{in}\intitlepunct}%
  }%
}

% Remove the parentheses for the year in an article. This removes a lot
% of undesired parentheses in the bibliography, thereby improving the
% readability. Moreover, it makes the look of the bibliography more
% consistent.
\renewbibmacro*{issue+date}{%
  \setunit{\addcomma\space}
    \iffieldundef{issue}
      {\usebibmacro{date}}
      {\printfield{issue}%
       \setunit*{\addspace}%
       \usebibmacro{date}}%
  \newunit}

% Delimit the volume and the number of an article by a colon instead of
% by a dot, which I consider to be more readable.
\renewbibmacro*{volume+number+eid}{%
  \printfield{volume}%
  \setunit*{\addcolon}%
  \printfield{number}%
  \setunit{\addcomma\space}%
  \printfield{eid}%
}

% Do not use a colon for the publisher location. Instead, connect
% publisher, location, and date via commas.
\renewbibmacro*{publisher+location+date}{%
  \printlist{publisher}%
  \setunit*{\addcomma\space}%
  \printlist{location}%
  \setunit*{\addcomma\space}%
  \usebibmacro{date}%
  \newunit%
}

% Ditto for other entry types.
\renewbibmacro*{organization+location+date}{%
  \printlist{location}%
  \setunit*{\addcomma\space}%
  \printlist{organization}%
  \setunit*{\addcomma\space}%
  \usebibmacro{date}%
  \newunit%
}

% Display the label of a bibliographic entry in bare style, without any
% brackets. I like this more than the default.
%
% Note that this is *really* the proper and official way of doing this.
\DeclareFieldFormat{labelnumberwidth}{#1\adddot}

% Ensure that DOIs are followed by a non-breakable space.
\DeclareFieldFormat{doi}{%
  \mkbibacro{DOI}\addcolon\addnbspace
    \ifhyperref
      {\href{http://dx.doi.org/#1}{\nolinkurl{#1}}}
      %
      {\nolinkurl{#1}}
}

% Use proper hair spaces between initials as suggested by Bringhurst and
% others.
\renewcommand*\bibinitdelim {\addnbthinspace}
\renewcommand*\bibnamedelima{\addnbthinspace}
\renewcommand*\bibnamedelimb{\addnbthinspace}
\renewcommand*\bibnamedelimi{\addnbthinspace}

% Make the font size of citations smaller. Depending on your selected
% font, you might not need this.
\renewcommand*{\citesetup}{%
  \biburlsetup
  \small
}

\DeclareLanguageMapping{english}{english-mimosis}

\addbibresource{bibliography.bib}

%%%%%%%%%%%%%%%%%%%%%%%%%%%%%%%%%%%%%%%%%%%%%%%%%%%%%%%%%%%%%%%%%%%%%%%%
% Fonts
%%%%%%%%%%%%%%%%%%%%%%%%%%%%%%%%%%%%%%%%%%%%%%%%%%%%%%%%%%%%%%%%%%%%%%%%

\ifxetexorluatex%
	\usepackage{amssymb}
	\let\mathbbalt\mathbb
	\usepackage{unicode-math}
	\let\mathbb\mathbbalt% UNIVERSAL RESET TO ORIGINAL \mathbb
	\setmainfont{IBM Plex Serif}[Scale=MatchLowercase]
	\setsansfont{IBM Plex Sans}[Scale=MatchLowercase]
	\setmonofont{IBM Plex Mono}[Scale=MatchLowercase]
	% Try again after restart
	% \setmathfont[range={\mathscr,\mathbfscr}]{XITS Math}
\else
	\usepackage[lf]{ebgaramond}
	\usepackage[oldstyle,scale=0.7]{sourcecodepro}
	\singlespacing%
\fi

\newacronym[description={Principal component analysis}]{PCA}{PCA}{principal component analysis}
\newacronym                                            {SNF}{SNF}{Smith normal form}
\newacronym[description={Topological data analysis}]   {TDA}{TDA}{topological data analysis}
\newacronym[description={Computational Imaging and Inverse Problems research group}]{CIIP}{CIIP}{Computational Imaging and Inverse Problems}
\newacronym{TUM}{TUM}{Technical University of Munich}
\newacronym{CT}{CT}{computed tomography}
\newacronym{ART}{ART}{Algebraic Reconstruction Technique}
\newacronym{SIRT}{SIRT}{Simultaneous Iterative Reconstruction Technique}
\newacronym{ISTA}{ISTA}{Iterative Shrinkage-Thresholding Algorithms}
\newacronym{FISTA}{FISTA}{Fast Iterative Shrinkage-Thresholding Algorithm}
\newacronym{CPU}{CPU}{Central Processing Unit}
\newacronym{GPU}{GPU}{Graphics Processing Unit}
\newacronym{LUT}{LUT}{Look-up Table}
\newacronym{MSE}{MSE}{Mean Squared Error}
\newacronym{RMSE}{RMSE}{Root Mean Square Error}
\newacronym{NRMSE}{NRMSE}{Normalized Root Mean Square Error}
\newacronym{PSNR}{PSNR}{Peak Signal to Noise Ration}
\newacronym{SSIM}{SSIM}{Structural Similarity Index Measure}
\newacronym{BSpline}{B-Spline}{Basic-Spline}

\makeindex%
\makeglossaries%

%%%%%%%%%%%%%%%%%%%%%%%%%%%%%%%%%%%%%%%%%%%%%%%%%%%%%%%%%%%%%%%%%%%%%%%%
% Nomenclature
%%%%%%%%%%%%%%%%%%%%%%%%%%%%%%%%%%%%%%%%%%%%%%%%%%%%%%%%%%%%%%%%%%%%%%%%
\usepackage{nomencl}
\makenomenclature%

%%%%%%%%%%%%%%%%%%%%%%%%%%%%%%%%%%%%%%%%%%%%%%%%%%%%%%%%%%%%%%%%%%%%%%%%
% Ordinals
%%%%%%%%%%%%%%%%%%%%%%%%%%%%%%%%%%%%%%%%%%%%%%%%%%%%%%%%%%%%%%%%%%%%%%%%

\makeatletter
\@ifundefined{st}{%
	\newcommand{\st}{\textsuperscript{\textup{st}}\xspace}
}{}
\@ifundefined{rd}{%
	\newcommand{\rd}{\textsuperscript{\textup{rd}}\xspace}
}{}
\@ifundefined{nd}{%
	\newcommand{\nd}{\textsuperscript{\textup{nd}}\xspace}
}{}
\makeatother

\renewcommand{\th}{\textsuperscript{\textup{th}}\xspace}

%%%%%%%%%%%%%%%%%%%%%%%%%%%%%%%%%%%%%%%%%%%%%%%%%%%%%%%%%%%%%%%%%%%%%%%%
% Incipit
%%%%%%%%%%%%%%%%%%%%%%%%%%%%%%%%%%%%%%%%%%%%%%%%%%%%%%%%%%%%%%%%%%%%%%%%

%%%%%%%%%%%%%%%%%%%%%%%%%%%%%%%%%%%%%%%%%%%%%%%%%%%%%%%%%%%%%%%%%%%%%%%%
% General Information about my thesis
%%%%%%%%%%%%%%%%%%%%%%%%%%%%%%%%%%%%%%%%%%%%%%%%%%%%%%%%%%%%%%%%%%%%%%%%
\newcommand*{\getUniversity}{Technische Universität München}
\newcommand*{\getTitleEng}{Differentiable projection operations for X-ray computed tomography}
\newcommand*{\getTitleGer}{Differenzierbare Projektionsoperatoren für Röntgen-Computertomographie}
\newcommand*{\getFaculty}{Department of Informatics}
\newcommand*{\getDoctype}{Master's Thesis in Data Engineering and Analytics}
\newcommand*{\getSupervisor}{PD Dr.\ rer.\ nat.\ Tobias Lasser}
\newcommand*{\getAdvisor}{Advisor}
\newcommand*{\getAuthor}{David Frank}
\newcommand*{\getSubmissionDate}{15.06.2021}
\newcommand*{\getSubmissionLocation}{Munich}


\title{\texttt{\getTitleEng}}
\subtitle{\texttt{\getTitleGer}}

\author{\getAuthor{}}

\begin{document}

\frontmatter
\begin{titlepage}
    \makeatletter
    \begin{center}
        % Let's put the TUM logo here
        \includegraphics[height=20mm]{figures/logos/tum_logo.png}\\[0.5cm]
        % Department
        \begin{Huge}
            \MakeUppercase{\getFaculty}
        \end{Huge}\\[0.5cm]
        % University
        \begin{large}
            \MakeUppercase{\getUniversity}
        \end{large}\\[2cm]
        % Type of thesis
        \begin{Large}
            \getDoctype%
        \end{Large}\\[2cm]
        %
        \begin{Huge}
            \@title\par
        \end{Huge}
        \vspace{5mm}
        %
        \begin{Large}
            \@subtitle
        \end{Large}\\
        %
        \emph{by}\\
        \getAuthor
        % Let's inmclude the Informatics logo as well
        \vfill
        \includegraphics[height=20mm]{figures/logos/info_logo.png}\\
    \end{center}
    % Directly append the title page, to avoid the empty page...
    \newpage
    \thispagestyle{empty}
    \begin{center}
        % Let's put the TUM logo here
        \includegraphics[height=20mm]{figures/logos/tum_logo.png}\\[0.5cm]
        % Department
        \begin{Huge}
            \MakeUppercase{\getFaculty}
        \end{Huge}\\[0.5cm]
        % University
        \begin{large}
            \MakeUppercase{\getUniversity}
        \end{large}\\[2cm]
        % Type of thesis
        \begin{Large}
            \getDoctype%
        \end{Large}\\[1.5cm]
        %
        \begin{Huge}
            \@title\par
        \end{Huge}
        \vspace{10mm}
        %
        \begin{Huge}
            \texttt{\getTitleGer{}}
        \end{Huge}
        \vspace{10mm}
        %
        \begin{tabular}{l l}
            Author:          & \getAuthor{} \\
            Supervisor:      & \getSupervisor{} \\
            Submission Date: & \getSubmissionDate{} \\
        \end{tabular}
        % Let's inmclude the Informatics logo as well
        \vfill
        \includegraphics[height=20mm]{figures/logos/info_logo.png}\\
    \end{center}
    \makeatother
\end{titlepage}

\thispagestyle{empty}
\vspace*{0.8\textheight}
\noindent
I confirm that this master's thesis is my own work and I have documented all sources and material used.

\vspace{15mm}
\noindent
\getSubmissionLocation{}, \getSignatureDate{} \hspace{50mm} \getAuthor{}

\thispagestyle{empty}
\chapter*{Acknowledgements}
\noindent%

Firstly, of course, I have to thank my supervisor Tobias Lasser! We have been working together now
for more than 5 years. It all started with a Bachelor seminar, turned into my Bachelor's thesis,
followed by many years of research assistant positions.

During the last years of my studies, I found myself struggling with the pressure and stress. Not
only during that time, but in general, Tobias always supported me in many ways, writing letters of
recommendation, providing me an amazing opportunity to do an internship in Denmark (shootout to Per
Christian Hansen and Hans Henrik Sørensen, both are amazing people), providing me access to
facilities or just show interest in weird my interest.

I sincerely thank Tobias, for the freedom he offered me. During all the different projects we worked
on, I never had the feeling of doing something, I was forced to and did not like myself. Rather, it
was the other way around. I could come up with esoteric ideas and given enough reason, or enough
persistence, I could spend time on it and explore. This is and was a great way to grow for me
personally. I never got the feeling of Tobias being my superior, but rather I felt treated as equal.
I've worked in a couple of industry jobs during the years, and though I learned a ton there as well,
I always longed for this exact feeling and came back. And I can't thank Tobias enough for that.

Then I also want to thank my family. I want to thank my parents. They supported me in so many ways,
throughout my studies. Of course, they provided me with the chances to my education and supported me
financially. But they also supported me emotionally. Both of my parents shaped me a great deal. I
fell like, I take after my mother more, intertwined with the humor and geekiness of my father. And
personally, I'm quite delighted with this specific mix.

I want to thank my brother, he is quite a bit older than I am, but throughout my youth, I loved
watching him play video games. This and his geekiness for computers and tech shaped many of my
interests, I still have till today. Precisely those interested brought me to study computer science
in the first place.

Big thanks to my sister. We grew up together and lived in a room for the best part of our live. But
during that time, we've laughed so so much. Till today, we share much of our humor and I don't think
that there is anyone out there, who understands my humor better than her. None can understand my
thought jumps just quite like her. It took me a couple of years to understand, that that is not
self-evident, oh well.

Lastly, I want to thank --- in no particular order --- Magi, Luzi, Mira, Eli, Anna, Phillip, Alex,
Alaya, Magda, Pil, Storm, Nana, Anabel, Luco, Virgina, Richy for everything. For all the
discussions, late night talks, walks, breakfast talks, coffee breaks. For all the emotional support,
for putting up with me in all my different moods, for providing distraction. But most of all, you
lots shaped me the most in the last years. Without you, I wouldn't be the person, I am today.

Without all of you, I would have not made it.

\begin{flushright}
    Thank you so so much \(\varheartsuit\)
\end{flushright}

%%%%%%%%%%%%%%%%%%%%%%%%%%%%%%%%%%%%%%%%%%%%%%%%%%%%%%%%%%%%%%%%%%%%%%%%
\chapter*{Abstract}
%%%%%%%%%%%%%%%%%%%%%%%%%%%%%%%%%%%%%%%%%%%%%%%%%%%%%%%%%%%%%%%%%%%%%%%%

\noindent%
Numerous applications in engineering can be mathematically described by so-called \textit{inverse
	problems}. All applications, where an initial quantity is recovered only from a set of observations,
can be described as such. One large application is \textit{tomographic reconstruction}. There, the
observed measurements are always a finite set of projections. Tomographic reconstruction covers
important imaging modalities such as X-ray attenuation computed tomography (CT). CT scanners based
on X-ray attenuation have been a major innovation in medical diagnostics. X-ray attenuation CT
provides a way to non-intrusively obtain images from the inside of objects. However, it is known
that X-ray attenuation based imaging modalities lack soft-tissue
contrast~\cite{pfeiffer_phase_2006}. Other imaging modalities can overcome this issue, one example
is differential phase-contrast CT\@.

No matter the exact imaging modalities, in order to find a solution to the inverse problem in the
field of tomographic reconstruction, the problem domain has to be discretized. A common approach is
the \textit{series expansion}. Here the original signal, is described by a linear combination of a
coefficient vector with a set of basis functions. Arguably, the most common choice of basis function
is the pixel (or voxel) basis function. Other alternatives have been proposed, such as
\textit{spherically-symmetric} (blob)~\cite{lewitt_multidimensional_1990} or \textit{B-Spline}
~\cite{unser_fast_1991} basis function. Especially, spherically-symmetric basis function have been
popular in tomographic reconstruction. Both basis function provide a higher accuracy compared to
voxel basis functions. But importantly, they are continuously differentiable (at least under certain
conditions). This is an important property, especially in the context of differential phase-contrast
CT\@.

Another common ground for tomographic reconstruction, apart from the reconstruction from
projections, is the mathematical description of the so-called \textit{forward model}. The forward
model is the description of the process of acquisition of the projections. For all mentioned method
throughout this thesis, this model is based on the line integral. I.e.\ the projections are acquired
from a (possibly infinite) set of infinitely thin lines tracing through the desired problem domain.

As with the problem domain itself, the line integral also has to be discretized. Special software,
often referred to as projectors, handles the computation of the line integral. Most commonly used
projection software inherently assume the voxel basis function as a discretization model of the
problem domain. However, this creates issues for imaging modalities that assume a differentiable
signal in the problem domain, such as differential phase-contrast CT\@.

In the scope of this thesis, two such projectors are implemented for the tomographic reconstruction
framework \textit{elsa}. One projector is based on spherically-symmetric and the other on B-Spline
basis functions. Two different projectors are implemented to firstly, lay the foundation for
first-class support of differential phase-contrast CT in elsa. But secondly, ensure this approach is
feasible.

Both projectors significantly outperform proven projector methods such as proposed by
\citeauthor*{siddon_fast_1985}~\cite{siddon_fast_1985} or
\citeauthor*{joseph_improved_1982}~\cite{joseph_improved_1982} in terms of accuracy. Typical
artifacts present in other kind of projectors are not visible. Both the forward projection and
reconstructions yield better results in multiple different cases. However, due the overlapping
nature of the basis functions, the computational burden is increased. This is especially evident for
the three-dimensional case.


\tableofcontents

\mainmatter
%%%%%%%%%%%%%%%%%%%%%%%%%%%%%%%%%%%%%%%%%%%%%%%%%%%%%%%%%%%%%%%%%%%%%%%%
% List of TODOs
%%%%%%%%%%%%%%%%%%%%%%%%%%%%%%%%%%%%%%%%%%%%%%%%%%%%%%%%%%%%%%%%%%%%%%%%
\todototoc
\listoftodos

%%%%%%%%%%%%%%%%%%%%%%%%%%%%%%%%%%%%%%%%%%%%%%%%%%%%%%%%%%%%%%%%%%%%%%%%
% Introduction
%%%%%%%%%%%%%%%%%%%%%%%%%%%%%%%%%%%%%%%%%%%%%%%%%%%%%%%%%%%%%%%%%%%%%%%%
\part[Introduction]{%
	Introduction\\
	%
	\vspace{1cm}
	%
	% \begin{minipage}[l]{\textwidth}
	% 	%
	% 	\textnormal{%
	% 		\normalsize
	% 		%
	% 		\begin{singlespace*}
	% 			\onehalfspacing
	% 			%
	% 			You can also use parts in order to partition your great work
	% 			into larger `chunks'. This involves some manual adjustments in
	% 			terms of the layout, though.
	% 		\end{singlespace*}
	% 	}
	% \end{minipage}
}\label{part:introduction}

\chapter{Introduction}\label{chap:introduction}

this should not be empty!!


\section{Motivation}\label{chap:Motivation}

\section{Scope}\label{chap:scope}

The scope of this work is two fold. First, a major part is the support of some kind of
differentiable basis functions in the \textit{elsa}~\cite{lasser_elsa_2019} framework. Only
recently, the team around Prof.\ Dr.\ Franz Pfizer from the Technical University of Munich developed
an advanced Dark-Field imaging to the human scale~\cite{viermetz_dark-field_2022}. This is an
exiting development, which also has an impact on team around my supervisor PD Dr.\ rer.\ nat.\
Tobias Lasser. As it stands today, elsa is not able to reconstruct either phase-contrast CT nor Dark-field
CT\@. This thesis does not implement full support for reconstruction of phase-contrast CT\@.
However, this thesis should lay the foundation for this support. In the scope of this thesis is a
projector, which handles different kinds of differentiable basis functions. As these kind basis
functions are the basis to support, phase-contrast CT\@.

Secondly, this thesis should be a guide for students trying to get into the field of tomographic
reconstruction. E.g.\ Students starting a project at the working group of PD Dr.\ rer.\ nat.\ Tobias
Lasser. This thesis tries to cover many areas, which are already or soon to be implemented in elsa.
And so, this should be of use to people trying to get into elsa, or parts of this will be worked
into the documentation of elsa.

I want the theoretical part to be of help for other students, but also for my future self as a
reference. Hence, I will try to investigate problems on different levels. One level is the
motivational and high level overview. This covers the larger picture, on what certain methods
provide benefits over other methods and how they relate. Plus certain similarities and differences.
Another level is an intuitive approach. Personally, throughout my studies, I found it immensely
useful to gain a intuition for a method. Together, with the motivational aspects and classification,
I could dig deeper into further details. Sadly, I will not be able to dig deep into all different
aspects. As best I can, I'll provide further reading material for those interested.

\section{Outline}\label{sec:outline}

For problems in tomographic reconstruction, there are a couple of important steps involved. These
include discretization of the problem, describing the model behind the reonconstruction and finally
actually solving the problem.

The first part of the thesis, \autoref{chap:imaging_modalities}, covers different imaging
modalities. It covers their use cases, a little bit of history, and a basic intuition on how they
work. Afterwards in \autoref{chap:radon_transform_and_related}, the mathematical abstraction for
these different physical models is laied down. Then \autoref{chap:image_representation} is all about
discretization of the problem domain. There different basis functions, that can be used to discrete
the image domain. With a specific focus on differentiable basis functions. In
\autoref{chap:tomographic_reconstruction} reconstruction techniques are presented. Its all about
recovering an image from projections. First, a couple of analytical solutions are presented, then a
deeper dive into iterative reconstruction algorithms is provided.

In the practical section, \autoref{chap:elsa} provides a light introduction into the C++ framework
\textit{elsa} developed at the \textit{Computational Imaging and Inverse Problems} research group at
the Technichal University of Munich. The next \autoref{chap:projector} provides details into
implementations of the physical forward and backward model. And a detailed discussion of
implementation choices for the projector developed as part of this thesis. In
\autoref{chap:experiments} a detailed evaluation of the projector is conducted.

The thesis ends with the conclusion in \autoref{chap:conclusion}.

\section{Disclaimer}\label{sec:disclaimer}

I want my research to be as open, reproducible and comprehensible as possible. Hence, the
development of the source code I've wrote is open source, discussions (though they were few) can be
found in the elsa GitLab repository \insertref{elsa GitLab repo}, the \LaTeX{} source code of thesis
can be found on my personal GitHub page \insertref{Thesis GitHub repo}. To the best of my
possibilities, I try to include the code I've used to generate all of my plots and graphs, I try to
state dependencies and make building and using my work as easy as I can. But of course, I am aware,
what is easy to me, might be cumberstone to the next, and impossible to the other.

\inlinetodo{disclaimer gender-neutral language}


%%%%%%%%%%%%%%%%%%%%%%%%%%%%%%%%%%%%%%%%%%%%%%%%%%%%%%%%%%%%%%%%%%%%%%%%
% Background
%%%%%%%%%%%%%%%%%%%%%%%%%%%%%%%%%%%%%%%%%%%%%%%%%%%%%%%%%%%%%%%%%%%%%%%%
\part[Foundation]{%
	Foundation\\
	%
	\vspace{1cm}
	%
	% \begin{minipage}[l]{\textwidth}
	% 	%
	% 	\textnormal{%
	% 		\normalsize
	% 		%
	% 		\begin{singlespace*}
	% 			\onehalfspacing%
	% 		\end{singlespace*}
	% 	}
	% \end{minipage}
}\label{part:foundation}

\chapter{Notation and Terminology}\label{chap:notation}

\inlinetodo{This section has to be done}
Let \(A\) be an operator, then \(A^\ast\) is the adjoint operation. class of linear bounded
operators from two vector spaces.
Hilbert space, Real numbers, matrix, vector, approx linear bounded operators, transpose

\chapter{Imaging Modalities}\label{chap:imaging_modalities}

Numerous different fields boil down to a very similar problem statement. Based on a given
measurement, how can one retrieve the original measured object, assuming the measurement process is
known. This is also often referred to as reconstruction form projections~\cite{herman_basis_2015}.
This is precisely, the definition of reconstruction used throughout this thesis.

These fields include, but are not limited to, variation methods in imaging (e.g.\ image denoising,
inpainting, super resolution and more~\cite{scherzer_variational_2009}), X-ray attenuation CT,
differential phase-contrast CT, X-ray dark-field contrast CT, light-field tomography, seismic
imaging, nondestructive material testing~\cite{carpio_inverse_2008}. All of these problems are part
of the class of so-called \textit{inverse problems}.

A little detour to motivate the importance of inverse problems: The Hubble Space Telescope was
launched into space in 1990. However, it initially had an issue with the spherical aberration in
its optics. A hardware fix was deployed in 1993. But due to the cost of the operation in general,
the images of the telescope were still used and a lot of processing went into the images. A lot of
different techniques were used to recover as much information as possible. This includes techniques
involving the solution of inverse problem similar as discussed in this thesis. The details are out
of scope, but the interested reader can look into~\cite{white_restoration_1992,adorf_hubble_1995}.

\begin{definition}[Inverse Problem]\label{def:inverse_problem}
	Loosely speaking, the solution to inverse problems is the cause of an effect. Turning this
	into the mathematical setting, let \(H\) and \(K\) be Hilbert spaces (i.e.\ a vector space,
	which has a scalar product and is complete), then let \(m \in K\) the measurements and \(f
	\in H\) the original cause. The system which causes the effect can be modeled by a
	linear bounded operator \(A \in \mathscr{L}(H, K)\). Then the problem is formulated as
	\[ A f = m \]
	The inverse problem is to retrieve \(f\) (the cause) given \(m\) (the measurements) and \(A\)
	(the physical model).
\end{definition}

As Hilbert spaces, are complete vector spaces with a scalar product \(\langle\cdot,\cdot\rangle\)
is defined. Hence, they make sufficiently nice spaces to work in. For the purposes of this thesis,
usually these spaces are subspaces of \(\mathbb{R}^n\).

I want to point out a minor, but important detail. Different imaging modalities, which vary
drastically in scope of physical properties, can be mathematically reduces to a common concept. This
enables reasoning on common ground. Improvements in one field, can benefit other fields. Further, a
common language can be used. This, I personally find very much fascinating and is one of my
motivations to keep learning.

Back to the inverse problems. Broadly speaking, problems in math can be categorized in two
categories. Problems are either \textit{well-posed} or they are \textit{ill-posed}. Following the
definition of Hadamard~\cite{hadamard_sur_1902} a problem is well posed if all the following
properties are fulfilled:

\begin{itemize}
	\item \textbf{Existence}: The exists a solution.
	\item \textbf{Uniqueness}: It is the only solution.
	\item \textbf{Stability}: The solution depends continuously on the data.
\end{itemize}

If any of these conditions doesn't hold it is ill-posed. An example for well-posed problems is the
heat equation with specified initial conditions. Inverse problems are basically all
ill-posed~\cite{hansen_discrete_2010}, at least the ones considered in this thesis.

For an ill-posed inverse problem, either the inverse \(A^{-1}\) does not exist, the solution does
not lie in \(K\), or is not continuous (c.f.~\cite[Chapter~4]{natterer_mathematics_1986}). From a
practical standpoint the first two properties can be solved quite easily. If no solution exists, one
can substitute it by a different problem, e.g.\ the looking for the solution to the least squares
problem. If multiple or infinitely many solutions exists, one can choose the one with minimal norm.
However, if the third property is violated, the solutions to a system with two close measurements
\(g\) and \(g^\epsilon\) need not be close.

From a practical point this is quite important. Imagine two blurred images of the same scene, that
only differ slightly from one another. If the stability criterion is violated, the solution to the
inverse problem trying to deblur the images, need not be close to each other. Hence, proper care
needs to be taken in the development of algorithms, that search for solutions for the inverse
problem. See \citeauthor{hansen_discrete_2010}\cite{hansen_discrete_2010} for some nice illustrative
examples for each property.

In the specific field of tomographic reconstruction, the challenge is to reconstruction an image
from a finite set of projections. Specifically, most of the applications considered are projections
from X-ray sources. However, other imaging modalities such as light field microscopy exists.

An essential aspect of tomographic reconstruction is the so called \textit{forward model}. It is the
mathematical abstraction of the physical measuring process (\(A\) in \autoref{def:inverse_problem}).
A mathematical approach and definition is given in \autoref{def:forward-model}. Throughout this
thesis, X-rays are treated as infinitely thin rays, and therefore, mathematically a single X-ray
going through the object can be model using a line. Along its path through the object interacts with
the matter and once it passed through it will hit some kind of detector . This can either measure
the attenuation of the X-ray, the refraction or the scattering, but the mathematical abstraction is
very similar.

This idea was already studied by \citeauthor{radon_uber_1917}. Hence, the resulting transformation
modeling the projection of an object to its projection is often referred to as the \textit{Radon
	Transform}. For two-dimensional reconstructions this holds true, but for higher dimensions,
this is not necessarily true anymore. Other transformations, such as the \textit{X-ray
	Transform}~\cite{solmon_x-ray_1976}, have been presented. An overview of both the Radon
Transformation and the X-ray Transformation is given in \autoref{chap:radon_transform_and_related}.

Again, note how the mathematical description closes the gap between different concepts of
measuring X-ray attenuation, refraction and scattering.

The remainder of this chapter will mainly introduce X-ray attenuation CT. This mostly involves the
derivation of the forward model, but it also includes the basic physical aspect necessary to
understand the derivation. Further, differential X-ray phase-contrast CT is introduced and the
common aspects of the forward model both modalities share are highlighted. This should also help to
gain an (visual) intuition for the mathematical introduction and definitions in the following
chapters.

Out of scope are physical details. Rather, they will be striped down and simplified to what is
necessary and useful for the scope of this thesis and the concepts presented. However, as much as
possible, resources for the interested reader are cited in the corresponding sections.

\section{X-ray Attenuation CT}\label{sec:xray_attenuation_ct}

The discovery of X-rays by Wilhelm Conrad Röntgen~\cite{rontgen_uber_1895} and its impact in medical
diagnostics have are tremendous. For the first time it was possible to 'look inside` an object
without opening it. As touched on in the introduction (\ref{chap:introduction}), this was used
to find and plan the removal of metal objects in bodies.

\citeauthor*{cormack_representation_1963}~\cite{cormack_representation_1963} laid the mathematical
foundations for X-ray attenuation CT. However, the breakthrough of X-ray attenuation CT scanners
only happened with the publication series of
\citeauthor{hounsfield_computerized_1973}\cite{hounsfield_computerized_1973},
\citeauthor{ambrose_computerized_1973}\cite{ambrose_computerized_1973} and
~\citeauthor{perry_computerized_1973}\cite{perry_computerized_1973}.

X-ray attenuation CT is an easy principle. The projections of the CT scanner are acquired using a
simple setup. The object one wants to scan is between the X-ray source and the detector. Without
going into details, X-ray sources for common CT scanners are vacuum tubes, which convert electrical
power to radiation. The detector consists of an array of pixels, which reacts sensitive to X-rays.
Then for each projection a `photo' is taken, the object is rotated in relation to the source and
detector, and then the next acquisitions is performed and so on. It is of no importance for the
reconstruction, if the object of interested is rotated or the source and detector are rotated. For a
detailed overview see \citeauthor{buzug_computed_2008}~\cite[Chapter~2]{buzug_computed_2008}.

The objectiv of tomographic reconstruction for X-ray attenuation, is the retrieval of the
attenuation coefficient for the complete object of interest. To achieve this, numouers projections
are necessary. Each projection is the X-ray shadow of the measured object.
\autoref{fig:sinogram_example_abdomen} shows three different projections for such an object. In this
specific case, the object of interest was rotated relative to the X-ray source and detector.
The graph on the right of each subimage, shows the measured projections values. The higher this
value is, the larger the attenuation (i.e.\ the darker the X-ray shadow) of the line hitting the
projection plane at this position.

\begin{figure}
	\centering
	\makebox[\textwidth]{ \makebox[1.3\textwidth]{%
			\begin{subfigure}{0.42\textwidth}
				\includegraphics[width=\textwidth]{./figures/sinogram_example/abdomen_sinogram_0.png}
				\label{fig:sinogram_example_0_degree}
			\end{subfigure}%
			\begin{subfigure}{0.42\textwidth}
				\includegraphics[width=\textwidth]{./figures/sinogram_example/abdomen_sinogram_45.png}
				\label{fig:sinogram_example_45_degree}
			\end{subfigure}%
			\begin{subfigure}{0.42\textwidth}
				\includegraphics[width=\textwidth]{./figures/sinogram_example/abdomen_sinogram_90.png}
				\label{fig:sinogram_example_90_degree}
			\end{subfigure}%
		}}%
	\caption{Three different projections of an object. In this case the X-ray source would be on
		the far left and generating parallel X-rays. From left to right: A projection from
		\(0^\circ\), \(45^\circ\) and \(90^\circ\). A single projection is the X-ray shadow
		of the desired object. It is also the sum of the attenuation coefficient of the
		integral along the infinite set of lines perpendicular to the projection plane (the
		line showing the attenuation values on the right)}\label{fig:sinogram_example_abdomen}
\end{figure}

For a simple study of the physical process in X-ray attenuation CT, one can start with the analysis
of a single X-ray going through an object with a homogeneous (i.e.\ single constant) attenuation
coefficient (see \autoref{fig:x-ray_homogeneous_attenuation}). The object one desires to
investigate, is between a X-ray source and a detector. The X-ray source generates X-rays with a
certain intensity \(I_0\). After the X-ray traverses the object, the detector measures a reduced
intensity denoted as \(I_1\). The connection between \(I_0\) and \(I_1\) is determined by the
attenuation coefficient \(\mu\) and the distance traveled through the matter \(s\). It is given by
\[ I_1 = I_0 e^{-\mu s} \]

\begin{figure}
	\centering
	\def\svgwidth{0.75\textwidth}
	\import{./figures/homogeneous_attenuation}{homogeneous_attenuation.pdf_tex}
	\caption{Simplified model of a single X-ray (red line) going through a material with
		homogeneous attenuation coefficient \(\mu\) (gray rectangle). \(I_0\) is the initial
		intensity of the X-ray, \(I_1\) is the measured intensity, given by attenuation
		coefficient \(\mu\) and the distance \(s\) the ray travels through the
		object.}\label{fig:x-ray_homogeneous_attenuation}
\end{figure}

Extending this model to accommodate changing or varying attenuation coefficients, one needs to
replace the constant coefficient \(\mu\), by a function \(\mu: \mathbb{R}^2 \mapsto \mathbb{R}\)
(for now, this is kept in the two-dimensional case). Then the measured intensity is given by the
integral along the line the ray travels along \(L\) of the function \(\mu\), see
\autoref{fig:x-ray_nonhomogeneous_attenuation} for an illustration.

\begin{figure}
	\centering
	\def\svgwidth{0.75\textwidth}
	\import{./figures/nonhomogeneous_attenuation}{nonhomogeneous_attenuation.pdf_tex}
	\caption{Model of a single X-ray (red line) going through a material with
		homogeneous attenuation coefficient \(\mu(x)\) (abdominal section). Again \(I_0\) is
		the initial intensity of the X-ray, \(I_1\) is the measured intensity, given
		by the integral along the line \(L\) (i.e.\ the ray given in red) of the
		attenuation coefficient function \(mu(x)\), where \(x\) is each point along the
		line \(L\).}\label{fig:x-ray_nonhomogeneous_attenuation}
\end{figure}

% Some help from https://www.fips.fi/slides/Bubba_SummerSchoolVFIP2019_1.pdf
Then the connection between the initial intensity, and measured intensity is connected by the
\textit{Beer-Lambert law}~\cite{buzug_computed_2008}:
\begin{equation}\label{eq:beer-Lambert-law}
	- \ln \frac{I}{I_0} = \int_L \mu (x) \, \mathrm{d}x
\end{equation}
The right-hand side of \autoref{eq:beer-Lambert-law} is the line integral of the attenuation
coefficient function \(\mu\) along the line \(L\). The line integrals for all lines which are
parallel to each other, are preceisly the above mentioned projections. This is what is examplenary
shown in \autoref{fig:sinogram_example_abdomen} for 3 different projection angles.

To successfully reconstruct an object from its projection one needs many projections from different
angles. The resulting measruments can be stacked together to an image. This is referred to as
\textit{sinogram}. The name originates from the sinoadial pattern a point draws in such an image,
see \autoref{fig:sinogram_simple_complete} for an example. There the sinogram for a single point in
the phantom is shown (together with the simple phantom) on the left. Further, the sinogram for the
abdomen used throughout this chapter can be seen in \autoref{fig:sinogram_complete}. The vertical
lines mark the projections taken in \autoref{fig:sinogram_example_abdomen}.

\begin{figure}
	\centering
	\makebox[\textwidth]{ \makebox[1.3\textwidth]{%
			\begin{subfigure}{0.325\textwidth}
				\includegraphics[width=\textwidth]{./figures/sinogram_example/simple_phatom.png}
				\caption{}\label{fig:sinogram_simple_phantom}
			\end{subfigure}
			\begin{subfigure}{0.325\textwidth}
				\includegraphics[width=\textwidth]{./figures/sinogram_example/simple_sinogram.png}
				\caption{}\label{fig:sinogram_simple_complete}
			\end{subfigure}
			\begin{subfigure}{0.325\textwidth}
				\includegraphics[width=\textwidth]{./figures/sinogram_example/abdomen_512.png}
				\caption{}\label{fig:sinogram_abdomen_phantom}
			\end{subfigure}
			\begin{subfigure}{0.325\textwidth}
				\includegraphics[width=\textwidth]{./figures/sinogram_example/abdomen_sinogram.png}
				\caption{}\label{fig:sinogram_abdomen_complete}
			\end{subfigure}
		}}
	\caption{Examples of sinograms. The name originates from the pattern a single points draw in
		the sinogram. This can be seen in
		\subref{fig:sinogram_simple_phantom}--\subref{fig:sinogram_simple_complete}. The
		sinogram for the abdomen data used throughout the chapter
		\subref{fig:sinogram_abdomen_phantom} can be in
		\subref{fig:sinogram_abdomen_complete}. The vertical lines mark the projections
		depicted in \autoref{fig:sinogram_example_abdomen}.
	}\label{fig:sinogram_complete}
\end{figure}

The first to proof that a function (in this case the \(\mu\)) can be described via its line
integral, was first shown by \textit{Radon Transform}~\cite{radon_uber_1917}
(see~\cite{radon_determination_1986} for the English translation). Further details on the Radon
Transform can be found in \autoref{chap:radon_transform_and_related}. However, for here it should be
noted, that the line integral of a function is the basis of many different imaging modalites.

Stepping back, to again a more practical approach. CT scanners evolves over time. Hence there do
exists many different, so-called, generations of CT scanners. The first, and mostly interesting ones
because a lot of the theory is build for those initially, are CT scanners, with a parallel beam
geometry (see \autoref{fig:parallel_beam_geometry}. There all rays are generated in parallel to each
other and hit the detector. This is achieved by having a X-ray source and the detector shift
perpendicular to the projection direction.

Another, common setting is the fan-bean geometry as shown in \autoref{fig:fan_beam_geometry}. It is
a two-dimensional setting, where the X-rays are emitted from a point-source and are not parallel to
each other, but rather (as the name suggest) have a fan like pattern. Further, the detector is often
curved. Other common geometry setups include inherently three-dimensional setup using a cone-line
beam shape and a two-dimensional detector. Newer CT scanners, also rely on multiple sources. A
detailed discussion is given in e.g.\ \cite{buzug_computed_2008}. Most of the geometric setups are
introduced to either reduce the acquisition time or increase the spatial resolution of the
reconstruction.

\begin{figure}
	\centering
	\makebox[\textwidth]{ \makebox[1.3\textwidth]{%
			\begin{subfigure}{0.6\textwidth}
				\incfigmaybe{./figures/parallel_beam_setup}
				\caption{Parallel Beam Setup}\label{fig:parallel_beam_geometry}
			\end{subfigure}%
			\begin{subfigure}{0.6\textwidth}
				\incfigmaybe{./figures/fan_beam_setup}
				\caption{Fan Beam Setup}\label{fig:fan_beam_geometry}
			\end{subfigure}%
		}}%
	\caption{Geometry setup of CT scanners}\label{fig:ct_geometry_setup}
\end{figure}

\section{Phase-Contrast CT}\label{sec:phasecontrast_ct}

In the X-ray attenuation CT setup, X-rays are only considered exhibiting attenuation. However, as
visible light, X-rays are also refracted and scattered. Methods based on refraction can measure the
phase-shift of the X-ray. These phase-contrast imaging modalities in general provide a general
advantage in soft-tissue contrast, as the difference in refraction index is higher for soft-tissues
than for the attenuation coefficient. This chapter aims to introduce basic concepts on the retrieval
of phase-contrast based imaging modules and derive the forward model used in such imaging
modalities. Specifically, emphasise is put on the similarities in the forward model regarding X-ray
attenuation CT\@.

There are different ways to measure the refractions and thus the phase-contrast. A setup, which
gained a lot of traction lately is the setup proposed by
~\cite{pfeiffer_phase_2006,pfeiffer_hard-x-ray_2008}. The setup is based on Talbot-Lau
interferometry. It allows both the retrieval of the X-ray refraction and scattering information
(which can be used for dark-field imaging). The setup consists as with conventional X-ray
attenuation imaging of the X-ray source and the detector. Additionally, 3 gratings are placed in the
setup: the \textit{source grating} G0, the \textit{phase-grating} G1 and the \textit{analyzer
	grating} G2. G1 is placed between the X-ray source and the object, the other two are placed
between the object and the detector. The source grating creates sufficiently high coherence, which
allows for a periodic interference behind the phase grating. Using the analyzer grating allows for
conventional X-ray detectors to be used. For a detailed discussion about the exact placement of the
gratings, see~\cite{donath_inverse_2009}.

A key advantage of this grating based setup is the usage of conventional X-ray sources. This makes
this setup suitable for (bio-) medical applications. Only recently this setup was incorporated in a
medical grate CT scanner~\cite{viermetz_dark-field_2022}.

The important observation in the grating-based setup, an object placed in the X-ray beam causes a
refraction of the beam by an angle \(\alpha\) (following~\cite{donath_inverse_2009}) in the
\(x\)-direction. The refraction angle is connected to the differential phase-shift \(\frac{\partial
	\Phi}{\partial x}\) by:
\[ \alpha = \frac{\lambda}{2 \pi} \frac{\partial \Phi}{\partial x} \]
\(\lambda\) is the X-ray wavelength, and the beam propagates along the \(z\)-axis.
Simplified, placing an object in the beam, causes a refraction, which results in a displacement of
the interference pattern created by the gratings. And by the above formula is connect to the
phase-shift introduced by the object. However, the interference patterns can not be spatially
resolved by the detector. However, performing multiple measurements with a later shifted analyzer
grating, creates a intensity modulation of the detector read-out. This is referred to as
\textit{phase stepping}. Apart from the additional stepping necessary, the process of acquiring
measurements is similar to the one in conventional X-ray attenuation CT\@. The different possible
values measured using phase-stepping are illustrated in \autoref{fig:grating_setup_what_happens}.
In this setup attenuation is reflected by drop of the average intensity. Compared to that,
phase-shifts, leads to a lateral displacement of the interferometer pattern, and scattering reduces
the amplitude of the oscillation.

\begin{figure}
	\centering
	\makebox[\textwidth]{ \makebox[1.3\textwidth]{%
			\begin{subfigure}{0.42\textwidth}
				\includegraphics[width=\textwidth]{./figures/phase_contrast/plot_attenuation.png}
			\end{subfigure}%
			\begin{subfigure}{0.42\textwidth}
				\includegraphics[width=\textwidth]{./figures/phase_contrast/plot_phase.png}
			\end{subfigure}
			\begin{subfigure}{0.42\textwidth}
				\includegraphics[width=\textwidth]{./figures/phase_contrast/plot_scattered.png}
			\end{subfigure}
		}}
	\caption{Visualization of the measured intensity changing measured at the detector, when
		performing phase-stepping. Attenuation is measured by a drop of the average
		intensity value (left most plot). Phase-shift leads to a lateral shift of the
		intensity signal (center plot). And scattering drops in amplitude (left plot)}%
	\label{fig:grating_setup_what_happens}
\end{figure}

Now that the setup is described, the next step is the derivation of the forward model. The intensity
signal \(I(x, y)\) measured in each pixel \((x, y)\) oscillates as a function of the stepping
direction. The phases \(\varphi(x, y)\) of the intensity oscillations in each pixel are connected to
the refraction angle \(\alpha(x, y)\), the distance \(d\) between G1 and G2 and the period \(p_2\)
of the analyzer grating G2 by~\cite{weitkamp_x-ray_2005}
\[ \varphi = 2 \pi \alpha \frac{d}{p_2} = \frac{\lambda d}{p_2} \frac{\partial \Phi}{\partial x} \]

Using the phenomenological notion of a complex index of refraction, a combined description of
refraction and attenuation can be given. It is defined as
\[ n = 1 - \delta + i \beta\]
The real part describes scattering and refraction, and the imaginary part describes attenuation.
A wave propagating with a wave vector \(\mvec{k}\) thoruhg a medium with refractive index \(n\), can
be described as
\[ \Psi(\mvec{r}) = \Psi_0 \cdot e^{i n \mvec{k} \cdot \mvec{r}} = \Psi_0 \cdot
	e^{i(1-\delta)\mvec{k} \cdot \mvec{r}} \cdot e^{-\beta \mvec{k} \cdot \mvec{r}}\]
Here the first exponential on the right hand side represents a phase factor and the second an
attenuation of the wave's amplitute. This model can be used to derive the Beer-Lambert law as it was
shown in \autoref{eq:beer-Lambert-law} (compare~\cite[Chapter~2.1]{hahn_statistical_2014}). If one
looks only at the phase related term, it can be reformualted to
\[ \Psi_P(\mvec{r}) \coloneq \Psi_0 \cdot e^{i \mvec{k} \cdot \mvec{r}} \cdot e^{-i\delta \mvec{k}
			\cdot \mvec{r}} \]
Using the fact, that a wave propagating through vacuum is given by
\[ \Psi_v \coloneq \Psi_0 \cdot e^{i \mvec{k} \cdot \mvec{r}} \]
the phase-shift term simplifies to
\[ \Psi_P(\mvec{r}) = \Psi_v \cdot e^{-i\delta \mvec{k} \cdot \mvec{r}} \]
next similar transformations as with the X-ray attenuation based model can be done. And then the
phase-shift for a wave travelin through a homogeneous material with a single refraction index \(n\)
is given by
\[ \Phi = \delta \mvec{k} \cdot \mvec{r} \]
As with X-ray attenation, this can be extended to inhomogeneous media, by integrating the refractive
index decrement function \(\delta\)
\[ \Phi = \int \delta(x, y, z) k_y \mathrm{d}y \]
Then the refraction angle \(\alpha\) is given by
\[ \alpha(x,y) = \frac{\partial}{\partial x} \delta(x, y, z) \mathrm{d}y \]
This section is mostly based on the derivation given in~\cite{hahn_statistical_2014}.
It should be of note, that it is not possible to directly measure the phase-shift with current
detector technology, but it is sufficient to measure the refraction index. Albeit, it should be
notated, that the angles are very small.

With that out of the way, we can see, that the forward model for differential phase-contrast is also
connected to the line integral. In this case the function integrated is not the attenuation
coefficient \(\mu\), but the refractive index decrement \(\delta\).

As a small excursion. As already eluded to a bit. The grating based setup can in addition to
attenuation and phase also measure scattering of the X-rays. Though, the scattering angles are
small, it is possible. Imaging modalities based on the scattering component are referred to as
\textit{dark-field}. The setup described above only measures scattering perpendicular to the
gratings. Hence, it is also possible to rotate the object around the central beam axis, i.e.\ in the
plane of the gratings. This opens the door for so call directional or \gls{AXDT}. The connection to
the previous modalities (apart from the similar setup), is also the connection of the line integral.
Thou more complicated, they still share this common aspect.

\begin{figure}
	\centering
	\makebox[\textwidth]{ \makebox[1.3\textwidth]{%
			\begin{subfigure}{0.60\textwidth}
				\includegraphics[width=\textwidth]{./figures/tooth/tooth_attenuation.png}
			\end{subfigure}%
			\begin{subfigure}{0.60\textwidth}
				\includegraphics[width=\textwidth]{./figures/tooth/tooth_phasecontrast.png}
			\end{subfigure}
		}}
	\caption{Example reconstruction of a medical tooth sample that was. On the left, the
		reconstruction of the attenuation part of the measured signal, on the right the
		reconstruction of the differential phase-contrast data. The absorption has been
		windowed to values in the range of \([0, 0.33]\), and for the phase-contrast image,
		an interval of \([-\frac{\pi}{8}, \frac{\pi}{8}]\) was used. Taken
		from~\cite{wieczorek_anisotropic_2017}. With permission from Tobias Lasser}
	\label{fig:medical_tooth_sample}
\end{figure}

As a final part of this chapter, \autoref{fig:medical_tooth_sample} presents the reconstruction of a
medical tooth sample, which was measured with a system as described above. The image of the left
shows the reconstruction of the attenuation based signal and on the right, the image based on the
differential phase-contrast is shown. This should mainly highlight, the difference in details gained
by the different methods.

\chapter{Radon Transform and its related transform}\label{chap:radon_transform_and_related}

In the previous chapter, the notion of the inverse problem is introduced. Further a forward model
for both X-ray attenuation and phase-contrast CT is derived. There the derivation was mostly driven
by the physical properties of the X-rays. This chapter dives deeper in the mathematical formulations
of forward models. And specifically important transformations in the field of tomographic
reconstruction.

The physical model is the link between the unknown signal and it's measurements. Usually, the model
is referred to as forward model. First a couple of basic definitions are given, followed by a deeper
dive in to the mathematical formulation of the forward model for X-ray attenuation CT\@.

\begin{definition}[Signal]\label{def:signal}
	Let \(f\colon \mathbb{R}^n \to \mathbb{R}\) be a \(n\)-dimensional continuous function,
	whose support is bounded. It is referred to it as a \(n\)-dimensional signal. And often it
	will be referred to as signal, without the special mention of \(n\)-dimensional.
\end{definition}

Other names for signal are common, such as \textit{image}. However, image is often an overloaded
term and can be misunderstood. Hence, it does not suit to a general notion. However, in specific
cases, such as the reconstruction of a two-dimensional function, it works well and thus, I will
refer to images as two-dimensional signal. But limited to the solution domain of the inverse
problem. The word image, in my opinion, also does not suit for example the sinogram, as it is not in
the sense an image, but rather a stacked form of many measured signals. The same is true for
volumes. In the three-dimensional setting, this is exactly the kind of signal on wishes to
reconstruct.

A note for completeness sake. cecall \autoref{def:inverse_problem} of the inverse problem. For our
purposes \(H = \mathbb{R}^I\) and \(K = \mathbb{R}^J\) with \(I \in \mathbb{N}\) and \(J \in
\mathbb{N}\) two numbers. Now, the forward model can be defined.

\begin{definition}[Forward Model]\label{def:forward-model}
	To reconstruct an unknown \(I\)-dimensional signal \(f: \mathbb{R}^I \to \mathbb{R}\), a set
	of \(J\) scalar measurements is necessary. A single scalar valued measurement \(m_j \in
	\mathbb{R}\), with \(j \in \{1, \dots, J\}\) is defined in therms of the physical model:
	\[ m_j = \mathscr{M}_j(f)\]
	where
	\[ \mathscr{M}_j\colon (\Omega \to \mathbb{R}) \to \mathbb{R} \]
	and \(\Omega \subseteq \mathbb{R}^I\). The mapping is required to be linear.
\end{definition}

This is a very general definition. This is useful to mathematically model a wide variety of
different applications. Usually, some restrictions are put upon the function \(f\), such that is has
compact support, i.e.\ is \(0\) outside a certain region, and that it is sufficiently nice. All the
applications in the previous chapter, can be modeled using this general definition.

The next sections are devoted to mathematical examples of such forward models. As the maybe most
important or at least most famous model, the \textit{Radon Transform} is discusses in detail.
However, for X-ray imaging modalites, the Radon Transform is only a good model for the
two-dimensional case. Therefore, the special \textit{X-ray Transform} is also presented.

\section{Radon Transform}\label{sec:radon_transform}

Recall from \autoref{sec:xray_attenuation_ct}, the connection between the initial intensity and the
measured intensity was the line integral of the attenuation coefficients of the material. This
transformation of a function is known as the Radon Transform, which is attributed to Johann Radon
and his work in \citeyear{radon_uber_1917}~\cite{radon_uber_1917,radon_determination_1986}. It was
later rediscovered by \citeauthor{cormack_representation_1963}~\cite{cormack_representation_1963} in
the context of the invention of CT scanners. Without any further ado, let us directly dive into the
definition of the Radon Transform.

\begin{definition}[Radon Transform]\label{def:radon_transform}
	Let \(\Omega \subset \mathbb{R}^n\) and \(f\colon \Omega \to \mathbb{R}\), which is assumed
	to sufficiently nice. Then the mapping \(\radon{R}f\colon (\mathbb{R}^n \to \mathbb{R})
	\to (\mathbb{R} \times \mathscr{S}^{n-1} \to \mathbb{R})\) of \(f\), which maps \(f\) into
	the set of its integrals over the affine hyperplanes of \(\mathbb{R}^{n-1}\), is called the
	\textit{Radon Transform} (c.f.~\cite{natterer_mathematics_1986,buzug_computed_2008,carpio_inverse_2008}).
\end{definition}

Specifically, given the unit direction \(\theta \in \mathcal{S}^{n-1}\) and a scalar \(s \in
\mathbb{s}\), one can define the hyperplane \(\mathcal{H}^{n-1}(\theta, s) = \lbrace x \in
\mathbb{R}^n \, \colon \langle x, \theta \rangle = s  \rbrace\) with distance \(s\) to the origin
and perpendicular to \(\theta\). The Radon Transform \(\radon\) of \(f\) is defined by the line
integral over the hyperplane \(\mathcal{H}^{n-1}(\theta, s)\):
\[ \radon f(\theta, s) = \radon_\theta f(s) = \int_{\langle x, \theta \rangle = s} f(x) \, \, \mathrm{d}x \]

Note that for \(n=2\) the hyperplanes are lines, and hence match the forward model for X-ray
imaging. For the 3-dimensional case, this does not fit anymore. For this case, the so called X-ray
Transform \(\xray\) was developed~\cite{solmon_x-ray_1976}, which is very similar to the Radon
Transform, but integrates over lines for all dimensions, instead of hyperplanes. See
\autoref{sec:xray_transform} for more an the X-ray Transform.

Also note that many different notations are common. Specifically in two dimensions, it is common to
define it via the polar coordinates. Others define a subspace perpendicular to the projection
direction \(\theta^\perp\). But the basic principal is the same.

The Radon Transform is related to the Fourier Transformation. Also it is a special case of the
Hankel Transform. Another special case of the Radon Transform is the \textit{Abel Transform}. If the
function \(f\) is a spherically symmetric function the Radon Transform coincides with the Abel
Transform~\cite{buzug_computed_2008}.

Further, if the Radon Transform is defined as a bounded operator. One can define its adjoint
\(\radon^\ast\) as
\[ \radon^\ast g (x) = \int_{\mathcal{S}^{n-1}} g(\theta, \langle x, \theta \rangle) \, \, \mathrm{d} \theta \]
often \(g = \radon f\). The adjoint is often referred to as the \textit{back projection}. Compared
to the \textit{forward projection} which is \(\radon f\).

In the setting of tomographic reconstruction, one wishes to reconstruct the original signal \(f\)
from a set of measurements \(m\). Already, \citeauthor*{radon_uber_1917} showed a theoretical way of
the adjoint operations of the Radon Transform. However, these results are very theoretical. A more
practical approach uses the connection of the Radon Transform to the Fourier Transform.

\begin{definition}[Fourier Slice Theorem]\label{def:fourier_slice_theorem}
	Let \(f\colon \mathbb{R}^2 \to \mathbb{R}\) be sufficiently nice and \(\mathscr{F}_n\) the
	\(n\)-dimensional Fourier transform. Then
	\[ (\mathscr{F}_2f)(\theta, s) = (\mathscr{F}_1(\radon_\theta f(\cdot)))(s) \]
	It is also often referred to as \textit{projection-slice theorem} or \textit{central slice theorem}
\end{definition}

This is a very powerful theorem. In a more natural language, the \(1\)-dimensional Fourier Transform
of the projected data measured at the angle \(\theta\), is the same as a line through going the
origin of the \(2\)-dimensional Fourier Transform of the complete signal. The line is the line going
through the origin with a rotation angle \(\theta\). Hence, with enough projections, the \(f\) can be
described fully in the Fourier domain and reconstructed using the inverse \(2\)d Fourier Transform.

However, this method has a couple of drawback. \textit{Enough} projections is quite a vague
statement and a strong limitation. Especially considering the trend to reduce X-ray dosage and hence
reducing the number of projections acquired. Further, problems arise as the Fourier domain is
typically sampled in polar coordinates, but for this representation, one would like to access them
using Cartesian coordinates. This requires some for of interpolation. See the dissertation of
\citeauthor{vogel_tomographic_2015}\cite[Chapter~4.1.2]{vogel_tomographic_2015} for a more detailed
discussions and very nice illustrative figures. More methods to compute the adjoint and
reconstruction \(f\) are presented in \autoref{chap:tomographic_reconstruction}.

Of further interest to us is the first derivative of the Radon transform. As a variety of imaging
methods build on top of the derivative.

\begin{definition}[Derivative of the Radon Transform]
	The \(n\)th derivative of the Radon Transform is denoted by (compare
	e.g.\ \cite{nilchian_differential_2012,nilchian_fast_2013})
	\[ \radon^{(n)} = \frac{\partial^n}{\partial s^n} \radon f(\theta, s)\]
\end{definition}

The derivatives are linear operators, which are scale invariant, pseudo-distributive with respect to
convolution and projected translational invariant. The adjoint of the first derivative of the Radon
Transform is shown in~\cite{nilchian_differential_2012}, and for the \(n\)th derivative
see~\cite{nilchian_fast_2013}.

\section{X-ray Transform}\label{sec:xray_transform}

The \(n\)-dimensional Radon Transform computed the integral over \(n-1\)-dimensional hyperplanes.
However, for the setting of attenuation X-ray CT, one is interested in the line integration over
lines in \(n\)-dimensional spaces.

\begin{definition}[X-ray Transform]\label{def:x-ray_transform}
	Given \(\theta \in \mathscr{S}^{n-1}\) and \(x \in \mathbb{R}^n\), then
	\[ \xray f(\theta, x) = \xray_\theta(x) = \int_{-\infty}^{+\infty} f(x + t \theta) \, \mathrm{d}t\]
	is the X-ray Transform. It is the integral over the straight line through \(x\) with
	direction \(\theta\) (c.f.~\cite{natterer_mathematics_1986,solmon_x-ray_1976}).
\end{definition}

In the two-dimensional case, the Radon Transform and X-ray Transform are equivalent.
Then, the relation between the Radon Transform and the X-ray Transform is
\[\xray_\theta f(s\theta^\perp) = \radon_{\theta^\perp} f(s)\]
For the case of attenuation X-ray CT, the X-ray Transform is the physical forward model used. The
Fourier Slice Theorem \ref{def:fourier_slice_theorem} also holds true for the X-ray Transform.

Another transformations, which is not considered in this thesis is the Cone-Beam Transform
(c.f.~\cite[Chapter~2]{carpio_inverse_2008}). Also other important topics such as the Riesz
potential are left out.

\chapter{Image Representation}\label{chap:e_representation}

Images as defined in \autoref{def:signal}, are continuous functions. However, if one wishes to use
computers to solve the reconstruction tasks, there exists a need for discretization, as computers
are inherently discrete. Hence, one wishes to represent an signal in a discrete fashion.

\begin{definition}[Permissible representation]
	\label{def:permissible_representation}
	Let \(f\colon \mathbb{R}^n \to \mathbb{R}\) be a \(n\)-dimensional continuous function,
	\(N \in \mathbb{N}\) be a positive integer and \(\varphi_n\) a set basis function for
	\(1 \leq n \leq N\). Then the signal \(f\) can be approximated as a linear combinations
	of these basis functions and the coefficients \(c_n\):
	\[ f \approx \near{f}(\mvec{x}) = \sum_{k=1}^{N} c_k \varphi_k(\mvec{x}) \]
\end{definition}

\inlinetodo{Have figure of grid above image to represent the discretization}

For our purposes, we assume the function lies on a regular spaced discrete grid. Then, let
\(\varphi\) be a zero centered symmetrical basis function, \(\mvec{k} \in \mathbb{Z}^n\) be the
\(n\)-dimensional index of a grid cell, and \(\vec{x}_{\mvec{k}} \in \mathbb{R}^n\) the center
coordinate of the \(\mvec{k}\)-th grid cell. Then, the previous equation can be reformulated:
\[ \near{f}(\mvec{x}) = \sum_{\mvec{k} \in \mathbb{Z}^n} c_{\mvec{k}} \varphi(x - x_{\mvec{k}}) \]
This definition follows the notation given in~\cite{momey_new_2011}. This method to discretize an
signal is called \textit{series expansion} and is described in detail in
e.g.\ \cite{herman_basis_2015}.

Now, if one applies the Radon transformation to the discretized signal: \todo{generalize to all linear physical models}
\[ \radon\near{f}(\mvec{x}) = \radon\left( \sum_{\mvec{k} \in \mathbb{Z}^n} c_{\mvec{k}} \varphi(\mvec{x} - \mvec{x}_{\mvec{k}}) \right) \]
Due to the linearity of the Radon Transform this is equivalent to
\[ \radon\hat{f}(x) = \sum_{\symbfit{k} \in \mathbb{Z}^n} c_{\symbfit{k}}\radon\left( \varphi(x - x_{\symbfit{k}}) \right) \]
i.e.\ the Radon transformation of the signal, only act upon the basis function. Hence, it is
sufficient to study, how the Radon transformation acts upon the individual basis function.

Note that this holds for any linear physical model. Notably, this holds for the X-ray transform and
the first derivative of the Radon transform. But it is also true for the physical models behind the
applications discussed in the previous chapter.
\[ \mathscr{M}_j\near{f}(x) = \sum_{\mvec{k} \in \mathbb{Z}^n} c_{\mvec{k}}\mathscr{M}_j\left( \varphi(\mvec{x} - \mvec{x}_{\mvec{k}}) \right) \]

The key takeaway, is that the forward models related to the imaging modalities discussed in
\autoref{chap:imaging_modalities} only act on the basis function. Hence, one needs to only study how
the basis function behave under the given transform.

\section{Voxel Basis}\label{sec:voxel_basis}

The most likely most well known basis function in imaging is the pixel or voxel basis functions. The
voxel basis function is a piecewise linear function. The voxel basis function is most likely the
most widely used basis function. Most literature assumes the voxel basis function implicitly.

The centered voxel basis function of step width \(h\), is given by:
\begin{equation}\label{eq:voxel_basis_fn}
	\varphi^{\text{pixel}}(\mvec{x}) =
	\begin{cases}
		1, \abs{\mvec{x}} < \frac{h}{2} \\
		0, \text{otherwise}
	\end{cases}
\end{equation}
Here, the absolute value is coefficient wise, as soon as the absolute value of any coefficient of
the vector \(\mvec{x} \in \mathbb{R}^n\) is larger half the step size, the function will return
\(0\).

\inlinetodo{Add figure showing the centered basis function, show how a continuous function can
	approximated by pixels}

An signal approximated by the voxel basis function, in the series expansion method is equivalent to
the nearest neighbourhood interpolation.

As pointed out by \citeauthor*{lewitt_multidimensional_1990}
in~\cite{lewitt_multidimensional_1990,lewitt_alternatives_1992} the voxel-basis function isn't
necessarily a good choice for biomedical imaging. Further, it is discontinuous at the boundaries.

The analytical formulation of the Radon transform of the pixel basis function
(compare~\cite{toft_radon_1996}) is given by:
\begin{equation}\label{eq:radon_voxel_basis}
	\radon\varphi^{\text{pixel}}(\rho, \theta) =
	\begin{cases}
		0                                                  & x_1 > 0                         \\
		\sqrt{4 + (x_1 - x_{-1})^2} = \frac{2}{\cos\theta} & x_1 < 1\;\text{and}\;x_{-1} < 1 \\
		\sqrt{(1 - x_1)^2 + (1 - x_{-1})^2}                & x_1 < 1\;\text{and}\;x_{-1} > 1
	\end{cases}
\end{equation}

As the voxel basis function is discontinuous at the boundaries, there does not exist a way to
compute the derivative of the radon transform relying on the basis function. Steps such as numerical
derivation must be used.

\todo{understand this properly and explain this properly}

\section{Blob Basis}\label{sec:blob_basis}

First introduced by Lewitt in~\cite{lewitt_multidimensional_1990}, spherically symmetric volume
elements (often referred to as blobs) are an alternative to the pixel basis.
~\cite{lewitt_alternatives_1992} describes how blobs can be used in iterative reconstruction
algorithms as a basis instead of pixels.

Blob basis functions have been adopted in multiple different fields. Among others electron
microscopy~\cite{marabini_3d_1998, garduno_optimization_2001}, positron emission tomography
(PET)~\cite{jacobs_comparative_1999, chlewicki_noise_2004}, single-photon emission tomography
(SPECT)~\cite{wang_3d_2004, yendiki_comparison_2004}, attenuation X-ray
CT~\cite{jacobs_iterative_1999, carvalho_helical_2003, isola_motion-compensated_2008},
phase-contrast CT~\cite{kohler_iterative_2011, xu_investigation_2012}, reconstruction of coronary
trees~\cite{zhou_blob-based_2008}, breast tomosynthesis~\cite{wu_breast_2010}, reduction of metal
artifacts~\cite{levakhina_two-step_2010} or computed laminography~\cite{trampert_spherically_2017}.

\inlinetodo{Read papers and assert what blobs brings to the table}

Generally, many fields report increased accuracy with a comparable performance. In other fields,
such as phase-contrast CT, blobs enable the usage of iterative reconstructions without an extra step
of numerical differentiation.

The generalized Kaiser-Bessel basis function as proposed by Lewitt, is defined as:
\begin{equation}\label{eq:blob_basis_fn}
	\varphi^{\text{blob}}_{m, \alpha, a}(r) =
	\begin{cases}
		\frac{I_m\left( \alpha \sqrt{1 - \left(\frac{r}{a}\right)^2} \right)} {I_m\left( \alpha \right)} \left( \sqrt{1 - \left(\frac{r}{a}\right)^2}\right)^m & 0 \le r \le a      \\
		0                                                                                                                                                      & \textit{otherwise}
	\end{cases}
\end{equation}
where \(I_m\) is the modified Kaiser-Bessel function of the first kind of order \(m\), \(r\) the
distance to the blob center, \(a\) the blob radius given in units of the grid, and \(\alpha\)
controlling the shape of the blob. \(m\) controls the continuity of the blob function.
\inlinetodo{Figures showing 2d blob and different parameters of blob}

The Radon Transform \todo{Or the X-ray transform?} of the blob basis function simplifies to the Abel
Transform~\cite{buzug_computed_2008}. This is possible as the blob basis function is rotationally
symmetric. Therefore, the Radon Transform of a blob basis function is by
(c.f.~\cite{lewitt_multidimensional_1990,lewitt_alternatives_1992})
\begin{align}\label{eq:radon_blob_basis}
	p(s) & = \int_{-\infty}^{+\infty} \varphi^{\text{blob}}_{m, \alpha, a}\left(t\right) \, \mathrm{d} t                                                                                                       \\
	     & = 2 \int_0^{\sqrt{(a^2-s^2)}} \varphi^{\text{blob}}_{m, \alpha, a}\left(\sqrt{s^2 - r^2}\right) \, \mathrm{d} r                                                                                     \\
	     & = \frac{a}{I_m(\alpha)} \left( \frac{2\pi}{\alpha}\right)^{1/2} \left( \sqrt{1 - \left(\frac{s}{a}\right)^2} \right)^{m + 1/2} I_{m+1/2}\left( \alpha \sqrt{1 - \left(\frac{s}{a}\right)^2} \right)
\end{align}
\(s\) is the distance from the X-ray to the blob center, and \(\sqrt{a^2 - s^2}\) is one half of the
intersection length between the blob and the ray. The projected value only depends on the distance
from the X-ray to the blob center. This is a very nice property. This makes implementations quite
efficient.

\inlinetodo{figure for parameters for projected basis, figure for parameters for basis,
	\cite{benkarroum_blob_2015}}
\inlinetodo{Derivative of Blob basis function}

From an implementation standpoint, the half integer order of the modified Kaiser-Bessel function
of the first kind, can be quite nasty. Implementations do exist as it can be seen
in~\cite{temme_numerical_1975}. However, the floating point implementations are non-trivial. Plus,
for the case of C++, since C++17 the standard library provides mathematical special functions
~\cite{noauthor_c_nodate, noauthor_stdcyl_bessel_i_nodate}. But sadly, it is not yet entirely
cross-platform, as it is only supported by libstdc++~\cite{noauthor_libstdc_nodate-1}, and not
libc++. However, for our cases it is sufficient to assume \(m \in \mathbb{N}\). Then the above
equation can be further simplified.

The recurrence formulation for the modified Kaiser-Bessel function of the first kind is
(c.f.~\cite[Chapter~9]{abramowitz_handbook_1972}):
\begin{equation}\label{eq:kaiser_bessel_recurrence}
	I_{m+1}(x) = I_{m-1}(x) - \frac{2 m}{x}I_m(x)
\end{equation}
Further, the Kaiser-Bessel functions have representations with elementary functions. For the
modified Kaiser-Bessel function of the first kind, there are defined as (c.f.~\cite[Chapter~10]{abramowitz_handbook_1972}):
\begin{align}\label{eq:kaiser_bessel_half_integer}
	I_{0.5}(x) & = \sqrt{\frac{2}{\pi x}} \sinh(x)                                                                               \\
	I_{1.5}(x) & = \sqrt{\frac{2}{\pi x}} \left( \cosh(x) \frac{\sinh(x)}{x} \right)                                             \\
	I_{2.5}(x) & = \sqrt{\frac{2}{\pi x}} \left(\left(\frac{3}{x^2} + \frac{1}{x}\right)\sinh(x) - \frac{3}{x^2} \cosh(x)\right)
\end{align}
Then \autoref{eq:radon_blob_basis} can be simplified to not include any non-integer evaluations of
the modified Kaiser-Bessel function of the first kind. For example assuming, \(m = 0\), and to keep
everything a little more concise, let \(w = \sqrt{1 - \left(\frac{r}{a}\right)^2}\):
\begin{align}\label{eq:radon_blob_basis_order_0_simplified}
	p(s) & = \frac{a}{I_0(\alpha)} \left(\frac{2\pi}{\alpha}\right)^{1/2} \left( w \right)^{1/2} I_{1/2}\left( \alpha w \right)                     \\
	     & = \frac{a}{I_0(\alpha)} \left(\frac{2\pi w}{\alpha}\right)^{1/2} I_{1/2}\left( \alpha w \right)                                          \\
	     & = \frac{a}{I_0(\alpha)} \left(\frac{2\pi w}{\alpha}\right)^{1/2} \left( \frac{2}{\pi \alpha w}\right)^{1/2} \sinh \left(\alpha w \right) \\
	     & = \frac{2 a}{\alpha I_0(\alpha)} \sinh \left(\alpha w \right)
\end{align}
In the last step, \(\pi\) and \(w\) cancel out, and both the \(2^2\) and \(\alpha^2\) are moved out
of the square root, leaving it empty. Similar operations can be done for \(m = 1\) and \(m = 2\).

\section{B-Spline Basis}\label{sec:bspline_basis}

Splines are common in image and signal processing~\cite{unser_splines_1999}. Applications include
image interpolation, image transformations, image compressions or the calculation of the first and
second derivative. A common approach is the approximation of the function or image using Splines and
then working efficiently on the continuous representation of the splines. For B-Splines
specifically,~\citeauthor{unser_fast_1991}~\cite{unser_fast_1991} show the continuous image
representation using B-Splines.

This approach was adopted to tomographic reconstruction.~\cite{la_riviere_spline-based_1998}
proposed the calculation of the inverse 2D and 3D Radon transform based on B-Spline.
Similarly,~\cite{horbelt_discretization_2002} develops a B-Spline based filtered back projection.
Apart from attenuation CT, other medical applications of B-Splines include electron
tomography~\cite{tran_robust_2013, tran_inverse_2014}, positron emission tomography
(PET)~\cite{nichols_spatiotemporal_2002, li_fast_2007, verhaeghe_investigation_2007} and
single-photon emission tomography (SPECT)~\cite{guedon_b-spline_1991, reutter_fully_2007}.

Using B-Splines as a basis function was first presented by~\cite{momey_new_2011,
	momey_b-spline_2012, momey_spline_2015}. And a similar signal representation was adapted for
phase-contrast CT in~\cite{nilchian_fast_2013, nilchian_differential_2012, nilchian_spline_2015}.
They differ in the approximation of the evaluation of the X-ray transform. The former use a
footprint of the B-Splines, where the later rely on the first derivative of the B-Spline basis
function.

\begin{definition}[B-Spline]
	The most basic definition of a B-Spline of degree \(0\) and unit width is the step function:
	\begin{equation}
		\beta^0(x) = \mu(x) =
		\begin{cases}
			1, & \text{if } x \in \mathopen[\minus \frac{1}{2}, \frac{1}{2}\mathclose] \\
			0, & \text{otherwise}
		\end{cases}
	\end{equation}
	Then the univariate B-Spline of degree \(d\) can be constructed by \(d + 1\) convolution of \(\beta^0\)
	(compare~\cite{momey_new_2011}):
	\begin{equation}
		\beta^d(x) = \beta^0 * \beta^{d-1}(x) =
		\underbrace{\beta^0 * \dots * \beta^0(x)}_{d+1 \text{convolution terms}}
	\end{equation}
\end{definition}
\inlinetodo{Add figure for B-Splines of different degree, add function approximation}
Note that B-Splines of degree \(0\) is just the voxel-basis function.

Another way to compute the B-Spline basis of order \(d\) is given in~\cite{unser_fast_1991}:
\begin{equation}
	\beta^d(x) = \sum_{i=0}^{d+1} \frac{(-1)^i}{n!} \binom{d+1}{i}(x - i)^d\mu(x - i)
\end{equation}
where \(\binom{d+1}{i}\) is the binomial coefficient.

The derivatives are again B-Spline of degree \(d-1\) (compare~\cite{unser_splines_1999}):
\begin{equation}
	\frac{\partial \beta^d(x)}{\partial x} = \beta^{d-1}\left(x + \frac{1}{2}\right) -
	\beta^{d-1}\left(x - \frac{1}{2}\right)
\end{equation}
B-Splines are continuously differentiable up to order \(d-1\).

B-Splines are separable. Hence, \(n\)-dimensional B-Splines, often referred to as tensor product
B-Splines, can be constructed the following way:
\begin{equation}
	\beta^d(x) = \prod^n_{i=1} \beta^d(x_i)
\end{equation}
where \(x \in \mathbb{R}^n\)

B-Splines have a couple of really attractive properties. B-Splines are the shortest and smoothest
scaling functions for a given order of approximation~\cite{momey_b-spline_2012}. They are close to a
Gaussian function, with a sufficiently large $d$~\cite{momey_b-spline_2012}, all while preserving
compactness. Hence, they tend to spherically symmetric functions, while preserving local support.
Due to these properties, \citeauthor*{momey_new_2011}\cite{momey_new_2011} argue for B-Splines over
blobs. Further, they note the need to tune the parameters for of blobs for optimal results, which
adds complexity.

Importantly, blobs fail to satisfy the partition of unity~\cite{nilchian_fast_2013}. A basis
functions that satisfies the partition of unity, can approximate any input function arbitrarily
close.~\cite{nilchian_fast_2013} show the importance of the property for tomographic reconstruction.

In~\cite{horbelt_discretization_2002}, it was shown that the Radon transform of B-Splines are spline
bikernel. \citeauthor*{entezari_box_2012} show in~\cite{entezari_box_2012} explicitly how (tensor
product) B-Splines act under the X-ray and Radon transform.~\cite{nilchian_differential_2012} shows
how B-Splines act under the first derivative of the Radon transform.

The Radon Transform of a two-dimensional B-Spline was shown by
\citeauthor*{horbelt_discretization_2002}\cite{horbelt_discretization_2002}. Given the projection
angle \(\theta\), then it is:
\begin{equation}
	\radon\beta^d(s) = \beta^d_{\sin\theta} * \beta^d_{\cos\theta}(d)
\end{equation}
Hence it is the convolution of two splines of different width (denoted by the subscript). These are
referred to as spline bikernels. \citeauthor*{horbelt_discretization_2002} presents an explicit
formulation to compute it.

To extend this to the \(3\)-dimensional setting, one can look into \textit{box splines}. Box splines
can be seen as a generalization of B-Splines.

\begin{definition}[Box Spline]
	Box splines are the shadow of a hypercube in \(\mathbb{R}^n\), when projected down to a
	lower dimension \(\mathbb{R}^d\). Similarly to B-Splines, box splines can be defined via
	convolution:
	\begin{equation}
		M_\Xi(x) = M_{\xi_1} * \dots * M_{\xi_n}(x)
	\end{equation}
	where \(\Xi \coloneq \mathopen[ \xi_1 \xi_2 \dots \xi_n \mathclose] \in \mathbb{R}^{s \times n}\)
	is the matrix of directions. Each \(\xi\) defines a direction of the hypercube.
	\(\Xi\) completely defines the box spline (compare~\cite{de_boor_box_1993})
\end{definition}

\citeauthor{entezari_box_2012}~\cite{entezari_box_2012} proof that the X-ray transform of a
\(d\)-variate box spline is a \(d - 1\) variate box spline. Further, it is the box spline defined
the by projection of the direction matrix \(\Xi\). This means, going back to B-Splines, that for any
dimension, the X-ray transform of B-Splines are again B-Splines of lower dimension.

\chapter{Tomographic Reconstruction}\label{chap:tomographic_reconstruction}

Till this point, the physical models involved in tomographic reconstruction have been presented and
discretization has been discussed. The last missing piece is the solution to the inverse problem.
As tomographic reconstruction problems are inverse problems, many methods depend on common solutions
to this space.

Generally, one needs to find solution to the system \(A(f) = m\). There \(A\) is the forward model,
\(f\) is the \(n\)-dimensional signal one seeks to reconstruct, and \(m\) is the measured data, i.e.\
the projected data. In the specific case of X-ray attenuation CT, the forward model \(A\) is the
Radon Transform \(\radon\) or the X-ray Transform \(\xray\), the signal \(f\) is the function of
attenuation coefficients of the scanned object \(\mu\), and \(m\) is the data measured at the
detector (including noise).

\section{Analytical Reconstruction}\label{sec:analytical_reconstruction}

As already alluded to in the section about the Radon Transform \ref{sec:radon_transform}. There do
exist closed form analytical inversion methods for the Radon Transform. In the aforementioned
section, the Fourier Slice Theorem was introduced. Methods exist that use this approach as a means
to reconstruct the desired signal. However, they are not often used in practice in tomographic
reconstruction.

\begin{definition}[Back-Projection]\label{def:back_projection}
	Let \(f\colon \Omega \to \mathbb{R}\), where \(\Omega \in \mathbb{R}^2\) sufficiently nice.
	Further, let \(g_\phi \coloneq \radon f\) be the Radon Transform of \(f\) with the
	projection angle \(\phi\). Then
	\[ (R^\ast g)(x, y) \coloneq \int_0^\pi g_\phi(x\cos \phi + y \sin\phi) \mathrm{d}\phi \]
	is the unfiltered back-projection of \(g\) (c.f.~\cite{buzug_computed_2008}).
	\(\radon^\ast\) denotes the adjoint of Radon Transform.
\end{definition}

\inlinetodo{Add example images for 1, 2, 4, and so on projection angles}

The result of back-projecting the measured projections is a blurry version of the original function.
\citeauthor{buzug_computed_2008} describes post-processing as a possible solution. However, there
exists another option. If the projections are filtered in the Fourier domain and then the filtered
values are back-projected just as before, a sharper image can be obtained. This is referred to as
the filtered back-projection (FBP)~\cite{ramachandran_three-dimensional_1971}.

\begin{definition}[Filtered Back-Projection]\label{def:filtered_back_projection}
	Still, let \(f\colon \Omega \to \mathbb{R}\), where \(\Omega \in \mathbb{R}^2\) sufficiently
	nice, and \(g \coloneq \radon f\) be the Radon Transform of \(f\). Then
	\[ g^\delta(t, \phi) \coloneq (f \ast g(\cdot, \phi))(t) \]
	is the filtered projection. \(\delta(x) \approx \abs{x}\) is a filter, which is convoled
	with the projection data. Using \(\radon^\ast\) is the adjoint of the Radon Transform
	as in \autoref{def:back_projection}, then \(\radon^\ast g^\delta\) is the
	\textit{filtered back-projection}.
\end{definition}

The quality of the reconstruction depends on the filter and data acquisition. Further the FBP as
presented here, is limited to parallel beam geometry setups. There do exist other methods for fan
beam settings that require rebinning (i.e.\ sorting the projections, such that all rays are
parallel). However, overall the FBP leads to sharp images and it is widely used in medical CT
scanners~\cite{pan_why_2009}.

\inlinetodo{Use something like to show packprojection and filtered back projection
	\href{https://scikit-image.org/docs/dev/auto_examples/transform/plot_radon_transform.html} to create
	some nice graps and include code to reproduce results. Show how FBP fails}

\section{Towards the Matrix form}\label{sec:matrix_formulation}

The goal of this section is a short run through the steps that are necessary to go from the
continuous definition of the inverse problem as given in \autoref{def:inverse_problem} to a discrete
version, with which we can work on computers. In the previous section, the FBP still assumes a
continuous function to work with, which can create problems during reconstruction. For this reason,
it is desirable to find a suitable discretization and work with different algorithms.

In \autoref{def:forward-model}, the definition of forward model is given. This forward model
takes the place of the bounded linear operator \(A\) from \autoref{def:inverse_problem}. However,
this is all still in the continuous space. Using \autoref{def:permissible_representation}, \(f\) can
decompose into the sum of coefficients and basis functions:
\[ f \approx \near{f}(x) = \sum_{k=1}^{N} c_k \varphi_k(\mvec{x}) \]
as shown for the Radon Transform, the forward model can be applied and rearranging a little using
the linearity of the operator:
\[ m_j \approx \mathscr{M}_j(\hat{f}) = \sum_{k=1}^{N} c_k \mathscr{M}_j(\varphi_k) \]
\(a_{ji} \coloneq \mathscr{M}_j(\varphi_k)\) is the contribution of a single \(k\)th basis function
to the \(j\)th measurement.

\inlinetodo{Have an image like Vogel Beyond tomographic Figure 4.7}

Now, the measurements can be stacked to a vector \(m = (m_j) \in \mathbb{R}^J\), the same for
the coefficients \(c = (c_i) \in \mathbb{R}^I\), and the contributions \(a_{j} = (a_{ji}) \in
\mathbb{R}^I\). Then a single measurement can be written as a scalar product of the coefficient
vector and the contribution vector. But also importantly, the linear system can be defined:
\begin{equation}\label{eq:system_lin_equation}
	m \approx
	\begin{bmatrix}
		\rule[.5ex]{2em}{0.4pt} & a_1^T & \rule[.5ex]{2em}{0.4pt} \\
		\rule[.5ex]{2em}{0.4pt} & a_2^T & \rule[.5ex]{2em}{0.4pt} \\
		\vdots                                                    \\
		\rule[.5ex]{2em}{0.4pt} & a_J^T & \rule[.5ex]{2em}{0.4pt}
	\end{bmatrix} c \eqcolon A c
\end{equation}

This is a regular system of linear equation, it partitions the problem in the measurements, the
\textit{system matrix} \(A \in \mathbb{R}^{J \times I}\) and the coefficient vector \(c\). Also
note, that the choice of basis function is integrated into the system matrix. In
\autoref{fig:matrix_row}, the intuition for the system matrix is illustrated. There, for a single
measurement \(m_1\), the matrix row \(a_{1} = \begin{bmatrix}a_{1, 1}, a_{1, 2} \cdots
	a_{1,36}\end{bmatrix}\) is visualized. Each element of the row corresponds to an element of the
coefficient vector \(\mvec{c}\). One can also guess from this visualization the sparsity of the
rows, as already in this simple illustration, roughly half the entries of the row are \(0\).

\begin{figure}
	\centering
	%% Creator: Matplotlib, PGF backend
%%
%% To include the figure in your LaTeX document, write
%%   \input{<filename>.pgf}
%%
%% Make sure the required packages are loaded in your preamble
%%   \usepackage{pgf}
%%
%% Also ensure that all the required font packages are loaded; for instance,
%% the lmodern package is sometimes necessary when using math font.
%%   \usepackage{lmodern}
%%
%% Figures using additional raster images can only be included by \input if
%% they are in the same directory as the main LaTeX file. For loading figures
%% from other directories you can use the `import` package
%%   \usepackage{import}
%%
%% and then include the figures with
%%   \import{<path to file>}{<filename>.pgf}
%%
%% Matplotlib used the following preamble
%%   \usepackage{fontspec}
%%   \setmainfont{DejaVuSerif.ttf}[Path=\detokenize{/usr/lib/python3.10/site-packages/matplotlib/mpl-data/fonts/ttf/}]
%%   \setsansfont{DejaVuSans.ttf}[Path=\detokenize{/usr/lib/python3.10/site-packages/matplotlib/mpl-data/fonts/ttf/}]
%%   \setmonofont{DejaVuSansMono.ttf}[Path=\detokenize{/usr/lib/python3.10/site-packages/matplotlib/mpl-data/fonts/ttf/}]
%%
\begingroup%
\makeatletter%
\begin{pgfpicture}%
\pgfpathrectangle{\pgfpointorigin}{\pgfqpoint{5.963393in}{4.543538in}}%
\pgfusepath{use as bounding box, clip}%
\begin{pgfscope}%
\pgfsetbuttcap%
\pgfsetmiterjoin%
\definecolor{currentfill}{rgb}{1.000000,1.000000,1.000000}%
\pgfsetfillcolor{currentfill}%
\pgfsetlinewidth{0.000000pt}%
\definecolor{currentstroke}{rgb}{1.000000,1.000000,1.000000}%
\pgfsetstrokecolor{currentstroke}%
\pgfsetdash{}{0pt}%
\pgfpathmoveto{\pgfqpoint{0.000000in}{0.000000in}}%
\pgfpathlineto{\pgfqpoint{5.963393in}{0.000000in}}%
\pgfpathlineto{\pgfqpoint{5.963393in}{4.543538in}}%
\pgfpathlineto{\pgfqpoint{0.000000in}{4.543538in}}%
\pgfpathlineto{\pgfqpoint{0.000000in}{0.000000in}}%
\pgfpathclose%
\pgfusepath{fill}%
\end{pgfscope}%
\begin{pgfscope}%
\pgfpathrectangle{\pgfqpoint{0.000000in}{0.000000in}}{\pgfqpoint{5.963393in}{4.543538in}}%
\pgfusepath{clip}%
\pgfsetbuttcap%
\pgfsetmiterjoin%
\definecolor{currentfill}{rgb}{1.000000,1.000000,1.000000}%
\pgfsetfillcolor{currentfill}%
\pgfsetlinewidth{1.003750pt}%
\definecolor{currentstroke}{rgb}{0.000000,0.000000,0.000000}%
\pgfsetstrokecolor{currentstroke}%
\pgfsetdash{}{0pt}%
\pgfpathmoveto{\pgfqpoint{1.419855in}{0.567942in}}%
\pgfpathlineto{\pgfqpoint{1.419855in}{1.135884in}}%
\pgfpathlineto{\pgfqpoint{1.987798in}{1.135884in}}%
\pgfpathlineto{\pgfqpoint{1.987798in}{0.567942in}}%
\pgfpathlineto{\pgfqpoint{1.419855in}{0.567942in}}%
\pgfpathclose%
\pgfusepath{stroke,fill}%
\end{pgfscope}%
\begin{pgfscope}%
\pgfpathrectangle{\pgfqpoint{0.000000in}{0.000000in}}{\pgfqpoint{5.963393in}{4.543538in}}%
\pgfusepath{clip}%
\pgfsetbuttcap%
\pgfsetmiterjoin%
\definecolor{currentfill}{rgb}{1.000000,1.000000,1.000000}%
\pgfsetfillcolor{currentfill}%
\pgfsetlinewidth{1.003750pt}%
\definecolor{currentstroke}{rgb}{0.000000,0.000000,0.000000}%
\pgfsetstrokecolor{currentstroke}%
\pgfsetdash{}{0pt}%
\pgfpathmoveto{\pgfqpoint{1.419855in}{1.135884in}}%
\pgfpathlineto{\pgfqpoint{1.419855in}{1.703827in}}%
\pgfpathlineto{\pgfqpoint{1.987798in}{1.703827in}}%
\pgfpathlineto{\pgfqpoint{1.987798in}{1.135884in}}%
\pgfpathlineto{\pgfqpoint{1.419855in}{1.135884in}}%
\pgfpathclose%
\pgfusepath{stroke,fill}%
\end{pgfscope}%
\begin{pgfscope}%
\pgfpathrectangle{\pgfqpoint{0.000000in}{0.000000in}}{\pgfqpoint{5.963393in}{4.543538in}}%
\pgfusepath{clip}%
\pgfsetbuttcap%
\pgfsetmiterjoin%
\definecolor{currentfill}{rgb}{1.000000,1.000000,1.000000}%
\pgfsetfillcolor{currentfill}%
\pgfsetlinewidth{1.003750pt}%
\definecolor{currentstroke}{rgb}{0.000000,0.000000,0.000000}%
\pgfsetstrokecolor{currentstroke}%
\pgfsetdash{}{0pt}%
\pgfpathmoveto{\pgfqpoint{1.419855in}{1.703827in}}%
\pgfpathlineto{\pgfqpoint{1.419855in}{2.271769in}}%
\pgfpathlineto{\pgfqpoint{1.987798in}{2.271769in}}%
\pgfpathlineto{\pgfqpoint{1.987798in}{1.703827in}}%
\pgfpathlineto{\pgfqpoint{1.419855in}{1.703827in}}%
\pgfpathclose%
\pgfusepath{stroke,fill}%
\end{pgfscope}%
\begin{pgfscope}%
\pgfpathrectangle{\pgfqpoint{0.000000in}{0.000000in}}{\pgfqpoint{5.963393in}{4.543538in}}%
\pgfusepath{clip}%
\pgfsetbuttcap%
\pgfsetmiterjoin%
\definecolor{currentfill}{rgb}{1.000000,1.000000,1.000000}%
\pgfsetfillcolor{currentfill}%
\pgfsetlinewidth{1.003750pt}%
\definecolor{currentstroke}{rgb}{0.000000,0.000000,0.000000}%
\pgfsetstrokecolor{currentstroke}%
\pgfsetdash{}{0pt}%
\pgfpathmoveto{\pgfqpoint{1.419855in}{2.271769in}}%
\pgfpathlineto{\pgfqpoint{1.419855in}{2.839711in}}%
\pgfpathlineto{\pgfqpoint{1.987798in}{2.839711in}}%
\pgfpathlineto{\pgfqpoint{1.987798in}{2.271769in}}%
\pgfpathlineto{\pgfqpoint{1.419855in}{2.271769in}}%
\pgfpathclose%
\pgfusepath{stroke,fill}%
\end{pgfscope}%
\begin{pgfscope}%
\pgfpathrectangle{\pgfqpoint{0.000000in}{0.000000in}}{\pgfqpoint{5.963393in}{4.543538in}}%
\pgfusepath{clip}%
\pgfsetbuttcap%
\pgfsetmiterjoin%
\definecolor{currentfill}{rgb}{1.000000,1.000000,1.000000}%
\pgfsetfillcolor{currentfill}%
\pgfsetlinewidth{1.003750pt}%
\definecolor{currentstroke}{rgb}{0.000000,0.000000,0.000000}%
\pgfsetstrokecolor{currentstroke}%
\pgfsetdash{}{0pt}%
\pgfpathmoveto{\pgfqpoint{1.419855in}{2.839711in}}%
\pgfpathlineto{\pgfqpoint{1.419855in}{3.407653in}}%
\pgfpathlineto{\pgfqpoint{1.987798in}{3.407653in}}%
\pgfpathlineto{\pgfqpoint{1.987798in}{2.839711in}}%
\pgfpathlineto{\pgfqpoint{1.419855in}{2.839711in}}%
\pgfpathclose%
\pgfusepath{stroke,fill}%
\end{pgfscope}%
\begin{pgfscope}%
\pgfpathrectangle{\pgfqpoint{0.000000in}{0.000000in}}{\pgfqpoint{5.963393in}{4.543538in}}%
\pgfusepath{clip}%
\pgfsetbuttcap%
\pgfsetmiterjoin%
\definecolor{currentfill}{rgb}{1.000000,1.000000,1.000000}%
\pgfsetfillcolor{currentfill}%
\pgfsetlinewidth{1.003750pt}%
\definecolor{currentstroke}{rgb}{0.000000,0.000000,0.000000}%
\pgfsetstrokecolor{currentstroke}%
\pgfsetdash{}{0pt}%
\pgfpathmoveto{\pgfqpoint{1.419855in}{3.407653in}}%
\pgfpathlineto{\pgfqpoint{1.419855in}{3.975595in}}%
\pgfpathlineto{\pgfqpoint{1.987798in}{3.975595in}}%
\pgfpathlineto{\pgfqpoint{1.987798in}{3.407653in}}%
\pgfpathlineto{\pgfqpoint{1.419855in}{3.407653in}}%
\pgfpathclose%
\pgfusepath{stroke,fill}%
\end{pgfscope}%
\begin{pgfscope}%
\pgfpathrectangle{\pgfqpoint{0.000000in}{0.000000in}}{\pgfqpoint{5.963393in}{4.543538in}}%
\pgfusepath{clip}%
\pgfsetbuttcap%
\pgfsetmiterjoin%
\definecolor{currentfill}{rgb}{1.000000,1.000000,1.000000}%
\pgfsetfillcolor{currentfill}%
\pgfsetlinewidth{1.003750pt}%
\definecolor{currentstroke}{rgb}{0.000000,0.000000,0.000000}%
\pgfsetstrokecolor{currentstroke}%
\pgfsetdash{}{0pt}%
\pgfpathmoveto{\pgfqpoint{1.987798in}{0.567942in}}%
\pgfpathlineto{\pgfqpoint{1.987798in}{1.135884in}}%
\pgfpathlineto{\pgfqpoint{2.555740in}{1.135884in}}%
\pgfpathlineto{\pgfqpoint{2.555740in}{0.567942in}}%
\pgfpathlineto{\pgfqpoint{1.987798in}{0.567942in}}%
\pgfpathclose%
\pgfusepath{stroke,fill}%
\end{pgfscope}%
\begin{pgfscope}%
\pgfpathrectangle{\pgfqpoint{0.000000in}{0.000000in}}{\pgfqpoint{5.963393in}{4.543538in}}%
\pgfusepath{clip}%
\pgfsetbuttcap%
\pgfsetmiterjoin%
\definecolor{currentfill}{rgb}{1.000000,1.000000,1.000000}%
\pgfsetfillcolor{currentfill}%
\pgfsetlinewidth{1.003750pt}%
\definecolor{currentstroke}{rgb}{0.000000,0.000000,0.000000}%
\pgfsetstrokecolor{currentstroke}%
\pgfsetdash{}{0pt}%
\pgfpathmoveto{\pgfqpoint{1.987798in}{1.135884in}}%
\pgfpathlineto{\pgfqpoint{1.987798in}{1.703827in}}%
\pgfpathlineto{\pgfqpoint{2.555740in}{1.703827in}}%
\pgfpathlineto{\pgfqpoint{2.555740in}{1.135884in}}%
\pgfpathlineto{\pgfqpoint{1.987798in}{1.135884in}}%
\pgfpathclose%
\pgfusepath{stroke,fill}%
\end{pgfscope}%
\begin{pgfscope}%
\pgfpathrectangle{\pgfqpoint{0.000000in}{0.000000in}}{\pgfqpoint{5.963393in}{4.543538in}}%
\pgfusepath{clip}%
\pgfsetbuttcap%
\pgfsetmiterjoin%
\definecolor{currentfill}{rgb}{1.000000,1.000000,1.000000}%
\pgfsetfillcolor{currentfill}%
\pgfsetlinewidth{1.003750pt}%
\definecolor{currentstroke}{rgb}{0.000000,0.000000,0.000000}%
\pgfsetstrokecolor{currentstroke}%
\pgfsetdash{}{0pt}%
\pgfpathmoveto{\pgfqpoint{1.987798in}{1.703827in}}%
\pgfpathlineto{\pgfqpoint{1.987798in}{2.271769in}}%
\pgfpathlineto{\pgfqpoint{2.555740in}{2.271769in}}%
\pgfpathlineto{\pgfqpoint{2.555740in}{1.703827in}}%
\pgfpathlineto{\pgfqpoint{1.987798in}{1.703827in}}%
\pgfpathclose%
\pgfusepath{stroke,fill}%
\end{pgfscope}%
\begin{pgfscope}%
\pgfpathrectangle{\pgfqpoint{0.000000in}{0.000000in}}{\pgfqpoint{5.963393in}{4.543538in}}%
\pgfusepath{clip}%
\pgfsetbuttcap%
\pgfsetmiterjoin%
\definecolor{currentfill}{rgb}{1.000000,1.000000,1.000000}%
\pgfsetfillcolor{currentfill}%
\pgfsetlinewidth{1.003750pt}%
\definecolor{currentstroke}{rgb}{0.000000,0.000000,0.000000}%
\pgfsetstrokecolor{currentstroke}%
\pgfsetdash{}{0pt}%
\pgfpathmoveto{\pgfqpoint{1.987798in}{2.271769in}}%
\pgfpathlineto{\pgfqpoint{1.987798in}{2.839711in}}%
\pgfpathlineto{\pgfqpoint{2.555740in}{2.839711in}}%
\pgfpathlineto{\pgfqpoint{2.555740in}{2.271769in}}%
\pgfpathlineto{\pgfqpoint{1.987798in}{2.271769in}}%
\pgfpathclose%
\pgfusepath{stroke,fill}%
\end{pgfscope}%
\begin{pgfscope}%
\pgfpathrectangle{\pgfqpoint{0.000000in}{0.000000in}}{\pgfqpoint{5.963393in}{4.543538in}}%
\pgfusepath{clip}%
\pgfsetbuttcap%
\pgfsetmiterjoin%
\definecolor{currentfill}{rgb}{1.000000,1.000000,1.000000}%
\pgfsetfillcolor{currentfill}%
\pgfsetlinewidth{1.003750pt}%
\definecolor{currentstroke}{rgb}{0.000000,0.000000,0.000000}%
\pgfsetstrokecolor{currentstroke}%
\pgfsetdash{}{0pt}%
\pgfpathmoveto{\pgfqpoint{1.987798in}{2.839711in}}%
\pgfpathlineto{\pgfqpoint{1.987798in}{3.407653in}}%
\pgfpathlineto{\pgfqpoint{2.555740in}{3.407653in}}%
\pgfpathlineto{\pgfqpoint{2.555740in}{2.839711in}}%
\pgfpathlineto{\pgfqpoint{1.987798in}{2.839711in}}%
\pgfpathclose%
\pgfusepath{stroke,fill}%
\end{pgfscope}%
\begin{pgfscope}%
\pgfpathrectangle{\pgfqpoint{0.000000in}{0.000000in}}{\pgfqpoint{5.963393in}{4.543538in}}%
\pgfusepath{clip}%
\pgfsetbuttcap%
\pgfsetmiterjoin%
\definecolor{currentfill}{rgb}{1.000000,1.000000,1.000000}%
\pgfsetfillcolor{currentfill}%
\pgfsetlinewidth{1.003750pt}%
\definecolor{currentstroke}{rgb}{0.000000,0.000000,0.000000}%
\pgfsetstrokecolor{currentstroke}%
\pgfsetdash{}{0pt}%
\pgfpathmoveto{\pgfqpoint{1.987798in}{3.407653in}}%
\pgfpathlineto{\pgfqpoint{1.987798in}{3.975595in}}%
\pgfpathlineto{\pgfqpoint{2.555740in}{3.975595in}}%
\pgfpathlineto{\pgfqpoint{2.555740in}{3.407653in}}%
\pgfpathlineto{\pgfqpoint{1.987798in}{3.407653in}}%
\pgfpathclose%
\pgfusepath{stroke,fill}%
\end{pgfscope}%
\begin{pgfscope}%
\pgfpathrectangle{\pgfqpoint{0.000000in}{0.000000in}}{\pgfqpoint{5.963393in}{4.543538in}}%
\pgfusepath{clip}%
\pgfsetbuttcap%
\pgfsetmiterjoin%
\definecolor{currentfill}{rgb}{1.000000,1.000000,1.000000}%
\pgfsetfillcolor{currentfill}%
\pgfsetlinewidth{1.003750pt}%
\definecolor{currentstroke}{rgb}{0.000000,0.000000,0.000000}%
\pgfsetstrokecolor{currentstroke}%
\pgfsetdash{}{0pt}%
\pgfpathmoveto{\pgfqpoint{2.555740in}{0.567942in}}%
\pgfpathlineto{\pgfqpoint{2.555740in}{1.135884in}}%
\pgfpathlineto{\pgfqpoint{3.123682in}{1.135884in}}%
\pgfpathlineto{\pgfqpoint{3.123682in}{0.567942in}}%
\pgfpathlineto{\pgfqpoint{2.555740in}{0.567942in}}%
\pgfpathclose%
\pgfusepath{stroke,fill}%
\end{pgfscope}%
\begin{pgfscope}%
\pgfpathrectangle{\pgfqpoint{0.000000in}{0.000000in}}{\pgfqpoint{5.963393in}{4.543538in}}%
\pgfusepath{clip}%
\pgfsetbuttcap%
\pgfsetmiterjoin%
\definecolor{currentfill}{rgb}{1.000000,1.000000,1.000000}%
\pgfsetfillcolor{currentfill}%
\pgfsetlinewidth{1.003750pt}%
\definecolor{currentstroke}{rgb}{0.000000,0.000000,0.000000}%
\pgfsetstrokecolor{currentstroke}%
\pgfsetdash{}{0pt}%
\pgfpathmoveto{\pgfqpoint{2.555740in}{1.135884in}}%
\pgfpathlineto{\pgfqpoint{2.555740in}{1.703827in}}%
\pgfpathlineto{\pgfqpoint{3.123682in}{1.703827in}}%
\pgfpathlineto{\pgfqpoint{3.123682in}{1.135884in}}%
\pgfpathlineto{\pgfqpoint{2.555740in}{1.135884in}}%
\pgfpathclose%
\pgfusepath{stroke,fill}%
\end{pgfscope}%
\begin{pgfscope}%
\pgfpathrectangle{\pgfqpoint{0.000000in}{0.000000in}}{\pgfqpoint{5.963393in}{4.543538in}}%
\pgfusepath{clip}%
\pgfsetbuttcap%
\pgfsetmiterjoin%
\definecolor{currentfill}{rgb}{1.000000,1.000000,1.000000}%
\pgfsetfillcolor{currentfill}%
\pgfsetlinewidth{1.003750pt}%
\definecolor{currentstroke}{rgb}{0.000000,0.000000,0.000000}%
\pgfsetstrokecolor{currentstroke}%
\pgfsetdash{}{0pt}%
\pgfpathmoveto{\pgfqpoint{2.555740in}{1.703827in}}%
\pgfpathlineto{\pgfqpoint{2.555740in}{2.271769in}}%
\pgfpathlineto{\pgfqpoint{3.123682in}{2.271769in}}%
\pgfpathlineto{\pgfqpoint{3.123682in}{1.703827in}}%
\pgfpathlineto{\pgfqpoint{2.555740in}{1.703827in}}%
\pgfpathclose%
\pgfusepath{stroke,fill}%
\end{pgfscope}%
\begin{pgfscope}%
\pgfpathrectangle{\pgfqpoint{0.000000in}{0.000000in}}{\pgfqpoint{5.963393in}{4.543538in}}%
\pgfusepath{clip}%
\pgfsetbuttcap%
\pgfsetmiterjoin%
\definecolor{currentfill}{rgb}{1.000000,1.000000,1.000000}%
\pgfsetfillcolor{currentfill}%
\pgfsetlinewidth{1.003750pt}%
\definecolor{currentstroke}{rgb}{0.000000,0.000000,0.000000}%
\pgfsetstrokecolor{currentstroke}%
\pgfsetdash{}{0pt}%
\pgfpathmoveto{\pgfqpoint{2.555740in}{2.271769in}}%
\pgfpathlineto{\pgfqpoint{2.555740in}{2.839711in}}%
\pgfpathlineto{\pgfqpoint{3.123682in}{2.839711in}}%
\pgfpathlineto{\pgfqpoint{3.123682in}{2.271769in}}%
\pgfpathlineto{\pgfqpoint{2.555740in}{2.271769in}}%
\pgfpathclose%
\pgfusepath{stroke,fill}%
\end{pgfscope}%
\begin{pgfscope}%
\pgfpathrectangle{\pgfqpoint{0.000000in}{0.000000in}}{\pgfqpoint{5.963393in}{4.543538in}}%
\pgfusepath{clip}%
\pgfsetbuttcap%
\pgfsetmiterjoin%
\definecolor{currentfill}{rgb}{1.000000,1.000000,1.000000}%
\pgfsetfillcolor{currentfill}%
\pgfsetlinewidth{1.003750pt}%
\definecolor{currentstroke}{rgb}{0.000000,0.000000,0.000000}%
\pgfsetstrokecolor{currentstroke}%
\pgfsetdash{}{0pt}%
\pgfpathmoveto{\pgfqpoint{2.555740in}{2.839711in}}%
\pgfpathlineto{\pgfqpoint{2.555740in}{3.407653in}}%
\pgfpathlineto{\pgfqpoint{3.123682in}{3.407653in}}%
\pgfpathlineto{\pgfqpoint{3.123682in}{2.839711in}}%
\pgfpathlineto{\pgfqpoint{2.555740in}{2.839711in}}%
\pgfpathclose%
\pgfusepath{stroke,fill}%
\end{pgfscope}%
\begin{pgfscope}%
\pgfpathrectangle{\pgfqpoint{0.000000in}{0.000000in}}{\pgfqpoint{5.963393in}{4.543538in}}%
\pgfusepath{clip}%
\pgfsetbuttcap%
\pgfsetmiterjoin%
\definecolor{currentfill}{rgb}{1.000000,1.000000,1.000000}%
\pgfsetfillcolor{currentfill}%
\pgfsetlinewidth{1.003750pt}%
\definecolor{currentstroke}{rgb}{0.000000,0.000000,0.000000}%
\pgfsetstrokecolor{currentstroke}%
\pgfsetdash{}{0pt}%
\pgfpathmoveto{\pgfqpoint{2.555740in}{3.407653in}}%
\pgfpathlineto{\pgfqpoint{2.555740in}{3.975595in}}%
\pgfpathlineto{\pgfqpoint{3.123682in}{3.975595in}}%
\pgfpathlineto{\pgfqpoint{3.123682in}{3.407653in}}%
\pgfpathlineto{\pgfqpoint{2.555740in}{3.407653in}}%
\pgfpathclose%
\pgfusepath{stroke,fill}%
\end{pgfscope}%
\begin{pgfscope}%
\pgfpathrectangle{\pgfqpoint{0.000000in}{0.000000in}}{\pgfqpoint{5.963393in}{4.543538in}}%
\pgfusepath{clip}%
\pgfsetbuttcap%
\pgfsetmiterjoin%
\definecolor{currentfill}{rgb}{1.000000,1.000000,1.000000}%
\pgfsetfillcolor{currentfill}%
\pgfsetlinewidth{1.003750pt}%
\definecolor{currentstroke}{rgb}{0.000000,0.000000,0.000000}%
\pgfsetstrokecolor{currentstroke}%
\pgfsetdash{}{0pt}%
\pgfpathmoveto{\pgfqpoint{3.123682in}{0.567942in}}%
\pgfpathlineto{\pgfqpoint{3.123682in}{1.135884in}}%
\pgfpathlineto{\pgfqpoint{3.691624in}{1.135884in}}%
\pgfpathlineto{\pgfqpoint{3.691624in}{0.567942in}}%
\pgfpathlineto{\pgfqpoint{3.123682in}{0.567942in}}%
\pgfpathclose%
\pgfusepath{stroke,fill}%
\end{pgfscope}%
\begin{pgfscope}%
\pgfpathrectangle{\pgfqpoint{0.000000in}{0.000000in}}{\pgfqpoint{5.963393in}{4.543538in}}%
\pgfusepath{clip}%
\pgfsetbuttcap%
\pgfsetmiterjoin%
\definecolor{currentfill}{rgb}{1.000000,1.000000,1.000000}%
\pgfsetfillcolor{currentfill}%
\pgfsetlinewidth{1.003750pt}%
\definecolor{currentstroke}{rgb}{0.000000,0.000000,0.000000}%
\pgfsetstrokecolor{currentstroke}%
\pgfsetdash{}{0pt}%
\pgfpathmoveto{\pgfqpoint{3.123682in}{1.135884in}}%
\pgfpathlineto{\pgfqpoint{3.123682in}{1.703827in}}%
\pgfpathlineto{\pgfqpoint{3.691624in}{1.703827in}}%
\pgfpathlineto{\pgfqpoint{3.691624in}{1.135884in}}%
\pgfpathlineto{\pgfqpoint{3.123682in}{1.135884in}}%
\pgfpathclose%
\pgfusepath{stroke,fill}%
\end{pgfscope}%
\begin{pgfscope}%
\pgfpathrectangle{\pgfqpoint{0.000000in}{0.000000in}}{\pgfqpoint{5.963393in}{4.543538in}}%
\pgfusepath{clip}%
\pgfsetbuttcap%
\pgfsetmiterjoin%
\definecolor{currentfill}{rgb}{1.000000,1.000000,1.000000}%
\pgfsetfillcolor{currentfill}%
\pgfsetlinewidth{1.003750pt}%
\definecolor{currentstroke}{rgb}{0.000000,0.000000,0.000000}%
\pgfsetstrokecolor{currentstroke}%
\pgfsetdash{}{0pt}%
\pgfpathmoveto{\pgfqpoint{3.123682in}{1.703827in}}%
\pgfpathlineto{\pgfqpoint{3.123682in}{2.271769in}}%
\pgfpathlineto{\pgfqpoint{3.691624in}{2.271769in}}%
\pgfpathlineto{\pgfqpoint{3.691624in}{1.703827in}}%
\pgfpathlineto{\pgfqpoint{3.123682in}{1.703827in}}%
\pgfpathclose%
\pgfusepath{stroke,fill}%
\end{pgfscope}%
\begin{pgfscope}%
\pgfpathrectangle{\pgfqpoint{0.000000in}{0.000000in}}{\pgfqpoint{5.963393in}{4.543538in}}%
\pgfusepath{clip}%
\pgfsetbuttcap%
\pgfsetmiterjoin%
\definecolor{currentfill}{rgb}{1.000000,1.000000,1.000000}%
\pgfsetfillcolor{currentfill}%
\pgfsetlinewidth{1.003750pt}%
\definecolor{currentstroke}{rgb}{0.000000,0.000000,0.000000}%
\pgfsetstrokecolor{currentstroke}%
\pgfsetdash{}{0pt}%
\pgfpathmoveto{\pgfqpoint{3.123682in}{2.271769in}}%
\pgfpathlineto{\pgfqpoint{3.123682in}{2.839711in}}%
\pgfpathlineto{\pgfqpoint{3.691624in}{2.839711in}}%
\pgfpathlineto{\pgfqpoint{3.691624in}{2.271769in}}%
\pgfpathlineto{\pgfqpoint{3.123682in}{2.271769in}}%
\pgfpathclose%
\pgfusepath{stroke,fill}%
\end{pgfscope}%
\begin{pgfscope}%
\pgfpathrectangle{\pgfqpoint{0.000000in}{0.000000in}}{\pgfqpoint{5.963393in}{4.543538in}}%
\pgfusepath{clip}%
\pgfsetbuttcap%
\pgfsetmiterjoin%
\definecolor{currentfill}{rgb}{1.000000,1.000000,1.000000}%
\pgfsetfillcolor{currentfill}%
\pgfsetlinewidth{1.003750pt}%
\definecolor{currentstroke}{rgb}{0.000000,0.000000,0.000000}%
\pgfsetstrokecolor{currentstroke}%
\pgfsetdash{}{0pt}%
\pgfpathmoveto{\pgfqpoint{3.123682in}{2.839711in}}%
\pgfpathlineto{\pgfqpoint{3.123682in}{3.407653in}}%
\pgfpathlineto{\pgfqpoint{3.691624in}{3.407653in}}%
\pgfpathlineto{\pgfqpoint{3.691624in}{2.839711in}}%
\pgfpathlineto{\pgfqpoint{3.123682in}{2.839711in}}%
\pgfpathclose%
\pgfusepath{stroke,fill}%
\end{pgfscope}%
\begin{pgfscope}%
\pgfpathrectangle{\pgfqpoint{0.000000in}{0.000000in}}{\pgfqpoint{5.963393in}{4.543538in}}%
\pgfusepath{clip}%
\pgfsetbuttcap%
\pgfsetmiterjoin%
\definecolor{currentfill}{rgb}{1.000000,1.000000,1.000000}%
\pgfsetfillcolor{currentfill}%
\pgfsetlinewidth{1.003750pt}%
\definecolor{currentstroke}{rgb}{0.000000,0.000000,0.000000}%
\pgfsetstrokecolor{currentstroke}%
\pgfsetdash{}{0pt}%
\pgfpathmoveto{\pgfqpoint{3.123682in}{3.407653in}}%
\pgfpathlineto{\pgfqpoint{3.123682in}{3.975595in}}%
\pgfpathlineto{\pgfqpoint{3.691624in}{3.975595in}}%
\pgfpathlineto{\pgfqpoint{3.691624in}{3.407653in}}%
\pgfpathlineto{\pgfqpoint{3.123682in}{3.407653in}}%
\pgfpathclose%
\pgfusepath{stroke,fill}%
\end{pgfscope}%
\begin{pgfscope}%
\pgfpathrectangle{\pgfqpoint{0.000000in}{0.000000in}}{\pgfqpoint{5.963393in}{4.543538in}}%
\pgfusepath{clip}%
\pgfsetbuttcap%
\pgfsetmiterjoin%
\definecolor{currentfill}{rgb}{1.000000,1.000000,1.000000}%
\pgfsetfillcolor{currentfill}%
\pgfsetlinewidth{1.003750pt}%
\definecolor{currentstroke}{rgb}{0.000000,0.000000,0.000000}%
\pgfsetstrokecolor{currentstroke}%
\pgfsetdash{}{0pt}%
\pgfpathmoveto{\pgfqpoint{3.691624in}{0.567942in}}%
\pgfpathlineto{\pgfqpoint{3.691624in}{1.135884in}}%
\pgfpathlineto{\pgfqpoint{4.259566in}{1.135884in}}%
\pgfpathlineto{\pgfqpoint{4.259566in}{0.567942in}}%
\pgfpathlineto{\pgfqpoint{3.691624in}{0.567942in}}%
\pgfpathclose%
\pgfusepath{stroke,fill}%
\end{pgfscope}%
\begin{pgfscope}%
\pgfpathrectangle{\pgfqpoint{0.000000in}{0.000000in}}{\pgfqpoint{5.963393in}{4.543538in}}%
\pgfusepath{clip}%
\pgfsetbuttcap%
\pgfsetmiterjoin%
\definecolor{currentfill}{rgb}{1.000000,1.000000,1.000000}%
\pgfsetfillcolor{currentfill}%
\pgfsetlinewidth{1.003750pt}%
\definecolor{currentstroke}{rgb}{0.000000,0.000000,0.000000}%
\pgfsetstrokecolor{currentstroke}%
\pgfsetdash{}{0pt}%
\pgfpathmoveto{\pgfqpoint{3.691624in}{1.135884in}}%
\pgfpathlineto{\pgfqpoint{3.691624in}{1.703827in}}%
\pgfpathlineto{\pgfqpoint{4.259566in}{1.703827in}}%
\pgfpathlineto{\pgfqpoint{4.259566in}{1.135884in}}%
\pgfpathlineto{\pgfqpoint{3.691624in}{1.135884in}}%
\pgfpathclose%
\pgfusepath{stroke,fill}%
\end{pgfscope}%
\begin{pgfscope}%
\pgfpathrectangle{\pgfqpoint{0.000000in}{0.000000in}}{\pgfqpoint{5.963393in}{4.543538in}}%
\pgfusepath{clip}%
\pgfsetbuttcap%
\pgfsetmiterjoin%
\definecolor{currentfill}{rgb}{1.000000,1.000000,1.000000}%
\pgfsetfillcolor{currentfill}%
\pgfsetlinewidth{1.003750pt}%
\definecolor{currentstroke}{rgb}{0.000000,0.000000,0.000000}%
\pgfsetstrokecolor{currentstroke}%
\pgfsetdash{}{0pt}%
\pgfpathmoveto{\pgfqpoint{3.691624in}{1.703827in}}%
\pgfpathlineto{\pgfqpoint{3.691624in}{2.271769in}}%
\pgfpathlineto{\pgfqpoint{4.259566in}{2.271769in}}%
\pgfpathlineto{\pgfqpoint{4.259566in}{1.703827in}}%
\pgfpathlineto{\pgfqpoint{3.691624in}{1.703827in}}%
\pgfpathclose%
\pgfusepath{stroke,fill}%
\end{pgfscope}%
\begin{pgfscope}%
\pgfpathrectangle{\pgfqpoint{0.000000in}{0.000000in}}{\pgfqpoint{5.963393in}{4.543538in}}%
\pgfusepath{clip}%
\pgfsetbuttcap%
\pgfsetmiterjoin%
\definecolor{currentfill}{rgb}{1.000000,1.000000,1.000000}%
\pgfsetfillcolor{currentfill}%
\pgfsetlinewidth{1.003750pt}%
\definecolor{currentstroke}{rgb}{0.000000,0.000000,0.000000}%
\pgfsetstrokecolor{currentstroke}%
\pgfsetdash{}{0pt}%
\pgfpathmoveto{\pgfqpoint{3.691624in}{2.271769in}}%
\pgfpathlineto{\pgfqpoint{3.691624in}{2.839711in}}%
\pgfpathlineto{\pgfqpoint{4.259566in}{2.839711in}}%
\pgfpathlineto{\pgfqpoint{4.259566in}{2.271769in}}%
\pgfpathlineto{\pgfqpoint{3.691624in}{2.271769in}}%
\pgfpathclose%
\pgfusepath{stroke,fill}%
\end{pgfscope}%
\begin{pgfscope}%
\pgfpathrectangle{\pgfqpoint{0.000000in}{0.000000in}}{\pgfqpoint{5.963393in}{4.543538in}}%
\pgfusepath{clip}%
\pgfsetbuttcap%
\pgfsetmiterjoin%
\definecolor{currentfill}{rgb}{1.000000,1.000000,1.000000}%
\pgfsetfillcolor{currentfill}%
\pgfsetlinewidth{1.003750pt}%
\definecolor{currentstroke}{rgb}{0.000000,0.000000,0.000000}%
\pgfsetstrokecolor{currentstroke}%
\pgfsetdash{}{0pt}%
\pgfpathmoveto{\pgfqpoint{3.691624in}{2.839711in}}%
\pgfpathlineto{\pgfqpoint{3.691624in}{3.407653in}}%
\pgfpathlineto{\pgfqpoint{4.259566in}{3.407653in}}%
\pgfpathlineto{\pgfqpoint{4.259566in}{2.839711in}}%
\pgfpathlineto{\pgfqpoint{3.691624in}{2.839711in}}%
\pgfpathclose%
\pgfusepath{stroke,fill}%
\end{pgfscope}%
\begin{pgfscope}%
\pgfpathrectangle{\pgfqpoint{0.000000in}{0.000000in}}{\pgfqpoint{5.963393in}{4.543538in}}%
\pgfusepath{clip}%
\pgfsetbuttcap%
\pgfsetmiterjoin%
\definecolor{currentfill}{rgb}{1.000000,1.000000,1.000000}%
\pgfsetfillcolor{currentfill}%
\pgfsetlinewidth{1.003750pt}%
\definecolor{currentstroke}{rgb}{0.000000,0.000000,0.000000}%
\pgfsetstrokecolor{currentstroke}%
\pgfsetdash{}{0pt}%
\pgfpathmoveto{\pgfqpoint{3.691624in}{3.407653in}}%
\pgfpathlineto{\pgfqpoint{3.691624in}{3.975595in}}%
\pgfpathlineto{\pgfqpoint{4.259566in}{3.975595in}}%
\pgfpathlineto{\pgfqpoint{4.259566in}{3.407653in}}%
\pgfpathlineto{\pgfqpoint{3.691624in}{3.407653in}}%
\pgfpathclose%
\pgfusepath{stroke,fill}%
\end{pgfscope}%
\begin{pgfscope}%
\pgfpathrectangle{\pgfqpoint{0.000000in}{0.000000in}}{\pgfqpoint{5.963393in}{4.543538in}}%
\pgfusepath{clip}%
\pgfsetbuttcap%
\pgfsetmiterjoin%
\definecolor{currentfill}{rgb}{1.000000,1.000000,1.000000}%
\pgfsetfillcolor{currentfill}%
\pgfsetlinewidth{1.003750pt}%
\definecolor{currentstroke}{rgb}{0.000000,0.000000,0.000000}%
\pgfsetstrokecolor{currentstroke}%
\pgfsetdash{}{0pt}%
\pgfpathmoveto{\pgfqpoint{4.259566in}{0.567942in}}%
\pgfpathlineto{\pgfqpoint{4.259566in}{1.135884in}}%
\pgfpathlineto{\pgfqpoint{4.827509in}{1.135884in}}%
\pgfpathlineto{\pgfqpoint{4.827509in}{0.567942in}}%
\pgfpathlineto{\pgfqpoint{4.259566in}{0.567942in}}%
\pgfpathclose%
\pgfusepath{stroke,fill}%
\end{pgfscope}%
\begin{pgfscope}%
\pgfpathrectangle{\pgfqpoint{0.000000in}{0.000000in}}{\pgfqpoint{5.963393in}{4.543538in}}%
\pgfusepath{clip}%
\pgfsetbuttcap%
\pgfsetmiterjoin%
\definecolor{currentfill}{rgb}{1.000000,1.000000,1.000000}%
\pgfsetfillcolor{currentfill}%
\pgfsetlinewidth{1.003750pt}%
\definecolor{currentstroke}{rgb}{0.000000,0.000000,0.000000}%
\pgfsetstrokecolor{currentstroke}%
\pgfsetdash{}{0pt}%
\pgfpathmoveto{\pgfqpoint{4.259566in}{1.135884in}}%
\pgfpathlineto{\pgfqpoint{4.259566in}{1.703827in}}%
\pgfpathlineto{\pgfqpoint{4.827509in}{1.703827in}}%
\pgfpathlineto{\pgfqpoint{4.827509in}{1.135884in}}%
\pgfpathlineto{\pgfqpoint{4.259566in}{1.135884in}}%
\pgfpathclose%
\pgfusepath{stroke,fill}%
\end{pgfscope}%
\begin{pgfscope}%
\pgfpathrectangle{\pgfqpoint{0.000000in}{0.000000in}}{\pgfqpoint{5.963393in}{4.543538in}}%
\pgfusepath{clip}%
\pgfsetbuttcap%
\pgfsetmiterjoin%
\definecolor{currentfill}{rgb}{1.000000,1.000000,1.000000}%
\pgfsetfillcolor{currentfill}%
\pgfsetlinewidth{1.003750pt}%
\definecolor{currentstroke}{rgb}{0.000000,0.000000,0.000000}%
\pgfsetstrokecolor{currentstroke}%
\pgfsetdash{}{0pt}%
\pgfpathmoveto{\pgfqpoint{4.259566in}{1.703827in}}%
\pgfpathlineto{\pgfqpoint{4.259566in}{2.271769in}}%
\pgfpathlineto{\pgfqpoint{4.827509in}{2.271769in}}%
\pgfpathlineto{\pgfqpoint{4.827509in}{1.703827in}}%
\pgfpathlineto{\pgfqpoint{4.259566in}{1.703827in}}%
\pgfpathclose%
\pgfusepath{stroke,fill}%
\end{pgfscope}%
\begin{pgfscope}%
\pgfpathrectangle{\pgfqpoint{0.000000in}{0.000000in}}{\pgfqpoint{5.963393in}{4.543538in}}%
\pgfusepath{clip}%
\pgfsetbuttcap%
\pgfsetmiterjoin%
\definecolor{currentfill}{rgb}{1.000000,1.000000,1.000000}%
\pgfsetfillcolor{currentfill}%
\pgfsetlinewidth{1.003750pt}%
\definecolor{currentstroke}{rgb}{0.000000,0.000000,0.000000}%
\pgfsetstrokecolor{currentstroke}%
\pgfsetdash{}{0pt}%
\pgfpathmoveto{\pgfqpoint{4.259566in}{2.271769in}}%
\pgfpathlineto{\pgfqpoint{4.259566in}{2.839711in}}%
\pgfpathlineto{\pgfqpoint{4.827509in}{2.839711in}}%
\pgfpathlineto{\pgfqpoint{4.827509in}{2.271769in}}%
\pgfpathlineto{\pgfqpoint{4.259566in}{2.271769in}}%
\pgfpathclose%
\pgfusepath{stroke,fill}%
\end{pgfscope}%
\begin{pgfscope}%
\pgfpathrectangle{\pgfqpoint{0.000000in}{0.000000in}}{\pgfqpoint{5.963393in}{4.543538in}}%
\pgfusepath{clip}%
\pgfsetbuttcap%
\pgfsetmiterjoin%
\definecolor{currentfill}{rgb}{1.000000,1.000000,1.000000}%
\pgfsetfillcolor{currentfill}%
\pgfsetlinewidth{1.003750pt}%
\definecolor{currentstroke}{rgb}{0.000000,0.000000,0.000000}%
\pgfsetstrokecolor{currentstroke}%
\pgfsetdash{}{0pt}%
\pgfpathmoveto{\pgfqpoint{4.259566in}{2.839711in}}%
\pgfpathlineto{\pgfqpoint{4.259566in}{3.407653in}}%
\pgfpathlineto{\pgfqpoint{4.827509in}{3.407653in}}%
\pgfpathlineto{\pgfqpoint{4.827509in}{2.839711in}}%
\pgfpathlineto{\pgfqpoint{4.259566in}{2.839711in}}%
\pgfpathclose%
\pgfusepath{stroke,fill}%
\end{pgfscope}%
\begin{pgfscope}%
\pgfpathrectangle{\pgfqpoint{0.000000in}{0.000000in}}{\pgfqpoint{5.963393in}{4.543538in}}%
\pgfusepath{clip}%
\pgfsetbuttcap%
\pgfsetmiterjoin%
\definecolor{currentfill}{rgb}{1.000000,1.000000,1.000000}%
\pgfsetfillcolor{currentfill}%
\pgfsetlinewidth{1.003750pt}%
\definecolor{currentstroke}{rgb}{0.000000,0.000000,0.000000}%
\pgfsetstrokecolor{currentstroke}%
\pgfsetdash{}{0pt}%
\pgfpathmoveto{\pgfqpoint{4.259566in}{3.407653in}}%
\pgfpathlineto{\pgfqpoint{4.259566in}{3.975595in}}%
\pgfpathlineto{\pgfqpoint{4.827509in}{3.975595in}}%
\pgfpathlineto{\pgfqpoint{4.827509in}{3.407653in}}%
\pgfpathlineto{\pgfqpoint{4.259566in}{3.407653in}}%
\pgfpathclose%
\pgfusepath{stroke,fill}%
\end{pgfscope}%
\begin{pgfscope}%
\pgfpathrectangle{\pgfqpoint{0.000000in}{0.000000in}}{\pgfqpoint{5.963393in}{4.543538in}}%
\pgfusepath{clip}%
\pgfsetroundcap%
\pgfsetmiterjoin%
\definecolor{currentfill}{rgb}{0.121569,0.466667,0.705882}%
\pgfsetfillcolor{currentfill}%
\pgfsetlinewidth{3.011250pt}%
\definecolor{currentstroke}{rgb}{0.274510,0.509804,0.705882}%
\pgfsetstrokecolor{currentstroke}%
\pgfsetdash{}{0pt}%
\pgfpathmoveto{\pgfqpoint{5.395451in}{0.283971in}}%
\pgfpathlineto{\pgfqpoint{5.395451in}{4.259566in}}%
\pgfpathmoveto{\pgfqpoint{5.395451in}{0.283971in}}%
\pgfpathlineto{\pgfqpoint{5.281862in}{0.283971in}}%
\pgfpathmoveto{\pgfqpoint{5.395451in}{0.851913in}}%
\pgfpathlineto{\pgfqpoint{5.281862in}{0.851913in}}%
\pgfpathmoveto{\pgfqpoint{5.395451in}{1.419855in}}%
\pgfpathlineto{\pgfqpoint{5.281862in}{1.419855in}}%
\pgfpathmoveto{\pgfqpoint{5.395451in}{1.987798in}}%
\pgfpathlineto{\pgfqpoint{5.281862in}{1.987798in}}%
\pgfpathmoveto{\pgfqpoint{5.395451in}{2.555740in}}%
\pgfpathlineto{\pgfqpoint{5.281862in}{2.555740in}}%
\pgfpathmoveto{\pgfqpoint{5.395451in}{3.123682in}}%
\pgfpathlineto{\pgfqpoint{5.281862in}{3.123682in}}%
\pgfpathmoveto{\pgfqpoint{5.395451in}{3.691624in}}%
\pgfpathlineto{\pgfqpoint{5.281862in}{3.691624in}}%
\pgfpathmoveto{\pgfqpoint{5.395451in}{4.259566in}}%
\pgfpathlineto{\pgfqpoint{5.281862in}{4.259566in}}%
\pgfusepath{stroke,fill}%
\end{pgfscope}%
\begin{pgfscope}%
\pgfpathrectangle{\pgfqpoint{0.000000in}{0.000000in}}{\pgfqpoint{5.963393in}{4.543538in}}%
\pgfusepath{clip}%
\pgfsetbuttcap%
\pgfsetmiterjoin%
\definecolor{currentfill}{rgb}{0.968627,0.462745,0.556863}%
\pgfsetfillcolor{currentfill}%
\pgfsetlinewidth{1.003750pt}%
\definecolor{currentstroke}{rgb}{0.968627,0.462745,0.556863}%
\pgfsetstrokecolor{currentstroke}%
\pgfsetdash{}{0pt}%
\pgfpathmoveto{\pgfqpoint{5.374345in}{4.017040in}}%
\pgfpathlineto{\pgfqpoint{5.267280in}{3.935302in}}%
\pgfpathlineto{\pgfqpoint{5.258070in}{3.962164in}}%
\pgfpathlineto{\pgfqpoint{0.288576in}{2.258338in}}%
\pgfpathlineto{\pgfqpoint{0.279366in}{2.285200in}}%
\pgfpathlineto{\pgfqpoint{5.248860in}{3.989026in}}%
\pgfpathlineto{\pgfqpoint{5.239650in}{4.015889in}}%
\pgfpathlineto{\pgfqpoint{5.374345in}{4.017040in}}%
\pgfpathclose%
\pgfusepath{stroke,fill}%
\end{pgfscope}%
\begin{pgfscope}%
\definecolor{textcolor}{rgb}{0.000000,0.000000,0.000000}%
\pgfsetstrokecolor{textcolor}%
\pgfsetfillcolor{textcolor}%
\pgftext[x=1.703827in,y=0.851913in,,]{\color{textcolor}\sffamily\fontsize{7.000000}{8.400000}\selectfont \(\displaystyle a_{1, 31}\)}%
\end{pgfscope}%
\begin{pgfscope}%
\definecolor{textcolor}{rgb}{0.000000,0.000000,0.000000}%
\pgfsetstrokecolor{textcolor}%
\pgfsetfillcolor{textcolor}%
\pgftext[x=1.703827in,y=1.419855in,,]{\color{textcolor}\sffamily\fontsize{7.000000}{8.400000}\selectfont \(\displaystyle a_{1, 25}\)}%
\end{pgfscope}%
\begin{pgfscope}%
\definecolor{textcolor}{rgb}{0.000000,0.000000,0.000000}%
\pgfsetstrokecolor{textcolor}%
\pgfsetfillcolor{textcolor}%
\pgftext[x=1.703827in,y=1.987798in,,]{\color{textcolor}\sffamily\fontsize{7.000000}{8.400000}\selectfont \(\displaystyle a_{1, 19}\)}%
\end{pgfscope}%
\begin{pgfscope}%
\definecolor{textcolor}{rgb}{0.000000,0.000000,0.000000}%
\pgfsetstrokecolor{textcolor}%
\pgfsetfillcolor{textcolor}%
\pgftext[x=1.703827in,y=2.555740in,,]{\color{textcolor}\sffamily\fontsize{7.000000}{8.400000}\selectfont \(\displaystyle a_{1, 13}\)}%
\end{pgfscope}%
\begin{pgfscope}%
\definecolor{textcolor}{rgb}{0.000000,0.000000,0.000000}%
\pgfsetstrokecolor{textcolor}%
\pgfsetfillcolor{textcolor}%
\pgftext[x=1.703827in,y=3.123682in,,]{\color{textcolor}\sffamily\fontsize{7.000000}{8.400000}\selectfont \(\displaystyle a_{1, 7}\)}%
\end{pgfscope}%
\begin{pgfscope}%
\definecolor{textcolor}{rgb}{0.000000,0.000000,0.000000}%
\pgfsetstrokecolor{textcolor}%
\pgfsetfillcolor{textcolor}%
\pgftext[x=1.703827in,y=3.691624in,,]{\color{textcolor}\sffamily\fontsize{7.000000}{8.400000}\selectfont \(\displaystyle a_{1, 1}\)}%
\end{pgfscope}%
\begin{pgfscope}%
\definecolor{textcolor}{rgb}{0.000000,0.000000,0.000000}%
\pgfsetstrokecolor{textcolor}%
\pgfsetfillcolor{textcolor}%
\pgftext[x=2.271769in,y=0.851913in,,]{\color{textcolor}\sffamily\fontsize{7.000000}{8.400000}\selectfont \(\displaystyle a_{1, 32}\)}%
\end{pgfscope}%
\begin{pgfscope}%
\definecolor{textcolor}{rgb}{0.000000,0.000000,0.000000}%
\pgfsetstrokecolor{textcolor}%
\pgfsetfillcolor{textcolor}%
\pgftext[x=2.271769in,y=1.419855in,,]{\color{textcolor}\sffamily\fontsize{7.000000}{8.400000}\selectfont \(\displaystyle a_{1, 26}\)}%
\end{pgfscope}%
\begin{pgfscope}%
\definecolor{textcolor}{rgb}{0.000000,0.000000,0.000000}%
\pgfsetstrokecolor{textcolor}%
\pgfsetfillcolor{textcolor}%
\pgftext[x=2.271769in,y=1.987798in,,]{\color{textcolor}\sffamily\fontsize{7.000000}{8.400000}\selectfont \(\displaystyle a_{1, 20}\)}%
\end{pgfscope}%
\begin{pgfscope}%
\definecolor{textcolor}{rgb}{0.000000,0.000000,0.000000}%
\pgfsetstrokecolor{textcolor}%
\pgfsetfillcolor{textcolor}%
\pgftext[x=2.271769in,y=2.555740in,,]{\color{textcolor}\sffamily\fontsize{7.000000}{8.400000}\selectfont \(\displaystyle a_{1, 14}\)}%
\end{pgfscope}%
\begin{pgfscope}%
\definecolor{textcolor}{rgb}{0.000000,0.000000,0.000000}%
\pgfsetstrokecolor{textcolor}%
\pgfsetfillcolor{textcolor}%
\pgftext[x=2.271769in,y=3.123682in,,]{\color{textcolor}\sffamily\fontsize{7.000000}{8.400000}\selectfont \(\displaystyle a_{1, 8}\)}%
\end{pgfscope}%
\begin{pgfscope}%
\definecolor{textcolor}{rgb}{0.000000,0.000000,0.000000}%
\pgfsetstrokecolor{textcolor}%
\pgfsetfillcolor{textcolor}%
\pgftext[x=2.271769in,y=3.691624in,,]{\color{textcolor}\sffamily\fontsize{7.000000}{8.400000}\selectfont \(\displaystyle a_{1, 2}\)}%
\end{pgfscope}%
\begin{pgfscope}%
\definecolor{textcolor}{rgb}{0.000000,0.000000,0.000000}%
\pgfsetstrokecolor{textcolor}%
\pgfsetfillcolor{textcolor}%
\pgftext[x=2.839711in,y=0.851913in,,]{\color{textcolor}\sffamily\fontsize{7.000000}{8.400000}\selectfont \(\displaystyle a_{1, 33}\)}%
\end{pgfscope}%
\begin{pgfscope}%
\definecolor{textcolor}{rgb}{0.000000,0.000000,0.000000}%
\pgfsetstrokecolor{textcolor}%
\pgfsetfillcolor{textcolor}%
\pgftext[x=2.839711in,y=1.419855in,,]{\color{textcolor}\sffamily\fontsize{7.000000}{8.400000}\selectfont \(\displaystyle a_{1, 27}\)}%
\end{pgfscope}%
\begin{pgfscope}%
\definecolor{textcolor}{rgb}{0.000000,0.000000,0.000000}%
\pgfsetstrokecolor{textcolor}%
\pgfsetfillcolor{textcolor}%
\pgftext[x=2.839711in,y=1.987798in,,]{\color{textcolor}\sffamily\fontsize{7.000000}{8.400000}\selectfont \(\displaystyle a_{1, 21}\)}%
\end{pgfscope}%
\begin{pgfscope}%
\definecolor{textcolor}{rgb}{0.000000,0.000000,0.000000}%
\pgfsetstrokecolor{textcolor}%
\pgfsetfillcolor{textcolor}%
\pgftext[x=2.839711in,y=2.555740in,,]{\color{textcolor}\sffamily\fontsize{7.000000}{8.400000}\selectfont \(\displaystyle a_{1, 15}\)}%
\end{pgfscope}%
\begin{pgfscope}%
\definecolor{textcolor}{rgb}{0.000000,0.000000,0.000000}%
\pgfsetstrokecolor{textcolor}%
\pgfsetfillcolor{textcolor}%
\pgftext[x=2.839711in,y=3.123682in,,]{\color{textcolor}\sffamily\fontsize{7.000000}{8.400000}\selectfont \(\displaystyle a_{1, 9}\)}%
\end{pgfscope}%
\begin{pgfscope}%
\definecolor{textcolor}{rgb}{0.000000,0.000000,0.000000}%
\pgfsetstrokecolor{textcolor}%
\pgfsetfillcolor{textcolor}%
\pgftext[x=2.839711in,y=3.691624in,,]{\color{textcolor}\sffamily\fontsize{7.000000}{8.400000}\selectfont \(\displaystyle a_{1, 3}\)}%
\end{pgfscope}%
\begin{pgfscope}%
\definecolor{textcolor}{rgb}{0.000000,0.000000,0.000000}%
\pgfsetstrokecolor{textcolor}%
\pgfsetfillcolor{textcolor}%
\pgftext[x=3.407653in,y=0.851913in,,]{\color{textcolor}\sffamily\fontsize{7.000000}{8.400000}\selectfont \(\displaystyle a_{1, 34}\)}%
\end{pgfscope}%
\begin{pgfscope}%
\definecolor{textcolor}{rgb}{0.000000,0.000000,0.000000}%
\pgfsetstrokecolor{textcolor}%
\pgfsetfillcolor{textcolor}%
\pgftext[x=3.407653in,y=1.419855in,,]{\color{textcolor}\sffamily\fontsize{7.000000}{8.400000}\selectfont \(\displaystyle a_{1, 28}\)}%
\end{pgfscope}%
\begin{pgfscope}%
\definecolor{textcolor}{rgb}{0.000000,0.000000,0.000000}%
\pgfsetstrokecolor{textcolor}%
\pgfsetfillcolor{textcolor}%
\pgftext[x=3.407653in,y=1.987798in,,]{\color{textcolor}\sffamily\fontsize{7.000000}{8.400000}\selectfont \(\displaystyle a_{1, 22}\)}%
\end{pgfscope}%
\begin{pgfscope}%
\definecolor{textcolor}{rgb}{0.000000,0.000000,0.000000}%
\pgfsetstrokecolor{textcolor}%
\pgfsetfillcolor{textcolor}%
\pgftext[x=3.407653in,y=2.555740in,,]{\color{textcolor}\sffamily\fontsize{7.000000}{8.400000}\selectfont \(\displaystyle a_{1, 16}\)}%
\end{pgfscope}%
\begin{pgfscope}%
\definecolor{textcolor}{rgb}{0.000000,0.000000,0.000000}%
\pgfsetstrokecolor{textcolor}%
\pgfsetfillcolor{textcolor}%
\pgftext[x=3.407653in,y=3.123682in,,]{\color{textcolor}\sffamily\fontsize{7.000000}{8.400000}\selectfont \(\displaystyle a_{1, 10}\)}%
\end{pgfscope}%
\begin{pgfscope}%
\definecolor{textcolor}{rgb}{0.000000,0.000000,0.000000}%
\pgfsetstrokecolor{textcolor}%
\pgfsetfillcolor{textcolor}%
\pgftext[x=3.407653in,y=3.691624in,,]{\color{textcolor}\sffamily\fontsize{7.000000}{8.400000}\selectfont \(\displaystyle a_{1, 4}\)}%
\end{pgfscope}%
\begin{pgfscope}%
\definecolor{textcolor}{rgb}{0.000000,0.000000,0.000000}%
\pgfsetstrokecolor{textcolor}%
\pgfsetfillcolor{textcolor}%
\pgftext[x=3.975595in,y=0.851913in,,]{\color{textcolor}\sffamily\fontsize{7.000000}{8.400000}\selectfont \(\displaystyle a_{1, 35}\)}%
\end{pgfscope}%
\begin{pgfscope}%
\definecolor{textcolor}{rgb}{0.000000,0.000000,0.000000}%
\pgfsetstrokecolor{textcolor}%
\pgfsetfillcolor{textcolor}%
\pgftext[x=3.975595in,y=1.419855in,,]{\color{textcolor}\sffamily\fontsize{7.000000}{8.400000}\selectfont \(\displaystyle a_{1, 29}\)}%
\end{pgfscope}%
\begin{pgfscope}%
\definecolor{textcolor}{rgb}{0.000000,0.000000,0.000000}%
\pgfsetstrokecolor{textcolor}%
\pgfsetfillcolor{textcolor}%
\pgftext[x=3.975595in,y=1.987798in,,]{\color{textcolor}\sffamily\fontsize{7.000000}{8.400000}\selectfont \(\displaystyle a_{1, 23}\)}%
\end{pgfscope}%
\begin{pgfscope}%
\definecolor{textcolor}{rgb}{0.000000,0.000000,0.000000}%
\pgfsetstrokecolor{textcolor}%
\pgfsetfillcolor{textcolor}%
\pgftext[x=3.975595in,y=2.555740in,,]{\color{textcolor}\sffamily\fontsize{7.000000}{8.400000}\selectfont \(\displaystyle a_{1, 17}\)}%
\end{pgfscope}%
\begin{pgfscope}%
\definecolor{textcolor}{rgb}{0.000000,0.000000,0.000000}%
\pgfsetstrokecolor{textcolor}%
\pgfsetfillcolor{textcolor}%
\pgftext[x=3.975595in,y=3.123682in,,]{\color{textcolor}\sffamily\fontsize{7.000000}{8.400000}\selectfont \(\displaystyle a_{1, 11}\)}%
\end{pgfscope}%
\begin{pgfscope}%
\definecolor{textcolor}{rgb}{0.000000,0.000000,0.000000}%
\pgfsetstrokecolor{textcolor}%
\pgfsetfillcolor{textcolor}%
\pgftext[x=3.975595in,y=3.691624in,,]{\color{textcolor}\sffamily\fontsize{7.000000}{8.400000}\selectfont \(\displaystyle a_{1, 5}\)}%
\end{pgfscope}%
\begin{pgfscope}%
\definecolor{textcolor}{rgb}{0.000000,0.000000,0.000000}%
\pgfsetstrokecolor{textcolor}%
\pgfsetfillcolor{textcolor}%
\pgftext[x=4.543538in,y=0.851913in,,]{\color{textcolor}\sffamily\fontsize{7.000000}{8.400000}\selectfont \(\displaystyle a_{1, 36}\)}%
\end{pgfscope}%
\begin{pgfscope}%
\definecolor{textcolor}{rgb}{0.000000,0.000000,0.000000}%
\pgfsetstrokecolor{textcolor}%
\pgfsetfillcolor{textcolor}%
\pgftext[x=4.543538in,y=1.419855in,,]{\color{textcolor}\sffamily\fontsize{7.000000}{8.400000}\selectfont \(\displaystyle a_{1, 30}\)}%
\end{pgfscope}%
\begin{pgfscope}%
\definecolor{textcolor}{rgb}{0.000000,0.000000,0.000000}%
\pgfsetstrokecolor{textcolor}%
\pgfsetfillcolor{textcolor}%
\pgftext[x=4.543538in,y=1.987798in,,]{\color{textcolor}\sffamily\fontsize{7.000000}{8.400000}\selectfont \(\displaystyle a_{1, 24}\)}%
\end{pgfscope}%
\begin{pgfscope}%
\definecolor{textcolor}{rgb}{0.000000,0.000000,0.000000}%
\pgfsetstrokecolor{textcolor}%
\pgfsetfillcolor{textcolor}%
\pgftext[x=4.543538in,y=2.555740in,,]{\color{textcolor}\sffamily\fontsize{7.000000}{8.400000}\selectfont \(\displaystyle a_{1, 18}\)}%
\end{pgfscope}%
\begin{pgfscope}%
\definecolor{textcolor}{rgb}{0.000000,0.000000,0.000000}%
\pgfsetstrokecolor{textcolor}%
\pgfsetfillcolor{textcolor}%
\pgftext[x=4.543538in,y=3.123682in,,]{\color{textcolor}\sffamily\fontsize{7.000000}{8.400000}\selectfont \(\displaystyle a_{1, 12}\)}%
\end{pgfscope}%
\begin{pgfscope}%
\definecolor{textcolor}{rgb}{0.000000,0.000000,0.000000}%
\pgfsetstrokecolor{textcolor}%
\pgfsetfillcolor{textcolor}%
\pgftext[x=4.543538in,y=3.691624in,,]{\color{textcolor}\sffamily\fontsize{7.000000}{8.400000}\selectfont \(\displaystyle a_{1, 6}\)}%
\end{pgfscope}%
\begin{pgfscope}%
\pgfpathrectangle{\pgfqpoint{0.000000in}{0.000000in}}{\pgfqpoint{5.963393in}{4.543538in}}%
\pgfusepath{clip}%
\pgfsetbuttcap%
\pgfsetroundjoin%
\definecolor{currentfill}{rgb}{0.000000,0.000000,0.000000}%
\pgfsetfillcolor{currentfill}%
\pgfsetlinewidth{1.003750pt}%
\definecolor{currentstroke}{rgb}{0.000000,0.000000,0.000000}%
\pgfsetstrokecolor{currentstroke}%
\pgfsetdash{}{0pt}%
\pgfsys@defobject{currentmarker}{\pgfqpoint{-0.020833in}{-0.020833in}}{\pgfqpoint{0.020833in}{0.020833in}}{%
\pgfpathmoveto{\pgfqpoint{0.000000in}{-0.020833in}}%
\pgfpathcurveto{\pgfqpoint{0.005525in}{-0.020833in}}{\pgfqpoint{0.010825in}{-0.018638in}}{\pgfqpoint{0.014731in}{-0.014731in}}%
\pgfpathcurveto{\pgfqpoint{0.018638in}{-0.010825in}}{\pgfqpoint{0.020833in}{-0.005525in}}{\pgfqpoint{0.020833in}{0.000000in}}%
\pgfpathcurveto{\pgfqpoint{0.020833in}{0.005525in}}{\pgfqpoint{0.018638in}{0.010825in}}{\pgfqpoint{0.014731in}{0.014731in}}%
\pgfpathcurveto{\pgfqpoint{0.010825in}{0.018638in}}{\pgfqpoint{0.005525in}{0.020833in}}{\pgfqpoint{0.000000in}{0.020833in}}%
\pgfpathcurveto{\pgfqpoint{-0.005525in}{0.020833in}}{\pgfqpoint{-0.010825in}{0.018638in}}{\pgfqpoint{-0.014731in}{0.014731in}}%
\pgfpathcurveto{\pgfqpoint{-0.018638in}{0.010825in}}{\pgfqpoint{-0.020833in}{0.005525in}}{\pgfqpoint{-0.020833in}{0.000000in}}%
\pgfpathcurveto{\pgfqpoint{-0.020833in}{-0.005525in}}{\pgfqpoint{-0.018638in}{-0.010825in}}{\pgfqpoint{-0.014731in}{-0.014731in}}%
\pgfpathcurveto{\pgfqpoint{-0.010825in}{-0.018638in}}{\pgfqpoint{-0.005525in}{-0.020833in}}{\pgfqpoint{0.000000in}{-0.020833in}}%
\pgfpathlineto{\pgfqpoint{0.000000in}{-0.020833in}}%
\pgfpathclose%
\pgfusepath{stroke,fill}%
}%
\begin{pgfscope}%
\pgfsys@transformshift{0.283971in}{2.271769in}%
\pgfsys@useobject{currentmarker}{}%
\end{pgfscope}%
\end{pgfscope}%
\begin{pgfscope}%
\definecolor{textcolor}{rgb}{0.000000,0.000000,0.000000}%
\pgfsetstrokecolor{textcolor}%
\pgfsetfillcolor{textcolor}%
\pgftext[x=0.283971in,y=2.158180in,,]{\color{textcolor}\sffamily\fontsize{9.000000}{10.800000}\selectfont Source}%
\end{pgfscope}%
\begin{pgfscope}%
\definecolor{textcolor}{rgb}{0.000000,0.000000,0.000000}%
\pgfsetstrokecolor{textcolor}%
\pgfsetfillcolor{textcolor}%
\pgftext[x=5.452245in,y=0.567942in,left,base]{\color{textcolor}\sffamily\fontsize{9.000000}{10.800000}\selectfont \(\displaystyle m_7\)}%
\end{pgfscope}%
\begin{pgfscope}%
\definecolor{textcolor}{rgb}{0.000000,0.000000,0.000000}%
\pgfsetstrokecolor{textcolor}%
\pgfsetfillcolor{textcolor}%
\pgftext[x=5.452245in,y=1.135884in,left,base]{\color{textcolor}\sffamily\fontsize{9.000000}{10.800000}\selectfont \(\displaystyle m_6\)}%
\end{pgfscope}%
\begin{pgfscope}%
\definecolor{textcolor}{rgb}{0.000000,0.000000,0.000000}%
\pgfsetstrokecolor{textcolor}%
\pgfsetfillcolor{textcolor}%
\pgftext[x=5.452245in,y=1.703827in,left,base]{\color{textcolor}\sffamily\fontsize{9.000000}{10.800000}\selectfont \(\displaystyle m_5\)}%
\end{pgfscope}%
\begin{pgfscope}%
\definecolor{textcolor}{rgb}{0.000000,0.000000,0.000000}%
\pgfsetstrokecolor{textcolor}%
\pgfsetfillcolor{textcolor}%
\pgftext[x=5.452245in,y=2.271769in,left,base]{\color{textcolor}\sffamily\fontsize{9.000000}{10.800000}\selectfont \(\displaystyle m_4\)}%
\end{pgfscope}%
\begin{pgfscope}%
\definecolor{textcolor}{rgb}{0.000000,0.000000,0.000000}%
\pgfsetstrokecolor{textcolor}%
\pgfsetfillcolor{textcolor}%
\pgftext[x=5.452245in,y=2.839711in,left,base]{\color{textcolor}\sffamily\fontsize{9.000000}{10.800000}\selectfont \(\displaystyle m_3\)}%
\end{pgfscope}%
\begin{pgfscope}%
\definecolor{textcolor}{rgb}{0.000000,0.000000,0.000000}%
\pgfsetstrokecolor{textcolor}%
\pgfsetfillcolor{textcolor}%
\pgftext[x=5.452245in,y=3.407653in,left,base]{\color{textcolor}\sffamily\fontsize{9.000000}{10.800000}\selectfont \(\displaystyle m_2\)}%
\end{pgfscope}%
\begin{pgfscope}%
\definecolor{textcolor}{rgb}{0.000000,0.000000,0.000000}%
\pgfsetstrokecolor{textcolor}%
\pgfsetfillcolor{textcolor}%
\pgftext[x=5.452245in,y=3.975595in,left,base]{\color{textcolor}\sffamily\fontsize{9.000000}{10.800000}\selectfont \(\displaystyle m_1\)}%
\end{pgfscope}%
\end{pgfpicture}%
\makeatother%
\endgroup%

	\caption{Visualization of weights of a single row of the system matrix. \inlinetodo{Find
			better explanation, this should be standalone}}\label{fig:matrix_row}
\end{figure}

\section{Iterative Reconstruction}\label{sec:iterative_reconstruction}

As explained in \autoref{chap:image_representation}, the problems considered in this thesis are
ill-posed. Hence, care has to be taken when solving the linear system of equations in
\autoref{eq:system_lin_equation}. A common approach is to considered least squares problem instead.

\begin{definition}[Least Squared Problem]\label{def:least_squares_problem}
	The least squares problem is defined as
	\[ \argmin_c \frac{1}{2} \norm{Ac - m}^2_2 \]
	The solution to the least squares problem is given by the normal equation
	\[ A^T A c = A^T m \]
\end{definition}

Note here, that \(Ac\) is considered the forward projection and \(A^T m\) is the backward
projection.

However, the system matrix is usually too large to store in system memory. Therefore, algorithms are
necessary, which do not require the knowledge of the complete system matrix. The software computing
the system matrix on the fly, is often referred to as projectors. A deep dive into the
implementation will be conduced in \autoref{chap:projector}.

\subsection{Landweber Iteration}\label{subsec:landweber_iteration}

A well studied class of iterative algorithms is the \textit{landweber
	iteration}~\cite{landweber_iteration_1951}. It has been discovered in many different ways in
the past. The algorithm was introduced in the tomographic space by
\citeauthor{gilbert_iterative_1972}~\cite{gilbert_iterative_1972} under the name of
\gls{SIRT}.

\begin{definition}[Landweber Iteration]\label{def:landweber_iteration}
	Given a linear system of equations as defined in \autoref{def:inverse_problem}, the
	\textit{Landweber iteration} finds a solution to the corresponding least squares problem. The update
	step for \(k = 0, 1, \dots\) is given by
	\[
		c^{(k+1)} = c^{(k)} + \lambda^{(k)} A^T(m - Ac^{(k)})
	\]
	\(\lambda^{(k)} \in \mathbb{R}\) is a sequence of relaxation parameter that must satisfy
	\(0 < \lambda^{(k)} < 2 \norm{A^T A}_2^{-1}\quad \forall k \in \mathbb{N}\).
\end{definition}

Generally, Landweber iterations (in the basic case) only rely on the forward \(Ac^{(k)}\) and backward
\(A^T m\) projections. The innermost part of the update function, is a forward projection of the
current guess. Next, the residual to the measurement is taken and finally the error is back
projected and used as an update to the current guess.

The basic Landweber iteration is a special case of gradient descent. If \(f(c) = \frac{1}{2}
\norm{Ac - m}_2^2\), the update can be written in terms of the gradient
\[
	c^{(k+1)} = c^{(k)} - \lambda^{(k)} \nabla f(c^{(k)})
\]
A generalisation of the Landweber iteration can be given by
\[
	c^{(k+1)} = c^{(k)} + \lambda^{(k)} DA^TM(m - Ac^{(k)})
\]
The expat algorithm and it's convergence behavior, depend on the exact choice of the matrices \(D\)
and \(M\). The basic Landweber iteration as presented above has \(D = M = I\). If \(D = \frac{1}{J}
\text{diag}(\norm{a_j}^2_2)^{-1}\), with \(\norm{a_j}^2_2\) being the squared \(L_2\) norm of the
\(j\)th row of the system matrix~\cite[Chapter~6.2]{hansen_discrete_2010}.

\subsection{Algebraic Reconstruction Technique}\label{subsec:algebraic_reconstruction_technique}

The \gls{ART} was proposed by \citeauthor{gordon_algebraic_1970}\cite{gordon_algebraic_1970}.
However, outside of tomography the method is often knows as Kaczmarz
method~\cite{kaczmarz_approximate_1993}. Though, \gls{ART} is a slight modification of the original
Kaczmarz method.

The basic idea of the algorithm, is to view each row of the system matrix as a hyperplane and update
the solution by iteratively project it onto the hyperplane. If the system matrix is square \(I = J\)
and of full rank, all hyperplanes intersect at one point and \gls{ART} will converge to it. However, if
the system is overdetermined (\(J > I\)) and noisy, which it usually is, the hyperplanes will not
intersect at a single point, but rather at in close proximity to each other.

\begin{definition}[Algebraic Reconstruction Technique]\label{def:art}
	Given the system of linear equations \(Ac = m\), and an initial solution guess \(c^{(0)} \in
	\mathbb{R}^I\) (often the zero vector). Then the solution can be iteratively updated for
	\(k = 0, 1, \dots\)
	\[
		c^{(k+1)} = c^{(k)} + \lambda^{j(k)} \frac{m_{j(k)} - \langle a_{j(k)}, c^{j(k)} \rangle}{\norm{a_{j(k)}}}a_{j(k)}
	\]
	where \(\lambda^{(k)} \in \left(0, 1\right]\) is a sequence of relaxation parameters, and
	\(j(k)\) is a mapping to select an appropriate row for each iteration. In the simple case
	\(j(k) = (k \mod J) + 1\), but it can also be randomized~\cite{strohmer_randomized_2007}.
\end{definition}

The original Kaczmarz method had \(\lambda(k) = 1\, \forall k \in \mathbb{N}\), lowering the
relaxation parameter can improve the reconstruction in noisy settings. Further, the method can be
written as a Landweber-type method~\cite{hansen_discrete_2010}, but it's rather uncommon.

Compared to the Landweber iterations given in the previous section, the Kaczmarz methods accesses
the rows of the system matrix \(A\) sequentially (but maybe not in ascending order). On the other
hand Landweber type methods access all rows of the system matrix simultaneously. Hence, the
`simultaneous' as part of \gls{SIRT}\@.

\inlinetodo{add figure for this}

% \subsection{CG}\label{subsec:conjuage_gradient}
%
% CG and such
%
% \subsection{First-order methods}\label{subsec:first_order_methods}
%
% First other methods such as Gradient Descent and it's derivatives

\section{Regularization}\label{sec:regularization}

So far all solvers presented here have only looked at the least squares problem. I.e.\ they only
concern themselves with the forward model and do not incorporate any further constrains. However,
usually we have some information about the images we wish to reconstruct. For example, one would
expect them to be smooth. One hopes that regularization stabilizes the solution. In the sense that
small perturbations by noise in the measurements, still yields solutions close to the exact
solutions.

\begin{definition}[Regularized Problem]\label{def:regularized_problem}
	Let \(R(c)\) be a \textit{penalty function} or \textit{regularizer}, then the least square
	problem can be expanded to
	\[
		\argmin_c \frac{1}{2} \norm{A c - m}_2^2 + \lambda R(c)
	\]
	this is referred to as \textit{regularized problem}. \(lambda\) is a regularization
	parameter. It denotes the weight of the penalty term.
\end{definition}

This can also be generalized to other problems. Let \(T(c)\) be a data fidelity term (e.g.\
the least squares one, or the negative log-likelihood). Then the regularized problem can be
described as
\[ \argmin_c T(c) + \lambda R(c) \]
Usually, \(R\) is chosen to be non-linear and often is expected to be continuously differentiable.
This equation has three parts. The first \(T(c)\) is the data term. It measures how well a
prediction models the noisy data. However, we do not want to fit the noise in the data. This is the
second, the regularization term. \(\lambda\) controls the importance or the balance of each the
previous terms.

\subsection{Tikhonov Regularization}\label{subsec:tikhonov_regularization}

A well studied and often used regularization is named after Andrey Nikolayevich
Tikhonov~\cite{tihonov_solution_1963}. The penalty restricts the solution based on the Euclidean
norm.

\begin{definition}[Tikhonov Regularization]\label{def:tikhonov_regularization}
	The penalty term for Tikhonov regularization is given by
	\[
		R(c)_{\text{Tikhonov}} = \norm{\Gamma c}_2^2
	\]
\end{definition}
A common case is a simple scaling function i.e.\ \(\Gamma = \alpha I\), thus the Tikhonov
regularizer penalizes \(c\) with large \(L_2\) norm. Therefore, it is also referred to as
\(L_2\)-regularization. But also the first and second derivative operator is commonly
used~\cite{golub_tikhonov_1999} The hope is, that Tikhonov regularization suppresses high-frequency
noise.

On a further note, the Tikhonov regularization can be reformulated to a least squares problem again
\[
	\argmin_c \frac{1}{2}
	\left\lVert
	\begin{pmatrix}
		A \\
		\lambda \Gamma
	\end{pmatrix}
	c -
	\begin{pmatrix}
		m \\
		0
	\end{pmatrix}
	\right\rVert_2^2
\]
Here \(\begin{pmatrix}
	A \\
	\lambda \Gamma
\end{pmatrix}\) is a stacked matrix, with the system matrix on top, and the Tikhonov matrix in the
bottom. A different commonly used notation is
\[
	(A^T A + \lambda \Gamma^T \Gamma)x = A^T m
\]
The problem remains linear and thus can be solved using the previously discussed iterative
reconstruction algorithms. However, in general, for non Tikhonov regularization this does not hold
true. Thus, the problem is rendered non-linear and different optimization techniques have to be
used.

\subsection{\(L_1\)-Regularization}\label{subsec:l1_regularization}

Another common regularization method is based on the \(L_1\) norm, i.e.\ the sum of absolute
values.
\begin{definition}[\(L_1\)-Regularization]\label{def:l1_regularization}
	The penalty term for Tikhonov regularization is given by
	\[
		R(c)_{L_1} = \norm{c}_1
	\]
	See~\cite{tibshirani_regression_1996,tibshirani_lasso_2013,beck_fast_2009}
\end{definition}
Compared to the \(L_2\) regularization, the \(L_1\) regularization enforces sparsity. I.e.\ the
assumption is that the representation is in some way sparse, and should be enforced. Also, it is
more robust to outliers~\cite{beck_fast_2009}. For information on how these problems can be solved
see \citeauthor{beck_fast_2009}~\cite{beck_fast_2009}. As it will be used in the experimental
sections, specifically \gls{ISTA} and  \gls{FISTA}.

\begin{definition}[ISTA]\label{def:ista}
	The update step for \gls{ISTA} is given by
	\[
		c^{(k+1)} = \mathscr{T_\alpha} (c^{(k)} - 2 \lambda A^T (A c^{(k)} - m))
	\]
	where \(t\) is an appropriate step size and \(\mathscr{T_\alpha}\) is the shrinkage operator
	defined by
	\[
		\mathscr{T_\alpha}(c)_j = \max(\abs{c_i} - \alpha, 0) \sign(c_j)
	\]
\end{definition}
Similar to Landweber like methods, the residual of the forward projection current prediction and the
measurement vector is back projected. Next, the projected error is subtracted from the current
estimate and then the shrinkage operator is applied.

However, the convergence of \gls{ISTA} is rather slow (compare~\cite{beck_fast_2009} and its
references). \gls{FISTA} improves on the convergence of \gls{ISTA}, but it is out of scope for this
thesis. Please refer to \citeauthor{beck_fast_2009}~\cite{beck_fast_2009} for further reading.

% \subsection{TV Regularization}\label{subsec:tv_regularization}
%
% \begin{listing}
% 	\begin{minted}{cpp}
% int main() {
%     fmt::print("hello, world\n");
%     return 0;
% }
%     \end{minted}
% 	\caption{"Some sampe C code"}
% \end{listing}
% \begin{listing}
% 	\begin{minted}{python}
% import numpy as np
%
% np.linspace(0, 1)
%     \end{minted}
% 	\caption{"Some sampe python code"}
% \end{listing}


%%%%%%%%%%%%%%%%%%%%%%%%%%%%%%%%%%%%%%%%%%%%%%%%%%%%%%%%%%%%%%%%%%%%%%%%
% Implementation
%%%%%%%%%%%%%%%%%%%%%%%%%%%%%%%%%%%%%%%%%%%%%%%%%%%%%%%%%%%%%%%%%%%%%%%%
\part[Practical]{%
	Practical\\
	%
	% \vspace{1cm}
	% %
	% \begin{minipage}[l]{\textwidth}
	% 	%
	% 	\textnormal{%
	% 		\normalsize
	% 		%
	% 		\begin{singlespace*}
	% 			\onehalfspacing
	% 			%
	% 			After all the theory, let's focus on practical notions of previously stated
	% 			theory. Parts, which can be a little messier than the theory makes us believe.
	% 		\end{singlespace*}
	% 	}
	% \end{minipage}
}\label{part:practical}

\chapter{elsa}\label{chap:elsa}

I want to dedicate a chapter to the C++ framework elsa~\cite{lasser_elsa_2019} developed at the
\gls{CIIP} research group. As the practical part of my thesis is implemented for elsa, I feel that a
short introduction is more than appropriate.

elsa is a modern, flexible and open-source framework for tomographic reconstruction. I consider elsa
to follow for three important principles. First, and possibly most importantly for the user side, it
should be close the either mathematical notation, or at least mathematical concepts, second, it
should be useful for the research done at the \gls{CIIP} research group and finally, it should be as
easy as possible for other to reproduce results from the research. These different aspects have to
be wagered against each other, as they might stand in conflict to each other.

elsa is able to handle general optimization problems similar to the form given in
\autoref{eq:optimization_problem}. However, it can handle a little more general problems of the form
\begin{equation}\label{eq:complex_optimization_problem}
	\argmin_{\mvec{c}} \quad T(A, \mvec{c}, \mvec{m}) + \sum_{k=1}^K \lambda_k R_k(\mvec{c})
\end{equation}
where \(T\) is the data fidelity term depending on the forward model \(A\), the solution
\(\mvec{c}\), the measurements \(\mvec{m}\), \(K\) penalty or regularization terms depending on
\(\mvec{c}\) and \(\lambda_k\) is a regularization parameter for the \(k\)th penalty term (compare
\autoref{def:regularized_problem} with a single penalty term).

Building the framework around this general problem statement enables the implementation of multiple
imaging modalities in one framework. Currently, only X-ray attenuation CT is implemented. However,
from a theoretical standpoint other modalities such as phase-contrast CT and (anisotropic) dark-field
CT can be incorporated. Plus many aspects of the framework are independent of the specific
application, and reusable. I.e.\ one needs no different implementation of algorithms solving the
above given equations such as \gls{ART}, or \gls{ISTA}\@.

This is a key difference between other frameworks such as the ASTRA
Toolbox~\cite{van_aarle_fast_2016}, TIGRE~\cite{biguri_tigre_2016} or
FreeCT~\cite{hoffman_technical_2016}. All of these frameworks are known and widely used frameworks in
the field of tomographic reconstruction. Also, there are other frameworks supporting multiple
imaging modalities. However, their goal is not such a general approach. This generality is key for
the research at the \gls{CIIP} research group.

From a software engineering standpoint, a lot of effort is put into reproducibility, maintenance and
correctness, by following as numerous best practices as we can and see fit. This includes
having an extensive unit testing suite (though, we can still improve there), a merge request based
workflow, which heavily relies on continuous integration, and as much as possible, we follow modern
C++ guidelines and rely on modern features from C++17. Further, to ease prototyping and improve
accessibility Python bindings are provided as well.

The general architecture is build in a couple of layers. At the bottom, all data handling is done
by the \mintinline{cpp}{DataContainer} class. It is the one type to store and handle all
\(n\)-dimensional data in the framework. It handles storage for data accessible by the \gls{CPU} and
the \gls{GPU} and transfers between those two. Further, it leverages expression templates to reduce
the general memory footprint of applications. Above that, the mathematical concepts are implemented.
This includes concepts for residuals, functionals (such as \(l_1\), \(l_2\), or the Huber norm) and
linear operators (modeling both the forward models and operators such as scaling or finite
differences). Finally, it also includes an abstraction for the different algorithms solving
optimization problems.

Currently, two different forward models are implemented for X-ray attenuation CT. Both have a
\gls{CPU} and \gls{GPU} (specifically NVidia's CUDA) version available, as the forward and backward
projection are usually the computationally most expensive. The specific implemented methods are
referred to as Siddon's and Joseph's method, see the next \autoref{chap:projector} for details.

I will refer the interested reader to
\href{https://gitlab.lrz.de/IP/elsa/}{https://gitlab.lrz.de/IP/elsa/}. There, one can find guides to
install and use elsa extensively, plus examples.

\chapter{Projector}\label{chap:projector}

As shown in previous chapters, calculation of the system matrix coefficients is one of the key
components of tomographic reconstruction. Thus, this has been an important part of research. The
routines or algorithms, which calculate the matrix are frequently referred to as projectors. For the
forward projection, the contribution of each voxel to each detector pixel is calculated. For the
backward projection, the contribution of each detector pixel each a voxel is computed.

In the first part of this chapter, a detailed overview over existing research and implementations of
projectors is given. Next details on the projector used for this thesis is given. This is followed
by a detailed study of accuracy and performance of the new projector compared to other projection
methods.

\section{Types of Projectors}\label{sec:projector_types}

One of the first mentions of projector routines can be found in the RECLBL library
package~\cite{huesman_reclbl_1977}. Namely, the two types of projectors introduced there are the
\textit{voxel-driven} (or \textit{pixel-driven} for the 2D case) and the \textit{ray driven}. Many
state-of-the-art projectors are still based on the ideas of these projectors.

The voxel driven approach is very simple. For the forward projection, each voxel is visited, and the
voxel center is projected onto the detector. Finally, the contribution of the voxel to the detector,
is computed using some form of interpolation. In~\cite{peters_algorithms_1981} bilinear
interpolation between the two neighboring detector pixels is used.~\cite{harauz_interpolation_1983}
improves on the approach by using bi-cubic spline interpolation. However, the approach is used
rarely due to the introduced artifacts (C.f.~\cite[Chapter~3.3]{levakhina_three-dimensional_2014}).
I.e.\ if the resolution of the detector is finer compared to the volume, detector pixels might never
be assigned a value. The same operation can be performed for the back projection, however instead of
updating the value at the detector pixel, update the voxel value.

Instead of projecting the voxel center and interpolating, one can project the complete voxel onto
the projector plane. This approach was taken by~\cite{long_3d_2010, long_3d_2010-1} and is usually
referred to as separable footprint. They use trapezoidal functions to approximate the footprint both
accurately and efficiently. Hence, the contribution of voxels to the detector pixels is based on
these trapezoidal functions. It as also been ported to \gls{GPU} as shown in~\cite{wu_gpu_2011,
	xie_effective_2015, chapdelaine_new_2018}. To the best of my knowledge, for voxel based
approaches, this is the state-of-the-art approach and outperforms other approaches.

This approach was translated to B-Splines in~\cite{momey_b-spline_2012, momey_spline_2015}. There
B-Splines are assumed to be the basis function at pixel centers, and then the B-Splines are
projected onto the projector plane. As shown in \autoref{chap:signal_representation} about B-Spline
basis functions the projection of \(n\)-dimensional B-Splines yield a \(n-1\)-dimensional B-Spline,
hence the projection is simple and accurate.~\cite{momey_b-spline_2012} already incorporated both
parallel-beam and cone-beam geometry.

Similarly,~\cite{ziegler_efficient_2006} proposed a footprint approach for blobs. And it was
improved and ported to the \gls{GPU} by~\cite{bippus_projector_2011}.~\cite{kohler_iterative_2011}
describes a blob projector for phase-contrast CT\@.

A shared problem of the voxel-based approaches, is the challenge of parallel implementations. During
the forward projection, shared access to the projector pixels might be needed. Hence, mitigation
strategies must be developed.

A conceptually different approach compared to the voxel-based approach, is the \textit{ray-based}
approach. There, the key idea is to trace rays through the volume. Note that the forward projection
of this approach, is trivially parallelizable. Each ray can be traced independently through the
volume.

A classical ray-driven approach is presented in
\citeauthor*{siddon_fast_1985}~\cite{siddon_fast_1985}, hence often referred to as Siddon's method.
There, the exact path length of a ray traversing through a volume is calculated. I.e.\ the exact
calculation of the line integral of a single ray through the volume. This is also illustrated in
\autoref{fig:visualization_siddon_traversal}. More recent work like~\cite{jacobs_fast_1998,
	christiaens_fast_1999, zhao_fast_2004, gao_fast_2012} improved on the Siddon's method,
especially in efficiency and performance.~\cite{de_greef_accelerated_2009, xiao_efficient_2012}
ported Siddon's methods to the \gls{GPU}\@.

\begin{figure}
	\centering
	% Taken from https://tex.stackexchange.com/a/398309
	\includegraphics[width=0.75\textwidth]{./figures/projectors_siddon/siddon_traversal.png}
	\caption{Visualization of Siddon's method. Siddon's method exactly computes the intersection
		length of each voxel (text inside each voxel) with a ray (cardinal red line), which
		is used as a weight for the forward and backward projections. The intersection
		length for each voxel is given by the distance between the points where the ray
		enters and exists each voxel (marked as black
		dots).}\label{fig:visualization_siddon_traversal}
\end{figure}

Another classic is presented in~\cite{joseph_improved_1982}. It assumes a smooth image and
interpolates between neighboring voxels along the ray path. It does so by a slice-interpolation,
i.e.\ the voxels perpendicular to the main ray direction are considered.~\cite{graetz_high_2020}
proposed a branchless \gls{GPU} version of this approach.

The backward projections of ray-driven approaches typically trace the ray from a detector pixel to
the source and update all visited voxels based on the weights calculated as in the forward
direction. And in the of Joseph's based methods, also voxels close by are also updated. Note again
here, that parallelization is hindered by the possibility of shared write access to voxels.

The Siddon's and Joseph's method are classic and common approaches for the forward and backward
projection. However, they do suffer from certain artifacts. This is especially true for Siddon's
method. Due to the interpolation of the Joseph's method, less artifacts appear there. Such artifacts
can be seen in the experiment section of this thesis (\autoref{chap:experiments}).

Typically, iterative reconstruction algorithms expect that the forward and backward projectors are
the adjoint of each other, i.e.\ they are \textit{matched} projector
pair~\cite{zeng_unmatched_2000}. If the system matrix \(A\) is constructed using one projector, then
the backward projector should construct \(A^T\), i.e.\ the transpose of the matrix. If the abstract
view of linear bounded operators is taken, this would be the aforementioned adjoint.

Using an unmatched projector pair might be beneficial, as one projector algorithm might be slower
for the forward projection than the back projection or wise-versa. Hence, one might want to pick the
fastest for the forward projection and another for the back projection. However, iterative
reconstruction algorithms usually expect the backward projection to be the adjoint of the forward
projection. If this is not met, convergence might not be guaranteed.
\citeauthor*{zeng_unmatched_2000}~\cite{zeng_unmatched_2000} describes continuous which unmatched
detector pairs must meet.

An entirely different approach to both the voxel- and ray-driven methods is the
\textit{distance-driven} approach~\cite{de_man_distance-driven_2002, de_man_distance-driven_2004}.
There, the voxel boundaries and detector pixel boundaries are projected onto a common axis, then the
overlap is used as a weight. This approach is still state of the art, however, is suffers from
inaccuracies for projections, which are close to \(45^\circ\). A branchless \gls{GPU} version was
proposed in~\cite{liu_gpu-based_2017}. The distance-driven approach doesn't suffer from any
high-frequency artifacts.

Projectors based on blobs have been studied for quite some time. Ray-driven approaches are based on
algorithms presented in~\cite{matej_practical_1996, popescu_ray_2004}. But rather than assuming an
infinitely thin ray, they assume a beam. Hence, they are somewhat similar to Joseph's projectors,
that they have to visit neighboring voxels, as the support of blobs is larger than that of pixels.
~\cite{levakhina_distance-driven_2010} proposed a variant of the distance-driven approach based on
blobs.

Another area of projectors compute the intersection area of multiple rays. Such methods, as
presented in~\cite{ha_study_2015, ha_efficient_2016, ha_look-up_2018}, calculate the intersection
between rays directed a detector boundary.

\section{Implementation of Differentiable Projection Operators}\label{sec:implementation}

The goal of the practical part of the thesis, is an implementation of a projector based on
differential basis functions. Specifically, the projectors implemented are based on the blob and
B-Spline basis function as presented in \autoref{chap:signal_representation}.

The forward model is implemented elsa, and can be found in the master branch. This section covers
first a general overview of the algorithm, plus some details on the implementation. Recall the
definition of image from \autoref{def:signal}, this method works for both \(2\) and \(3\) dimensional
images, however, to avoid repetition I will only talk about images and refer to both.

For both the forward and backward projection, the implemented algorithm can be categorized as a
ray-driven approach. I.e.\ it traverses each ray through the image going form all the poses defined
to each detector pixel.

Given a ray, the first step is a quick intersection test of the ray and the bounding box of the
image. If the ray hits the bounding box, it needs to traverse the regular spaced grid of the image.
The specific traversal algorithm is based on~\cite{amanatides_fast_1987}, however it is simplified.
In \autoref{fig:visualization_siddon_traversal} the steps the algorithm would take can be seen. It
ensures, each voxel is visited, even if it is only for a small amount.

For the purpose of this projector, however, the basis functions considered have a support larger
than a single voxel. Hence, the previous algorithm can be simplified to traverse in a
\textit{slice-by-slice} fashion. Specifically, this means, that the algorithm only steps a fixed
width in the leading direction of the given ray. The voxels visited this way are referred to as
\textit{center voxels}, as they are the center of a given slice. Then for each center voxel, the
neighboring voxels perpendicular to the leading direction are visited for a given distance. The
distance depends on the specific choice of basis function and their parameters. This is depicted in
\autoref{fig:visualization_slice_traversal}.

\begin{figure}[h]
	\centering
	\makebox[\textwidth]{ \makebox[1.3\textwidth]{%
			\begin{subfigure}{0.65\textwidth}
				\includegraphics[width=\textwidth]{./figures/projectors_slice/slice_traversal.png}
				\caption{}\label{fig:visu_slice_traver_a}%
			\end{subfigure}%
			\begin{subfigure}{0.65\textwidth}
				\includegraphics[width=\textwidth]{./figures/projectors_slice/slice_traversal_clean.png}
				\caption{}\label{fig:visu_slice_traver_b}%
			\end{subfigure}
		}}
	\caption{\subref{fig:visu_slice_traver_a} Visualization of the voxels visited
		for the slice traversal. The ray (cardinal red line) traverses the image. For each
		center voxel (darker blue), neighbor voxels (light blue) perpendicular to the
		leading direction (here the x-direction) are visited for a certain distance (here
		the distance is \(1\)). The center voxels are traversed from closer to the ray
		origin, in this visualization on the left, going to the right. The dotted black
		lines indicate the perpendicular distance from the voxel center to the ray.
		\subref{fig:visu_slice_traver_b} same setting as in
		\subref{fig:visu_slice_traver_a}, however, the support of the basis function is
		depicted (black circle), only for a single slice. There the intersection between the
		ray and the circle is the Radon Transform of the ray and the contribution of the
		basis function to the projection.
	}\label{fig:visualization_slice_traversal}
\end{figure}

The next important part of the projector is the evaluation of the weight for each visited voxel.
Looking at \autoref{fig:visualization_slice_traversal}, the dotted black lines indicate the
distances from the voxel center to the ray. These distances are used to compute the Radon Transform
of the given basis function.

In the case of blob basis functions, the closed form solution given in \autoref{eq:radon_blob_basis}
is used with \(r\) being exactly the distance just mentioned. In the case of B-Splines it is just a
touch more complicated. As explained in \autoref{sec:bspline_basis}, the projected B-Spline are
again B-Splines. However, B-Splines are not perfectly spherically symmetric. But they are close
enough to assume it (compare~\cite{momey_b-spline_2012, momey_spline_2015} for specifics) Hence, it
is assumed that the B-Spline are spherically symmetric, and therefore the evaluation (and using
their separability) simplifies to
\begin{equation}
	\radon\beta^d(r) = \beta^d(r) \prod^{n - 1}_{k=1} \beta^d(0)
\end{equation}
For both basis functions, a \gls{LUT} is computed first for the creation of the projector, and the
for the actual projections the values are looked up.

The same traversal and evaluation methods are used for both the forward and backward projection, and
hence no further discussion is needed there. However, one more point should be noted for the sake of
completeness. As \citeauthor*{momey_spline_2015}~\cite{momey_spline_2015} showed for B-Splines and
\citeauthor*{kohler_iterative_2011}~\cite{kohler_iterative_2011} for blobs, it is possible to use
footprints instead of a ray-driven approach for both basis functions. The only reason this wasn't
pursued during this thesis, is the lack of experience of the footprint based methods in elsa. This
definitely is a possible area of improvement in the projector.

Also note, the current version of the projector only supports X-ray attenuation CT\@. However, this
projector should be easily adapted to instead of the Radon Transform, yield the derivative of the
Radon Transform as weights. In theory, it should only be necessary to create a function to populate
the \gls{LUT}, and a very thing wrapper around it to provide the functionality needed for elsa to
use it as a projector.

As already touched on in \autoref{sec:blob_basis} and \autoref{sec:bspline_basis} both basis
functions provide a couple of important parameters. Especially, the blob basis function depends on
the exact values used for \(m\), \(\alpha\) and \(a\). Regarding B-Splines, they only have the
parameter regarding the order \(d\). For the projector based on B-Splines only cubic B-Splines are
used. The rationale is given in \autoref{sec:bspline_basis}. The projector using the blob basis
function is configurable, and the parameters can be passed to projector.

\chapter{Experiments}\label{chap:experiments}

After discussing many aspects of the tomographic reconstruction, and the presentation of
the implementation of two projectors. The projectors are thoroughly tested and investigated. This
chapter structured as follows: Before diving into the experiments itself, a quick overview of the
different error metrics is given. This is followed by a series of experiments. These experiments
include the forward projection of the projectors, and the reconstruction with 3 different phantoms.
The phantoms vary in complexity to analyze different behaviors. Also, an experiment using synthetic
noise is conducted. The final part of the analysis covers runtime performance aspects of the
projectors.

Reproducibility is important to me. Therefore, all experiments can be run with the main branch of
elsa. Please check out the \href{https://gitlab.lrz.de/IP/elsa}{GitLab repository} to find
instructions on how to build and run elsa. All experiments are based on the
\textit{example\_argparse} in the \textit{example} folder of the repository.

The experiments are run using elsa. Hence, some necessary details which hold for all the experiment
setups given later on. The overall geometric setup elsa follows is as described in
\citeauthor{hartley_multiple_2003}~\cite{hartley_multiple_2003}. The CT projection geometry uses a
cone beam geometry, with flat detector. Further, elsa is itself unitless. Therefore, all distances
are given in relation to pixel sizes.

\section{Error measurements}\label{sec:error_measurements}

Error measurements are a delicate topic. They can fake a false sense of security. Metrics might
indicate an improved result, however visually the result might be worse. Therefore, multiple errors
are taken into considerations here.

\begin{definition}[Mean Squared Error]
	One of the most famous error metrics: \textit{\gls{MSE}}. It is defined as the
	squared differences of each pixel value, weighted with the total number of pixels
	\[ MSE = \frac{1}{n} \sum_{i=1}^{n}(x_i - \hat{x}_i)^2\]
	where \(x_i\) is each voxel of the reference image, \(\hat{x}_i\) each voxel of the comparison
	image and \(n\) the total number of voxels in the image.
\end{definition}

Building on top of the \gls{MSE}, the \textit{\gls{RMSE}}, is the square root of the
\gls{MSE}.
\begin{definition}[Normalized Root Mean Squared Error]
	The \textit{\glsfirst{RMSE}} is defined as
	\[ RMSE = \sqrt{MSE(x, \hat{x}}) \]
	where \(x\) is the reference image and \(\hat{x}\) the comparison image. Further, the
	\textit{\glsfirst{NRMSE}} is defined as
	\[ NRMSE = \frac{RMSD}{x_{max} - x_{min}} \]
	with \(x_{min}\) and \(x_{max}\) are the minimum and maximum values of the reference image,
	respectively.
\end{definition}
Another frequently used metric is the \textit{\gls{PSNR}}
\begin{definition}[Peak Signal to Noise Ration]
	The \textit{\glsfirst{PSNR}} is defined as
	\[ PSNR = 10 \cdot \log_{10}\left( \frac{MAX}{\sqrt{MSE}} \right) \]
	where \(MAX\) is the maximum possible range depending on the data type of the images
\end{definition}
\inlinetodo{Add references}
All the so far given metrics are based on the \gls{MSE}, i.e.\ they measure based on exact voxel by
voxel values. They measure absolute errors and weight them depending on the exact method. However,
they don't necessarily have much in common with the visual perception of humans. The
\textit{\gls{SSIM}}~\cite{wang_image_2004,avanaki_exact_2009} is a perception-based approach.
Rather than a pixel to pixel relation, \gls{SSIM} use the relation of pixels in close proximity to
each other. Plus some other important values important to the human perception.
\begin{definition}[Structural Similarity Index Measure]
	given by Given a reference image \(x\) and a comparison image \(\hat{x}\) then the
	\textit{\glsfirst{SSIM}} is
	\[ SSIM(x, \hat{x}) = \frac{\left(2 \mu_x \mu_{\hat{x}} + c_1\right) \left( 2
			\sigma_{x, \hat{x}} + c_2 \right)}{\left(\mu_x^2
			+ \mu_{\hat{x}}^2 + c_1 \right) \left( \sigma_x^2 + \sigma_{\hat{x}}^2 + c_2 \right)} \]
	where \(\mu_x\) is the average of \(x\), \(\mu_{\hat{x}}\) is the average of \(\hat{x}\),
	\(\sigma_x^2\) is the variance of \(x\), \(\sigma_{\hat{x}}^2\) is the variance of
	\(\hat{x}\), and \(\sigma_{x, \hat{x}}\) is the covariance of \(x\) and \(\hat{x}\)
\end{definition}
The promise that the value returned by \gls{SSIM} a value in the range \([0, 1]\), and if the value
is closer to one, it is `better' for the human perception.

Throughout the experiment sections, all error metrics are calculated the same way. The error norms
are computed using the scikit python package~\cite{van_der_walt_scikit-image_2014}. The snippet of
python code in Listing \autoref{py:error_metric} illustrates this.

\begin{listing}[h]
	\begin{minted}{python}
from skimage.metrics import structural_similarity
from skimage.metrics import mean_squared_error
from skimage.metrics import normalized_root_mse
from skimage.metrics import peak_signal_noise_ratio

# Assuming both phantom and reconstruction are read in somehow
MSE = mean_squared_error(phantom, reconstruction)
NRMSE = normalized_root_mse(phantom, reconstruction)
PSNR = peak_signal_noise_ratio(phantom, reconstruction)
SSIM = structural_similarity(phantom, reconstruction,
        data_range=phantom.max() - phantom.min())
    \end{minted}
	\caption{Computation of \gls{MSE}, \gls{NRMSE}, \gls{PSNR} and \gls{SSIM} using the scikit
		python package.}\label{py:error_metric}
\end{listing}

\section{Forward Projection}\label{sec:experiments_forward_projection}

As explained in details in the background part of thesis, the forward projection is in the case of
attenuation X-ray CT, the line integral through the object. Its quality is essential for every
reconstruction task.

For this section, the sinograms of the Shepp-Logan phantom~\cite{shepp_fourier_1974} (shown in
\autoref{fig:overview_shepp_logan_phantom}) are compared. The sinogram is computed for each
different projector method (The projector using blobs and B-Splines, Siddon's and Joseph's method).
The projection is performed in \(2\)D. The resolution of the initial phantom is \(512 \times 512\)
pixel. The sinogram is computed for in total \(768\) projection angles arranged equally spaced in a
full circle around the object. The source is \(51200\) units away from the image center and the
detector is \(512\) units away. The one-dimensional flat detector has a resolution of \(\lfloor 512
* \sqrt{2} \rfloor = 724\) for each pose.

A similar setup was used for the three-dimensional forward projection. Apart from the difference in
dimension, it uses as phantom size of \(256 \times 256 \times 256\) voxels. The detector is flat and
a complete (i.e.\ \(360^\circ\)) circular trajectory around the object is used. The detector has a
resolution of \(362 \times 362\) pixels. In total \(384\) equally spaced projection poses are
acquired equivalent to a rotation around the \(y\)-axis, with the center of rotation being the
center of the volume. The distance from the source to the center is \(25600\) units and the
distance from the center of the volume to the principal point of the detector is \(256\) units.

The recommendations discussed in \autoref{chap:signal_representation} are followed for the
parameters of the basis functions. The projector using B-Splines specifically uses cubic B-Splines.
And the blob parameters as discussed by~\cite{levakhina_three-dimensional_2014} are followed, i.e.\
\(m = 2\), \(\alpha = 10.83\) and \(a = 2\).

\begin{figure}[h]
	\centering
	% Taken from https://tex.stackexchange.com/a/398309
	\makebox[\textwidth]{ \makebox[1.3\textwidth]{%
			\rotatebox[origin=x]{90}{\bfseries Forward projections\strut}
			\begin{subfigure}{0.3125\textwidth}
				\stackinset{c}{}{t}{-.2in}{}{%
					\includegraphics[width=\textwidth]{./figures/experiments/forward_projection/2dsinogram_Blob_windowed.png}}
				\caption{Blob based projector}\label{fig:sinogram_blob}
			\end{subfigure}%
			\begin{subfigure}{0.3125\textwidth}
				\stackinset{c}{}{t}{-.2in}{}{%
					\includegraphics[width=\textwidth]{./figures/experiments/forward_projection/2dsinogram_BSpline_windowed.png}}
				\caption{B-Spline based projector}\label{fig:sinogram_bspline}
			\end{subfigure}%
			\begin{subfigure}{0.3125\textwidth}
				\stackinset{c}{}{t}{-.2in}{}{%
					\includegraphics[width=\textwidth]{./figures/experiments/forward_projection/2dsinogram_Siddon_windowed.png}}
				\caption{Siddon's projector}\label{fig:sinogram_siddon}
			\end{subfigure}%
			\begin{subfigure}{0.3125\textwidth}
				\stackinset{c}{}{t}{-.2in}{}{%
					\includegraphics[width=\textwidth]{./figures/experiments/forward_projection/2dsinogram_Joseph_windowed.png}}
				\caption{Joseph's projector}\label{fig:sinogram_joseph}
			\end{subfigure}%
		}}

	\makebox[\textwidth]{ \makebox[1.3\textwidth]{%
			\rotatebox[origin=c]{90}{\bfseries Absolute Normalized Differences\strut}
			\begin{subfigure}{0.3125\textwidth}
				\stackinset{c}{}{t}{-.2in}{}{%
					\includegraphics[width=\textwidth]{./figures/experiments/forward_projection/2dsinogram_difference_Blob_Siddon_windowed.png}}
				\caption{Difference Blob and Siddon}\label{fig:sino_diff_blob_siddon}
			\end{subfigure}%
			\begin{subfigure}{0.3125\textwidth}
				\stackinset{c}{}{t}{-.2in}{}{%
					\includegraphics[width=\textwidth]{./figures/experiments/forward_projection/2dsinogram_difference_Blob_Joseph_windowed.png}}
				\caption{Difference Blob and Joseph}\label{fig:sino_diff_blob_joseph}
			\end{subfigure}%
			\begin{subfigure}{0.3125\textwidth}
				\stackinset{c}{}{t}{-.2in}{}{%
					\includegraphics[width=\textwidth]{./figures/experiments/forward_projection/2dsinogram_difference_BSpline_Siddon_windowed.png}}
				\caption{Difference B-Spline and Siddon}\label{fig:sino_diff_bspline_siddon}
			\end{subfigure}%
			\begin{subfigure}{0.3125\textwidth}
				\stackinset{c}{}{t}{-.2in}{}{%
					\includegraphics[width=\textwidth]{./figures/experiments/forward_projection/2dsinogram_difference_BSpline_Joseph_windowed.png}}
				\caption{Difference B-Spline and Joseph}\label{fig:sino_diff_bspline_joseph}
			\end{subfigure}%
		}}

	\makebox[\textwidth]{ \makebox[1.3\textwidth]{%
			\begin{subfigure}[b]{0.3125\textwidth}
				\includegraphics[width=\textwidth]{./figures/experiments/forward_projection/2dsinogram_difference_Blob_BSpline_windowed.png}
				\caption{Difference Blob and B-Spline}\label{fig:sino_diff_blob_bspline}
			\end{subfigure}
			\begin{subfigure}[b]{0.3321\textwidth}
				\includegraphics[width=\textwidth]{./figures/experiments/forward_projection/plot_sino_differences.png}
				\caption{Cross-section of \(180\)° projection}\label{fig:plot_sino_differences}
			\end{subfigure}
			\begin{subfigure}[b]{0.3321\textwidth}
				\includegraphics[width=\textwidth]{./figures/experiments/forward_projection/2dphantom_windowed.png}
				\caption{Shepp-Logan phantom}\label{fig:forward_shepp_logan_phantom}
			\end{subfigure}
		}}
	\caption{\subref{fig:sinogram_blob}--\subref{fig:sinogram_joseph} sinogram of \(256 \times
		256\) Shepp-Logan phantom for the Blob based, B-Spline based, Siddon's and Joseph's
		projector (left to right), with \(384\) different poses in a \(360\)° arc around the
		phantom using fan-beam geometry.%
		\subref{fig:sino_diff_blob_siddon}-\subref{fig:sino_diff_bspline_joseph} Absolute
		difference of normalized sinograms between (from left to right), the Blob based to
		Siddon's and Joseph's projector, then the B-Spline based to Siddon's and Joseph's
		projector. \subref{fig:sino_diff_blob_bspline} Absolute difference of normalized
		sinogram between the sinogram for the Blob based and B-Spline based projector.%
		\subref{fig:plot_sino_differences} Plot of cross sections at \(180\)° (i.e.\ the
		center row) of sinograms. The intensity difference of the Blob based projector
		compared to the others is clearly visible. \subref{fig:forward_shepp_logan_phantom} original
		Shepp-Logan phantom used to create the sinograms}%
	\label{fig:sinogram_shepp_logan}
\end{figure}

Looking at \autoref{fig:sinogram_shepp_logan}, an overview of the forward projection for the
different projectors is shown. As a first, an obvious observation: All of the forward projections
are accatable. There exist differences (as it can be seen in the second and third row of
\autoref{fig:sinogram_shepp_logan}), but one has to look more in-depth. Staying with the overview
first and looking at the first row, one should note the extreme difference in displayed values. The
projectors based on the blob basis functions is noticeably brigther. This is reconfirmed with the
plot in \autoref{fig:plot_sino_differences}. The plot shows a cross-section slice of the sinogram at
\(180^\circ\). There it is clear, the blob based projector has a throughout higher intensity value.
This also motivates the choice of absolute normalized differences. Focusing on the overall visible
artifacts, one can see in the different images
(\autoref{fig:sino_diff_blob_siddon} -- \autoref{fig:sino_diff_bspline_joseph}), artifacts most
notably in the differences to the Siddon's projector. The area around angles divadable by
\(90^\circ\) contains clearly ring like artifacts especially in the boundary portion of the
sinogram. The blob and B-Spline based projectors compared to the Joseph's have the most visible
difference at multiples of \(45^\circ\).
Lastly looking at the difference between the blob and and B-Spline based projector in
\autoref{fig:sino_diff_blob_bspline}, once can see clear differences in at multiples of
\(45^\circ\).

\begin{figure}[h]
	\centering

	\makebox[\textwidth]{ \makebox[1.3\textwidth]{%
			\rotatebox[origin=c]{90}{\bfseries Forward projections\strut}
			\begin{subfigure}{0.3125\textwidth}
				\stackinset{c}{}{t}{-.2in}{}{%
					\includegraphics[width=\textwidth]{./figures/experiments/forward_projection/2dsinogram_cropped_Blob.png}}
				\caption{Blob based projector}\label{fig:sinogram_blob_center_crop}
			\end{subfigure}%
			\begin{subfigure}{0.3125\textwidth}
				\stackinset{c}{}{t}{-.2in}{}{%
					\includegraphics[width=\textwidth]{./figures/experiments/forward_projection/2dsinogram_cropped_BSpline.png}}
				\caption{B-Spline based projector}\label{fig:sinogram_bspline_center_crop}
			\end{subfigure}%
			\begin{subfigure}{0.3125\textwidth}
				\stackinset{c}{}{t}{-.2in}{}{%
					\includegraphics[width=\textwidth]{./figures/experiments/forward_projection/2dsinogram_cropped_Siddon.png}}
				\caption{Siddon's projector}\label{fig:sinogram_siddon_center_crop}
			\end{subfigure}%
			\begin{subfigure}{0.3125\textwidth}
				\stackinset{c}{}{t}{-.2in}{}{%
					\includegraphics[width=\textwidth]{./figures/experiments/forward_projection/2dsinogram_cropped_Joseph.png}}
				\caption{Joseph's projector}\label{fig:sinogram_joseph_center_crop}
			\end{subfigure}%
		}}

	\makebox[\textwidth]{ \makebox[1.3\textwidth]{%
			\rotatebox[origin=c]{90}{\bfseries Absolute Normalized Differences\strut}
			\begin{subfigure}{0.3125\textwidth}
				\includegraphics[width=\textwidth]{./figures/experiments/forward_projection/2dsinogram_cropped_difference_Blob_Siddon.png}
				\caption{Difference Blob and Siddon}\label{fig:sinogram_difference_blob_siddon_crop}
			\end{subfigure}%
			\begin{subfigure}{0.3125\textwidth}
				\includegraphics[width=\textwidth]{./figures/experiments/forward_projection/2dsinogram_cropped_difference_Blob_Joseph.png}
				\caption{Difference Blob and Joseph}\label{fig:sinogram_difference_blob_joseph_crop}
			\end{subfigure}%
			\begin{subfigure}{0.3125\textwidth}
				\includegraphics[width=\textwidth]{./figures/experiments/forward_projection/2dsinogram_cropped_difference_BSpline_Siddon.png}
				\caption{Difference B-Spline and Siddon}\label{fig:sinogram_difference_bspline_siddon_crop}
			\end{subfigure}%
			\begin{subfigure}{0.3125\textwidth}
				\includegraphics[width=\textwidth]{./figures/experiments/forward_projection/2dsinogram_cropped_difference_BSpline_Joseph.png}
				\caption{Difference B-Spline and Joseph}\label{fig:sinogram_difference_bspline_joseph_crop}
			\end{subfigure}%
		}}

	\makebox[\textwidth]{ \makebox[1.3\textwidth]{%
			\begin{subfigure}[t]{0.3125\textwidth}
				\includegraphics[width=\textwidth]{./figures/experiments/forward_projection/2dsinogram_cropped_difference_Blob_BSpline.png}
				\caption{Sinogram with marked region of crops}\label{fig:sinogram_difference_blob_bspline_crop}
			\end{subfigure}
			\begin{subfigure}[t]{0.3125\textwidth}
				\includegraphics[width=\textwidth]{./figures/experiments/forward_projection/2dsinogram_Blob_rectangle.png}
				\caption{Sinogram with marked region of crops}\label{fig:sinogram_crop_overview}
			\end{subfigure}%
		}}
	\caption{\subref{fig:sinogram_blob_center_crop}--\subref{fig:sinogram_joseph_center_crop}
		crop into the sinogram shown in \autoref{fig:sinogram_shepp_logan}. Showing a part
		of the projections of around \(180\)° to \(225\)°.
		\subref{fig:sinogram_difference_blob_siddon_crop}--\subref{fig:sinogram_difference_blob_bspline_crop}
		same cropped region as the first row, but showing the absolute normalized
		differences. \subref{fig:sinogram_crop_overview} sinogram with rectangle marking the
		cropped region images in
		\subref{fig:sinogram_blob_center_crop}--\subref{fig:sinogram_difference_bspline_joseph_crop}
	}
	\label{fig:sinogram_shepp_logan_cropped}
\end{figure}

Taking a closer look in \autoref{fig:sinogram_shepp_logan_cropped}, a crop of the results from
\autoref{fig:sinogram_shepp_logan} can be seen. Specifically, a crop that contains a protion of the
projections with angles \([180^\circ, 225^\circ]\). In this close up, the typical ring like
artifacts
of ray-driven projectors can be clearly seen in the forward projection of the Siddon's projector
(\autoref{fig:sinogram_siddon_center_crop}). The other sinograms look clean even on this scale.
Looking at the second row of \autoref{fig:sinogram_shepp_logan_cropped}, the absolute normalized
difference images, do show more artifacts. Now also the Joseph's projector shows clear visible
ring-like artifacts at angles around \(90^\circ\), but also multiples of \(45^\circ\) degree.
The difference between the projectors based on the blob and B-Spline basis functions show
differences. The most pronounced at multiples of \(45^\circ\), but in the cropped image, each
\(22^\circ\) artifacts can be seen.

Without repeating to much, similar artifacts can be seen in the three-dimensional case, depicted in
\autoref{fig:3dsinogram_shepp_logan}. The Siddon's method performs worst. Artefacts can be found in
the Joseph's projector, but are few and subtle. Basically no artifacts can be found in the blob
based projector. However, for the B-Spline based projector the lateral slices of the projection
\(45^\circ\) shows clear strip like artifacts. They are not visible slices for the next closest
projection angles before and after. The adjacent views are shown in
\autoref{fig:sinogram_bspline_3d_closeup}, with the same slice again in the middle. The artifacts
are present every \(90^\circ\), i.e.\ at \(45^\circ\), \(135^\circ\), \(225^\circ\), and
\(315^\circ\). Note also, the brightness difference is notable again.

\begin{figure}[h]
	\centering

	\makebox[\textwidth]{ \makebox[1.3\textwidth]{%
			\rotatebox[origin=c]{90}{\bfseries \(45^\circ\) projection\strut}
			\begin{subfigure}{0.3125\textwidth}
				\stackinset{c}{}{t}{-.2in}{\bfseries Blob based\strut}{%
					\includegraphics[width=\textwidth]{./figures/experiments/forward_projection_3d/3dsinogram_Blob_48.png}}
				\label{fig:sinogram_blob_3d_slice_48}
			\end{subfigure}%
			\begin{subfigure}{0.3125\textwidth}
				\stackinset{c}{}{t}{-.2in}{\bfseries B-Spline based\strut}{%
					\includegraphics[width=\textwidth]{./figures/experiments/forward_projection_3d/3dsinogram_BSpline_48.png}}
				\label{fig:sinogram_bspline_3d_slice_48}
			\end{subfigure}%
			\begin{subfigure}{0.3125\textwidth}
				\stackinset{c}{}{t}{-.2in}{\bfseries Siddon's based\strut}{%
					\includegraphics[width=\textwidth]{./figures/experiments/forward_projection_3d/3dsinogram_Siddon_48.png}}
				\label{fig:sinogram_siddon_3d_slice_48}
			\end{subfigure}%
			\begin{subfigure}{0.3125\textwidth}
				\stackinset{c}{}{t}{-.2in}{\bfseries Joseph's based\strut}{%
					\includegraphics[width=\textwidth]{./figures/experiments/forward_projection_3d/3dsinogram_Joseph_48.png}}
				\label{fig:sinogram_joseph_3d_slice_48}
			\end{subfigure}%
		}}

	\makebox[\textwidth]{ \makebox[1.3\textwidth]{%
			\rotatebox[origin=c]{90}{\bfseries \(180^\circ\) projection\strut}
			\begin{subfigure}{0.3125\textwidth}
				\includegraphics[width=\textwidth]{./figures/experiments/forward_projection_3d/3dsinogram_Blob_192.png}
				\label{fig:sinogram_blob_3d_slice_192}
			\end{subfigure}%
			\begin{subfigure}{0.3125\textwidth}
				\includegraphics[width=\textwidth]{./figures/experiments/forward_projection_3d/3dsinogram_BSpline_192.png}
				\label{fig:sinogram_bspline_3d_slice_192}
			\end{subfigure}%
			\begin{subfigure}{0.3125\textwidth}
				\includegraphics[width=\textwidth]{./figures/experiments/forward_projection_3d/3dsinogram_Siddon_192.png}
				\label{fig:sinogram_siddon_3d_slice_192}
			\end{subfigure}%
			\begin{subfigure}{0.3125\textwidth}
				\includegraphics[width=\textwidth]{./figures/experiments/forward_projection_3d/3dsinogram_Joseph_192.png}
				\label{fig:sinogram_joseph_3d_slice_192}
			\end{subfigure}%
		}}

	\caption{Lateral slices of the forward projector of the \(3\)d Shepp-Logan phantom. Top row:
		lateral slice of the projection of \(45^\circ\). Bottom row: lateral slice of the
		projection of \(180^\circ\). Form left to right: forward projections using the blob
		based projector, B-Spline projector, Siddon's projector and Joseph's projector.
	}\label{fig:3dsinogram_shepp_logan}
\end{figure}

\begin{figure}[h]
	\centering

	\makebox[\textwidth]{ \makebox[1.3\textwidth]{%
			\begin{subfigure}{0.43\textwidth}
				\stackinset{c}{}{t}{-.2in}{}{%
					\includegraphics[width=\textwidth]{./figures/experiments/forward_projection_3d/3dsinogram_BSpline_47.png}}%
				\label{fig:sinogram_bspline_3d_closeup1}
			\end{subfigure}%
			\begin{subfigure}{0.43\textwidth}
				\stackinset{c}{}{t}{-.2in}{}{%
					\includegraphics[width=\textwidth]{./figures/experiments/forward_projection_3d/3dsinogram_BSpline_48.png}}%
				\label{fig:sinogram_bspline_3d_closeup2}
			\end{subfigure}%
			\begin{subfigure}{0.43\textwidth}
				\stackinset{c}{}{t}{-.2in}{}{%
					\includegraphics[width=\textwidth]{./figures/experiments/forward_projection_3d/3dsinogram_BSpline_49.png}}%
				\label{fig:sinogram_bspline_3d_closeup3}
			\end{subfigure}%
		}}
	\caption{Closeup of line artifacts found at the lateral slice of the projector using B-Spline
		basis functions. From left to right: slices of the projectors at \(44.0625^\circ\),
		\(45^\circ\), \(45.9375^\circ\)}\label{fig:sinogram_bspline_3d_closeup}
\end{figure}

\section{Reconstruction of Synthetic Data}\label{sec:experiments_synthethic_projection}

After the forward projection, numerous experiments using synthetic data is presented. In total 3
different phantoms are used, ranging from simple to complex. The first phantom is just a plain
rectangle. The rectangle is centered in the image and the image has a constant value of \(1\) inside
the rectangle, and \(0\) everywhere else. It is included as certain artifacts are spoted with ease
in such an easy setting. The next phantom, is the Shepp-Logan phantom~\cite{shepp_fourier_1974}. It
resembles a human head and is a standard test phantom. Due to its purely synthetic nature, it is
simple to compute error measures for reconstructions. The last phantom used is a rather complex one.
It is an example reconstruction of an human abdomen. However, the sinogram is generated from the
original image and it will be compared to it. Therefore, it is still a synthetic test, though the
complexity of the data resembles realistic medical data. All three phantoms are shown in
\autoref{fig:experiment_overview_phantoms}. Please note, that for all cases, the inverse
crime~\cite{wirgin_inverse_2004} is committed.

\begin{figure}[h]
	\centering
	\makebox[\textwidth]{ \makebox[1.3\textwidth]{%
			\begin{subfigure}[t]{0.40\textwidth}
				\includegraphics[width=\textwidth]{./figures/experiments/phantoms/rectangle_phantom.png}
				\caption{Rectangular Phantom}\label{fig:overview_rectanglular_phantom}
			\end{subfigure}%
			\begin{subfigure}[t]{0.40\textwidth}
				\includegraphics[width=\textwidth]{./figures/experiments/phantoms/shepp_logan.png}
				\caption{Shepp-Logan Phantom}\label{fig:overview_shepp_logan_phantom}
			\end{subfigure}%
			\begin{subfigure}[t]{0.40\textwidth}
				\includegraphics[width=\textwidth]{./figures/experiments/phantoms/abdomen_512_normalized.png}
				\caption{Reconstruction of a lateral slice of human abdomen used as phantom}\label{fig:overview_medical_phantom}
			\end{subfigure}%
		}}
	\caption{Overview of the different phantoms used for the synthetic tests. From left to
		right: Rectangular phantom, Shepp-Logan phantom, lateral cross-section human
		abdoman. All phantoms are are displayed with a windowed to values in the range \([0,
				1]\).}\label{fig:experiment_overview_phantoms}
\end{figure}

For all 3 phantoms the general setup is chosen to be the same. First in this section only
two-dimensional reconstructiosn are performed. The resolution of the phantom is \(512 \times 512\)
pixels. No additional noise was added. The sinogram is computed with each projector method and a
total of \(512\) projection angles equally spaced in an \(180^\circ\) arc around the object. Note
that the \(180^\circ\) projection angle is excluded, as it already does not provide additional
information. The one-dimensional detector has a resolution of \(\lfloor 512 * \sqrt{2} \rfloor =
724\)

As the reconstruction are performed using elsa, and elsa is itself unitless, distances are given in
respect to the size of voxels.   The size of voxels is \(1\) for all
dimensions. The distance from the simulated X-ray source is \(51200\) units away from the rotation
center (i.e. the center of the volume), and the principal point of the detector is \(512\) units
away.

The iterative reconstruction algorithm is \gls{FISTA}. The reoncsurctions for the rectangle and
Shepp-Logan phantom are run each for \(50\) iterations, and \(300\) iterations for the medical
phantom with the default regularization parameter of \(0.5\). This default regularization parameter
yielded among the best results.

As in \autoref{sec:experiments_forward_projection}, cubic B-Splines are used and the parameters for
the blob basis functions are \(m = 2\), \(\alpha = 10.83\) and \(a = 2\).

In the following sections, the experiments for each phantom are presented and directly discussed.

\subsection{Rectangular Phantom}

\begin{figure}[h]
	\centering
	\makebox[\textwidth]{ \makebox[1.3\textwidth]{%
			\rotatebox[origin=c]{90}{\bfseries Reconstruciton\strut}
			\begin{subfigure}{0.3125\textwidth}
				\stackinset{c}{}{t}{-.2in}{}{%
					\includegraphics[width=\textwidth]{./figures/experiments/reconstruction_rectangle/2dreconstruction_cropped_windowed_Blob.png}}
				\caption{Blob based projector}%
				\label{fig:rectangle_recon_artifacts_blob}
			\end{subfigure}%
			\begin{subfigure}{0.3125\textwidth}
				\stackinset{c}{}{t}{-.2in}{}{%
					\includegraphics[width=\textwidth]{./figures/experiments/reconstruction_rectangle/2dreconstruction_cropped_windowed_BSpline.png}}
				\caption{B-Spline based projector}%
				\label{fig:rectangle_recon_artifacts_bspline}
			\end{subfigure}%
			\begin{subfigure}{0.3125\textwidth}
				\stackinset{c}{}{t}{-.2in}{}{%
					\includegraphics[width=\textwidth]{./figures/experiments/reconstruction_rectangle/2dreconstruction_cropped_windowed_Siddon.png}}
				\caption{Siddon's projector}%
				\label{fig:rectangle_recon_artifacts_siddon}
			\end{subfigure}%
			\begin{subfigure}{0.3125\textwidth}
				\stackinset{c}{}{t}{-.2in}{}{%
					\includegraphics[width=\textwidth]{./figures/experiments/reconstruction_rectangle/2dreconstruction_cropped_windowed_Joseph.png}}
				\caption{Joseph's projector}%
				\label{fig:rectangle_recon_artifacts_joseph}
			\end{subfigure}%
		}}

	\makebox[\textwidth]{ \makebox[1.3\textwidth]{%
			\rotatebox[origin=c]{90}{\bfseries Difference to phantom\strut}
			\begin{subfigure}{0.3125\textwidth}
				\includegraphics[width=\textwidth]{./figures/experiments/reconstruction_rectangle/2dreconstruction_difference_cropped_windowed_Blob.png}
				\caption{Blob based projector}%
				\label{fig:rectangle_recon_difference_blob}
			\end{subfigure}%
			\begin{subfigure}{0.3125\textwidth}
				\includegraphics[width=\textwidth]{./figures/experiments/reconstruction_rectangle/2dreconstruction_difference_cropped_windowed_BSpline.png}
				\caption{B-Spline based projector}%
				\label{fig:rectangle_recon_difference_bspline}
			\end{subfigure}%
			\begin{subfigure}{0.3125\textwidth}
				\includegraphics[width=\textwidth]{./figures/experiments/reconstruction_rectangle/2dreconstruction_difference_cropped_windowed_Siddon.png}
				\caption{Siddon's projector}%
				\label{fig:rectangle_recon_difference_siddon}
			\end{subfigure}%
			\begin{subfigure}{0.3125\textwidth}
				\includegraphics[width=\textwidth]{./figures/experiments/reconstruction_rectangle/2dreconstruction_difference_cropped_windowed_Joseph.png}
				\caption{Joseph's projector}%
				\label{fig:rectangle_recon_difference_joseph}
			\end{subfigure}%
		}}

	\makebox[\textwidth]{
		\makebox[1.3\textwidth]{%
			\begin{subfigure}[b]{0.3125\textwidth}
				\includegraphics[width=\textwidth]{./figures/experiments/reconstruction_rectangle/2dphantom_rectangle.png}
				\caption{Rectangular phantom}%
				\label{fig:rectangle_recon_phantom}
			\end{subfigure}%
		}}
	\caption{\subref{fig:rectangle_recon_artifacts_blob}--\subref{fig:rectangle_recon_artifacts_joseph}
		crop of top left corner of the reconstruction rectangle with different projectors
		\subref{fig:rectangle_recon_difference_blob}--\subref{fig:rectangle_recon_difference_joseph}
		absolute normalized difference of reconstructed image and the phantom in
		\subref{fig:rectangle_recon_phantom}. In order from left to right and top down, the
		differences of the reconstruction using the Blob based, B-Spline, Siddon's and
		Joseph's projector and the original phantom. The window for each image is given on
		the right side for each image spearately \subref{fig:rectangle_recon_phantom}
		Rectangular phantom with marked region denoting the cropped region of the images in
		\subref{fig:rectangle_recon_artifacts_blob}--\subref{fig:rectangle_recon_difference_joseph}
	}%
	\label{fig:rectangle_recon_artifacts}
\end{figure}

The reconstruction of the simple rectangular phantom is mostly useful to clearly see the artifacts
of each method. In \autoref{fig:rectangle_recon_artifacts} a crop of the reconstruction can be seen.
There, one can see both for the Siddon's and Joseph's projector ring artifacts in both the center
and the corner of the rectangle. None of these kind of artifacts can be seen in the blob and
B-Spline based projectors. Looking at the second row with the differences images. These artifacts
are more clearly visible. Plus the edge of the rectangle is a cleaner in bot the blob and B-Spline
based projector compared to the Siddon's and Joseph's projector.

\begin{table}[h]%
	\centering
	\csvreader[
		head to column names,
		separator=semicolon,
		tabular = cccccc,
		table head = \toprule \textbf{Projector} & \textbf{\gls{MSE}} \downarrow &
		\textbf{\gls{NRMSE}} \downarrow & \textbf{\gls{PSNR}} (dB) \uparrow & \textbf{\gls{SSIM}} \uparrow \\\midrule,
		table foot = \bottomrule
	]{figures/experiments/reconstruction_rectangle/metrics.csv}{}{%
		\csvlinetotablerow%
	}
	\caption{Error metrics for the reconstruction of the rectangular phantom using 50 iterations
		of FISTA}%
	\label{tab:error_metric_rectangle}
\end{table}

The metrics discussed in \autoref{sec:error_measurements} for the reconstruction are shown in
\autoref{tab:error_metric_rectangle}. All \(l_2\) based norms (i.e. \gls{MSE}, \gls{NRMSE} and
\gls{PSNR}) are lower for the projectors based on Siddon's and Joseph's method. But all of them are
quite close to each other (around \(1e-4\) for \gls{MSE}). Notably they are all in the same order of
magnitude. \gls{SSIM} is similar as well for all, only the Siddon's method falls behind by a bit.

\subsection{Shepp-Logan Phantom}

\begin{figure}[h]
	\centering
	\makebox[\textwidth]{ \makebox[1.3\textwidth]{%
			\rotatebox[origin=c]{90}{\bfseries Reconstruction\strut}
			\begin{subfigure}{0.3125\textwidth}
				\stackinset{c}{}{t}{-.2in}{\bfseries Blob based\strut}{%
					\includegraphics[width=\textwidth]{./figures/experiments/reconstruction_shepp_logan/2dreconstruction_windowed_Blob.png}}
				\caption{}\label{fig:reconstruction_shepp_logan_blob}
			\end{subfigure}%
			\begin{subfigure}{0.3125\textwidth}
				\stackinset{c}{}{t}{-.2in}{\bfseries B-Spline based\strut}{%
					\includegraphics[width=\textwidth]{./figures/experiments/reconstruction_shepp_logan/2dreconstruction_windowed_BSpline.png}}
				\caption{}\label{fig:reconstruction_shepp_logan_bspline}
			\end{subfigure}%
			\begin{subfigure}{0.3125\textwidth}
				\stackinset{c}{}{t}{-.2in}{\bfseries Siddon's\strut}{%
					\includegraphics[width=\textwidth]{./figures/experiments/reconstruction_shepp_logan/2dreconstruction_windowed_Siddon.png}}
				\caption{}\label{fig:reconstruction_shepp_logan_siddon}
			\end{subfigure}%
			\begin{subfigure}{0.3125\textwidth}
				\stackinset{c}{}{t}{-.2in}{\bfseries Joseph's\strut}{%
					\includegraphics[width=\textwidth]{./figures/experiments/reconstruction_shepp_logan/2dreconstruction_windowed_Joseph.png}}
				\caption{}\label{fig:reconstruction_shepp_logan_joseph}
			\end{subfigure}%
		}}

	\makebox[\textwidth]{
		\makebox[1.3\textwidth]{%
			\rotatebox[origin=c]{90}{\bfseries Zoom into difference\strut}
			\begin{subfigure}{0.3125\textwidth}
				\includegraphics[width=\textwidth]{./figures/experiments/reconstruction_shepp_logan/2dreconstruction_difference_cropped_windowed_Blob.png}
				\caption{}\label{fig:reconstruction_shepp_logan_cropped_blob}
			\end{subfigure}%
			\begin{subfigure}{0.3125\textwidth}
				\includegraphics[width=\textwidth]{./figures/experiments/reconstruction_shepp_logan/2dreconstruction_difference_cropped_windowed_BSpline.png}
				\caption{}\label{fig:reconstruction_shepp_logan_cropped_bspline}
			\end{subfigure}%
			\begin{subfigure}{0.3125\textwidth}
				\includegraphics[width=\textwidth]{./figures/experiments/reconstruction_shepp_logan/2dreconstruction_difference_cropped_windowed_Siddon.png}
				\caption{}\label{fig:reconstruction_shepp_logan_cropped_siddon}
			\end{subfigure}%
			\begin{subfigure}{0.3125\textwidth}
				\includegraphics[width=\textwidth]{./figures/experiments/reconstruction_shepp_logan/2dreconstruction_difference_cropped_windowed_Joseph.png}
				\caption{}\label{fig:reconstruction_shepp_logan_cropped_joseph}
			\end{subfigure}%
		}}

	\makebox[\textwidth]{
		\makebox[1.3\textwidth]{%
			\begin{subfigure}[b]{0.3125\textwidth}
				\includegraphics[width=\textwidth]{./figures/experiments/reconstruction_shepp_logan/2dphantom_rectangle.png}
				\caption{}\label{fig:rectangle_shepp_logan_phantom}
			\end{subfigure}
		}}

	\caption{\subref{fig:reconstruction_shepp_logan_blob}--\subref{fig:reconstruction_shepp_logan_joseph}
		Reconstruction of the Shepp-Logan phantom using the (from left to right) the
		projectors using the blob and B-Spline basis function followed by the projector
		based on the Siddon's and Joseph's method.
		\subref{fig:reconstruction_shepp_logan_cropped_blob}--\subref{fig:reconstruction_shepp_logan_joseph}
		crop into the difference of each reconstruction with the original phantom, the same
		order of projectors is given. The window is given for each image individually.
		\subref{fig:rectangle_shepp_logan_phantom} the original phantom including a red
		rectangle indicating the croped region.
	}%
	\label{fig:reconstruction_shepp_logan}
\end{figure}

The reconstruction for the Shepp-Logan phantom is shown in figure
\autoref{fig:reconstruction_shepp_logan}. The top row shows the final reconstruction, the middle row
a crop into the difference image, and the bottom shows the region cropped into. For the first two
rows the reconstruction are performed using (from left to right) the projector using the blob basis
function, the B-Spline basis function, the Siddon's method and the Joseph's method.

As expected, the Siddon's method shows clear ring artifacts (best seen in
\autoref{fig:reconstruction_shepp_logan_siddon}). They are also present in the Joseph's method, but
they stand out less. Finally, both projectors based on the blob and B-Spline basis functions show no
sign of ring artifacts. Neither in the first row, nor in the cropped difference image.

\begin{table}[h]%
	\centering
	\csvreader[
		head to column names,
		separator=semicolon,
		tabular = cccccc,
		table head = \toprule \textbf{Projector} & \textbf{\gls{MSE}} \downarrow &
		\textbf{\gls{NRMSE}} \downarrow & \textbf{\gls{PSNR}} \uparrow & \textbf{\gls{SSIM}} \uparrow \\\midrule,
		table foot = \bottomrule
	]{figures/experiments/reconstruction_shepp_logan/metrics.csv}{}{%
		\csvlinetotablerow%
	}
	\caption{Error metrics for the reconstruction of the Shepp-Logan phantom running for \(50\)
		iterations of FISTA.}%
	\label{tab:error_metric_shepp_logan}
\end{table}

The metrics (listed in \autoref{tab:error_metric_shepp_logan}). Interestingly, the visually worst
reconstruction (Siddon's one), performs best regarding \(l_2\) based metrics (i.e.\ \gls{MSE},
\gls{NRMSE} and \gls{PSNR}). However, the visual perception is backed up by the \gls{SSIM}. The
Siddon's method performs quite a bit worse there (around \(8\%\) compared to both the blob and
B-Spline one), and both the blob and B-Spline based projector can outperform the Joseph's projector.
Though against the later one, it is a small difference.

\subsection{Human Abdomen}

\begin{figure}[h]
	\centering
	\makebox[\textwidth]{ \makebox[1.3\textwidth]{%
			\rotatebox[origin=c]{90}{\bfseries Reconstruction\strut}
			\begin{subfigure}{0.3125\textwidth}
				\stackinset{c}{}{t}{-.2in}{\bfseries Blob based\strut}{%
					\includegraphics[width=\textwidth]{./figures/experiments/reconstruction_medical/2dreconstruction_windowed_Blob.png}}
				\caption{}\label{fig:reconstruction_medical_blob}
			\end{subfigure}%
			\begin{subfigure}{0.3125\textwidth}
				\stackinset{c}{}{t}{-.2in}{\bfseries B-Spline based\strut}{%
					\includegraphics[width=\textwidth]{./figures/experiments/reconstruction_medical/2dreconstruction_windowed_BSpline.png}}
				\caption{}\label{fig:reconstruction_medical_bspline}
			\end{subfigure}%
			\begin{subfigure}{0.3125\textwidth}
				\stackinset{c}{}{t}{-.2in}{\bfseries Siddon's \strut}{%
					\includegraphics[width=\textwidth]{./figures/experiments/reconstruction_medical/2dreconstruction_windowed_Siddon.png}}
				\caption{}\label{fig:reconstruction_medical_siddon}
			\end{subfigure}%
			\begin{subfigure}{0.3125\textwidth}
				\stackinset{c}{}{t}{-.2in}{\bfseries Joseph's\strut}{%
					\includegraphics[width=\textwidth]{./figures/experiments/reconstruction_medical/2dreconstruction_windowed_Joseph.png}}
				\caption{}\label{fig:reconstruction_medical_joseph}
			\end{subfigure}%
		}}

	\makebox[\textwidth]{
		\makebox[1.3\textwidth]{%
			\rotatebox[origin=c]{90}{\bfseries Zoom into difference\strut}
			\begin{subfigure}{0.3125\textwidth}
				\includegraphics[width=\textwidth]{./figures/experiments/reconstruction_medical/2dreconstruction_difference_cropped_windowed_Blob.png}
				\caption{}\label{fig:reconstruction_medical_cropped_blob}
			\end{subfigure}%
			\begin{subfigure}{0.3125\textwidth}
				\includegraphics[width=\textwidth]{./figures/experiments/reconstruction_medical/2dreconstruction_difference_cropped_windowed_BSpline.png}
				\caption{}\label{fig:reconstruction_medical_cropped_bspline}
			\end{subfigure}%
			\begin{subfigure}{0.3125\textwidth}
				\includegraphics[width=\textwidth]{./figures/experiments/reconstruction_medical/2dreconstruction_difference_cropped_windowed_Siddon.png}
				\caption{}\label{fig:reconstruction_medical_cropped_siddon}
			\end{subfigure}%
			\begin{subfigure}{0.3125\textwidth}
				\includegraphics[width=\textwidth]{./figures/experiments/reconstruction_medical/2dreconstruction_difference_cropped_windowed_Joseph.png}
				\caption{}\label{fig:reconstruction_medical_cropped_joseph}
			\end{subfigure}%
		}}

	\makebox[\textwidth]{
		\makebox[1.3\textwidth]{%
			\begin{subfigure}[b]{0.3125\textwidth}
				\includegraphics[width=\textwidth]{./figures/experiments/reconstruction_medical/abdomen_512_normalized_rectangle.png}
				\caption{}\label{fig:medical_phantom_marked}
			\end{subfigure}
		}}

	\caption{\subref{fig:reconstruction_medical_blob}--\subref{fig:reconstruction_medical_joseph}
		Reconstruction of the medical phantom using the different projectors as a basis.
		From left, to right: the projector based on the blob basis, B-Spline basis, Siddon's
		method and finally Joseph's method. The second row
		(\subref{fig:reconstruction_medical_blob}--\subref{fig:reconstruction_medical_joseph})
		shows a crop into the difference image to the phantom data. The reconstructed image
		used the same projector as the reconstruction in the row above.
		\subref{fig:medical_phantom_marked} the original phantom including a red rectangle
		indicating the croped region.
	}%
	\label{fig:reconstruction_medical}
\end{figure}

The final phantom taken into account is the reconstruction of the human abdomen. The reoncsurctions
is shown in \autoref{fig:reconstruction_medical}. After the \(300\) solver iterations, all
reconstruction look decent. However, no great difference can be seen on this scale. Looking at the
cropped images in the second row of \autoref{fig:reconstruction_medical}, one can see that the new
projectors a cleaner compared especially to Siddon's based projector. Interestingly, the window
interval is a touch, but notibly higher for both the blob and B-Spline basis functions.

\begin{table}[h]%
	\centering
	\csvreader[
		head to column names,
		separator=semicolon,
		tabular = cccccc,
		table head = \toprule \textbf{Projector} & \textbf{\gls{MSE}} \downarrow &
		\textbf{\gls{NRMSE}} \downarrow & \textbf{\gls{PSNR}} \uparrow & \textbf{\gls{SSIM}} \uparrow \\\midrule,
		table foot = \bottomrule
	]{figures/experiments/reconstruction_medical/metrics.csv}{}{%
		\csvlinetotablerow%
	}
	\caption{Error metrics for the reconstruction of the medical phantom using FISTA running for
		\(300\) iterations}%
	\label{tab:error_metric_medical}
\end{table}

Interestingly, the \gls{SSIM} (right-most column in \autoref{tab:error_metric_medical}) for the blob
based projector is significantly lower compared to the other methods. The \(l_2\) based norms only
show a slightly worse performance of the blob based projector. This behaviour is consistent and ran
multiple times to ensure the correctness of this result.

\subsection{Behavior in the Presence of Noise}

Another interesting case is a detailed experiment with noisy data. The medical sample already
provided a sample including noise. But there it is ingrained in the phantom and hence can not be
analyzed independently. For this, the same overall setup was used as for the Shepp-Logan phantom in
the previous section, but additionally \(1\%\) of Gaussian noise is added to the sinogram before the
reconstruction process starts. I.e\ the mean of the Gaussian is \(0\) and the standard deviation is
set to be \(1\%\) of the value range of the sinogram. Further, the reconstruction is run for \(150\)
iterations, instead of \(50\), all the other parameters are set exactly as in the above case.

\begin{figure}[h]
	\centering
	\makebox[\textwidth]{ \makebox[1.3\textwidth]{%
			\rotatebox[origin=c]{90}{\bfseries Reconstruction\strut}
			\begin{subfigure}{0.3125\textwidth}
				\stackinset{c}{}{t}{-.2in}{\bfseries Blob based\strut}{%
					\includegraphics[width=\textwidth]{./figures/experiments/reconstruction_noise/2dreconstruction_windowed_Blob.png}}
				\caption{}\label{fig:reconstruction_noise_blob}
			\end{subfigure}%
			\begin{subfigure}{0.3125\textwidth}
				\stackinset{c}{}{t}{-.2in}{\bfseries B-Spline based\strut}{%
					\includegraphics[width=\textwidth]{./figures/experiments/reconstruction_noise/2dreconstruction_windowed_BSpline.png}}
				\caption{}\label{fig:reconstruction_noise_bspline}
			\end{subfigure}%
			\begin{subfigure}{0.3125\textwidth}
				\stackinset{c}{}{t}{-.2in}{\bfseries Siddon's\strut}{%
					\includegraphics[width=\textwidth]{./figures/experiments/reconstruction_noise/2dreconstruction_windowed_Siddon.png}}
				\caption{}\label{fig:reconstruction_noise_siddon}
			\end{subfigure}%
			\begin{subfigure}{0.3125\textwidth}
				\stackinset{c}{}{t}{-.2in}{\bfseries Joseph's\strut}{%
					\includegraphics[width=\textwidth]{./figures/experiments/reconstruction_noise/2dreconstruction_windowed_Joseph.png}}
				\caption{}\label{fig:reconstruction_noise_joseph}
			\end{subfigure}%
		}}

	\makebox[\textwidth]{
		\makebox[1.3\textwidth]{%
			\rotatebox[origin=c]{90}{\bfseries Zoom into difference\strut}
			\begin{subfigure}{0.3125\textwidth}
				\includegraphics[width=\textwidth]{./figures/experiments/reconstruction_noise/2dreconstruction_difference_cropped_windowed_Blob.png}
				\caption{}\label{fig:reconstruction_noise_cropped_blob}
			\end{subfigure}%
			\begin{subfigure}{0.3125\textwidth}
				\includegraphics[width=\textwidth]{./figures/experiments/reconstruction_noise/2dreconstruction_difference_cropped_windowed_BSpline.png}
				\caption{}\label{fig:reconstruction_noise_cropped_bspline}
			\end{subfigure}%
			\begin{subfigure}{0.3125\textwidth}
				\includegraphics[width=\textwidth]{./figures/experiments/reconstruction_noise/2dreconstruction_difference_cropped_windowed_Siddon.png}
				\caption{}\label{fig:reconstruction_noise_cropped_siddon}
			\end{subfigure}%
			\begin{subfigure}{0.3125\textwidth}
				\includegraphics[width=\textwidth]{./figures/experiments/reconstruction_noise/2dreconstruction_difference_cropped_windowed_Joseph.png}
				\caption{}\label{fig:reconstruction_noise_cropped_joseph}
			\end{subfigure}%
		}}
	\makebox[\textwidth]{
		\makebox[1.3\textwidth]{%
			\begin{subfigure}[b]{0.3125\textwidth}
				\includegraphics[width=\textwidth]{./figures/experiments/reconstruction_noise/2dphantom_rectangle.png}
				\caption{}\label{fig:phantom_noise_marked}
			\end{subfigure}
		}}

	\caption{\subref{fig:reconstruction_noise_blob}--\subref{fig:reconstruction_noise_joseph}
		Reconstruction of the Shepp-Logan phantom, when \(1\%\) of Gaussian noise is added
		to the sinogram. From left, to right: the projector based on the blob basis,
		B-Spline basis, Siddon's method and finally Joseph's method. The second row
		(\subref{fig:reconstruction_noise_cropped_blob}--\subref{fig:reconstruction_medical_cropped_joseph})
		shows a crop into the difference image to the phantom data. The reconstructed image
		used the same projector as the reconstruction in the row above.
		\subref{fig:phantom_noise_marked} the original phantom including a red rectangle
		indicating the croped region.
	}%
	\label{fig:reconstruction_noise}
\end{figure}

Both the blob and B-Spline based projectors handle gaussian noise nicely, as it can be seen in
\autoref{fig:reconstruction_noise}. They do show some for of artifacts, but that is to be expected
in the presence of noise. The center elipses are quite clean and the top most circle shows some
artifacts. However, it is quite a bit better controlled than the Siddon's and Joseph's method.

\begin{table}[h]%
	\centering
	\csvreader[
		head to column names,
		separator=semicolon,
		tabular = cccccc,
		table head = \toprule \textbf{Projector} & \textbf{\gls{MSE}} \downarrow &
		\textbf{\gls{NRMSE}} \downarrow & \textbf{\gls{PSNR}} \uparrow & \textbf{\gls{SSIM}} \uparrow \\\midrule,
		table foot = \bottomrule
	]{figures/experiments/reconstruction_noise/metrics.csv}{}{%
		\csvlinetotablerow%
	}
	\caption{Error metrics for the reconstruction of the noisy Shepp-Logan phantom using FISTA
		running for \(150\) iterations}%
	\label{tab:error_metric_noise}
\end{table}

This is also backed up by the error metrics given in \autoref{tab:error_metric_noise}. Both the
blob and B-Spline basis function based projectors perform better according to the \gls{SSIM}. And
with a decent lead ad that as well. Again they are both worse in the \gls{MSE} based metrics, but
just a bit at that.

\section{Quality Summary}\label{sec:experiments_quality_projection}

This section is a summary of the already discussed qualitative impression. As expected from the
literature, Siddon's method performs the worst among the methods compared. It shows clear ring like
artifacts in both the forward projection and the final reconstruction. In the forward projection is
performs specifically poor at angles around each \(90^\circ\). Furthermore, it poorly handles noise.
It should be noted however, that the Siddon's method performs equally well or sometimes better than
the other methods if one only looks at the \(l_2\) based error metrics (\gls{MSE}, \gls{NRMSE} and
\gls{PSNR}).

The Joseph's projector performs better. Is shows less artifacts in the forward projector, the
reconstructions are cleaner and noise is handled better. Thourghout the tests, it performs basically
identically in all \(l_2\) based error metrics, and beats the Siddon's method regarding the
\gls{SSIM} in all experiments.

\section{Performance Overview}\label{sec:experiments_performance_projection}

To investigate the runtime performance characteristics of the blob and B-Spline based projectors two
experiments are conducted. One for the two-dimensional and the other for the three-dimensional case.
The experiments run either the forward or the backward projector in a tight loop for a fixed number
of times. For the benchmarking of the forward projection, the phantom is generated once and then
reused for each iteration. For the backward projection, the sinogram is computed once and then
reused. After each iterator the elapsed time is stored and the next iteration is started. To ensure
both data and code is in cache, 2 warmup iterations are run. For both dimensions, only the size of
the original phantom changes. The number of projection positions is keep the same, to ensure only
the phantom size induces changes in the runtime.

The two-dimensional case uses a constant \(512\) projections positions and \(50\) iterations to
ensure a large enough sample size. Phantom sizes range from \(64 \times 64\) to \(512 \times 512\).
In the three-dimensional case, only \(64\) projection positions are used, the phantom sizes range
from \(32 \times 32 \times 32\) to \(64 \times 64 \times 64\) and only \(25\) iterations are run.

The same parameters for the basis function are used as with the other experiments. Importantly for
this section, both basis function have a support of \(2\).

The experiments are run on an system with an AMD Ryzen 7 3700X with \(8\) cores, \(16\) threads. The
experiments are performed an all \(16\) threads. The CPU has a base clock of \(3.6\) GHz, but is
able to boost up to \(4.4\) GHz under load, under certain conditions. The system has \(16013\) MiB
of main memory and the operating system is Linux based, running the '5.10.105-1-MANJARO` kernel. As
much as possible background tasks are closed during the benchmarks and the machines is left running
until finished without any user interaction.

\begin{table}[h]%
	\centering
	\csvreader[
		head to column names,
		separator=semicolon,
		tabular = cccccc,
		table head = \toprule \textbf{Dimension} & \textbf{Siddon} & \textbf{Joseph} \downarrow &
		\textbf{B-Spline based} \downarrow & \textbf{Blob based} \uparrow \\\midrule,
		table foot = \bottomrule
	]{figures/experiments/perf_testing/speedup.csv}{}{%
		\csvlinetotablerow%
	}
	\caption{Mean runtime slowdown over all compared sizes of the different projector methods,
		with the Siddon's projector as a baseline. The slowdown is the average over all the
		different experiment sizes. The first row shows the slow down for the
		two-dimensional case and the second row, for the three-dimensional case.}%
	\label{tab:runtime_slowdown}
\end{table}

\autoref{tab:runtime_slowdown} summaries the performance results. The results are given as slow
downs compared to the Siddon's projector. Hence, the B-Spline and Blob based projector are around
\(6\) to \(7\) times slower than the Siddon projector in the two-dimensional case. This is inline
with the finding of \citeauthor*{momey_b-spline_2012}~\cite{momey_b-spline_2012} (though they
compare against the distance-driven projector).

The performance drop for the three-dimensional case is close to another \(6\) time slowdown. But as
the number of visited voxels grows from \(5\) visited voxels to \(25\) in the three-dimensional
case, this is only to be expected.

In \autoref{fig:performance_2d_lineplot} the mean results and their standard deviations are ploted
for the two-dimensional case, and in \autoref{fig:performance_3d_lineplot} the results for the
three-dimensional case are plotted. Noticeably, the method's scale similarly as both the Siddon's
and Joseph's method, but the performance drop can be seen there as well.

\begin{figure}[h]
	\centering

	\makebox[\textwidth]{ \makebox[1.3\textwidth]{%
			% \rotatebox[origin=c]{90}{\bfseries Zoom into difference\strut}
			\begin{subfigure}{0.65\textwidth}
				\includegraphics[width=\textwidth]{./figures/experiments/perf_testing/plot_2d_line_forward.png}
				\caption{Runtime for the \(2\)d-forward projection}\label{fig:performance_2d_lineplot_forward}
			\end{subfigure}%
			\begin{subfigure}{0.65\textwidth}
				\includegraphics[width=\textwidth]{./figures/experiments/perf_testing/plot_2d_line_backward.png}
				\caption{Runtime for the \(2\)d-backward projection}\label{fig:performance_2d_lineplot_backward}
			\end{subfigure}%
		}}

	\caption{Runtime in seconds for the 4 different projection methods, both for the forward
		projection \subref{fig:performance_2d_lineplot_forward} and backward
		projection \subref{fig:performance_2d_lineplot_backward}. Lightly shaded area around
		the line plot show the tolerance interval based on the stdandard deviation of the
		samples for each size. Both axis use a base 2 logarithmic scale. Number of angles is
		fixed to \(512\) for all experiments.}%
	\label{fig:performance_2d_lineplot}
\end{figure}

\begin{figure}[h]
	\centering

	\makebox[\textwidth]{ \makebox[1.3\textwidth]{%
			% \rotatebox[origin=c]{90}{\bfseries Zoom into difference\strut}
			\begin{subfigure}{0.65\textwidth}
				\includegraphics[width=\textwidth]{./figures/experiments/perf_testing/plot_3d_line_forward.png}
				\caption{Runtime for the \(3\)d-forward projection}\label{fig:performance_3d_lineplot_forward}
			\end{subfigure}%
			\begin{subfigure}{0.65\textwidth}
				\includegraphics[width=\textwidth]{./figures/experiments/perf_testing/plot_3d_line_backward.png}
				\caption{Runtime for the \(3\)d-backward projection}\label{fig:performance_3d_lineplot_backward}
			\end{subfigure}%
		}}

	\caption{Runtime seconds for the 4 different projection methods, both for the
		three-dimensional forward projection \subref{fig:performance_2d_lineplot_forward}
		and backward projection \subref{fig:performance_2d_lineplot_backward}. Lightly
		shaded area around the line plot show the tolerance interval based on the stdandard
		deviation of the samples for each size. Both axis use a base 2 logarithmic scale.
		Number of angles is fixed to \(512\) for all experiments.}%
	\label{fig:performance_3d_lineplot}
\end{figure}

\autoref{fig:performance_2d_violin_forward} and \autoref{fig:performance_2d_violin_backward}
shows a detailed break down of the two-dimensional measurements from the previous plots. They are
split into the forward and backward projection and each projector has an individual plot. The blob
and B-Splien based projectors performance quite consistent compared to the Siddon's and Joseph's,
i.e. they have only few outliers.

\inlinetodo{Maybe make this a little more detailed, but idk just look at the plots and you see
	what's up}

\begin{figure}[h]
	\centering

	\makebox[\textwidth]{ \makebox[1.3\textwidth]{%
			% \rotatebox[origin=c]{90}{\bfseries Zoom into difference\strut}
			\begin{subfigure}{0.65\textwidth}
				\includegraphics[width=\textwidth]{./figures/experiments/perf_testing/plot_2d_violin_forward_Blob.png}
				\caption{}\label{fig:performance_2d_violin_forward_blob}
			\end{subfigure}%
			\begin{subfigure}{0.65\textwidth}
				\includegraphics[width=\textwidth]{./figures/experiments/perf_testing/plot_2d_violin_forward_BSpline.png}
				\caption{}\label{fig:performance_2d_violin_forward_bspline}
			\end{subfigure}%
		}}

	\makebox[\textwidth]{ \makebox[1.3\textwidth]{%
			% \rotatebox[origin=c]{90}{\bfseries Zoom into difference\strut}
			\begin{subfigure}{0.65\textwidth}
				\includegraphics[width=\textwidth]{./figures/experiments/perf_testing/plot_2d_violin_forward_Siddon.png}
				\caption{}\label{fig:performance_2d_violin_forward_siddons}
			\end{subfigure}%
			\begin{subfigure}{0.65\textwidth}
				\includegraphics[width=\textwidth]{./figures/experiments/perf_testing/plot_2d_violin_forward_Joseph.png}
				\caption{}\label{fig:performance_2d_violin_forward_joseph}
			\end{subfigure}%
		}}

	\caption{Violing plot for the forward projection of a selection of the sizes, individually
		given for the different projectors. Plot for the blob based projector
		\subref{fig:performance_2d_violin_forward_blob} B-Spline based
		\subref{fig:performance_2d_violin_forward_bspline}, Siddon's
		\subref{fig:performance_2d_violin_forward_siddons} and Joseph's
		\subref{fig:performance_2d_violin_forward_joseph}. The median is given as the white
		line, the thicker line is the interquantile range, the thin bars give the give the
		minimum and maximum respectively and the points outside of the minimum and maximum
		are the outliers. \(y\)-axes is in logarithm scale.}%
	\label{fig:performance_2d_violin_forward}
\end{figure}

\begin{figure}[h]
	\centering

	\makebox[\textwidth]{ \makebox[1.3\textwidth]{%
			% \rotatebox[origin=c]{90}{\bfseries Zoom into difference\strut}
			\begin{subfigure}{0.65\textwidth}
				\includegraphics[width=\textwidth]{./figures/experiments/perf_testing/plot_2d_violin_backward_Blob.png}
				\caption{}\label{fig:performance_2d_violin_backward_blob}
			\end{subfigure}%
			\begin{subfigure}{0.65\textwidth}
				\includegraphics[width=\textwidth]{./figures/experiments/perf_testing/plot_2d_violin_backward_BSpline.png}
				\caption{}\label{fig:performance_2d_violin_backward_bspline}
			\end{subfigure}%
		}}

	\makebox[\textwidth]{ \makebox[1.3\textwidth]{%
			% \rotatebox[origin=c]{90}{\bfseries Zoom into difference\strut}
			\begin{subfigure}{0.65\textwidth}
				\includegraphics[width=\textwidth]{./figures/experiments/perf_testing/plot_2d_violin_backward_Siddon.png}
				\caption{}\label{fig:performance_2d_violin_backward_siddon}
			\end{subfigure}%
			\begin{subfigure}{0.65\textwidth}
				\includegraphics[width=\textwidth]{./figures/experiments/perf_testing/plot_2d_violin_backward_Joseph.png}
				\caption{}\label{fig:performance_2d_violin_backward_joseph}
			\end{subfigure}%
		}}

	\caption{Violing plot for the backward projection of a selection of the sizes, individually
		given for the different projectors. Plot for the blob based projector
		\subref{fig:performance_2d_violin_forward_blob} B-Spline based
		\subref{fig:performance_2d_violin_forward_bspline}, Siddon's
		\subref{fig:performance_2d_violin_forward_siddons} and Joseph's
		\subref{fig:performance_2d_violin_forward_joseph}. The median is given as the white
		line, the thicker line is the interquantile range, the thin bars give the give the
		minimum and maximum respectively and the points outside of the minimum and maximum
		are the outliers. \(y\)-axes is in logarithm scale.}%
	\label{fig:performance_2d_violin_backward}
\end{figure}


%%%%%%%%%%%%%%%%%%%%%%%%%%%%%%%%%%%%%%%%%%%%%%%%%%%%%%%%%%%%%%%%%%%%%%%%
% Conclusion
%%%%%%%%%%%%%%%%%%%%%%%%%%%%%%%%%%%%%%%%%%%%%%%%%%%%%%%%%%%%%%%%%%%%%%%%
\part[Conclusion]{%
	Conclusion\\
	%
	% \vspace{1cm}
	% %
	% \begin{minipage}[l]{\textwidth}
	% 	%
	% 	\textnormal{%
	% 		\normalsize
	% 		%
	% 		\begin{singlespace*}
	% 			\onehalfspacing
	% 			%
	% 			Now we wrap it all up
	% 		\end{singlespace*}
	% 	}
	% \end{minipage}
}\label{part:conclusion}

\chapter{Conclusion}\label{chap:conclusion}

Let's wrap it all up


% This ensures that the subsequent sections are being included as root
% items in the bookmark structure of your PDF reader.
\bookmarksetup{startatroot}
\backmatter

\begingroup
\let\clearpage\relax
\glsaddall
\printglossary[type=\acronymtype]
\newpage
\printglossary
\endgroup

\printindex
\printbibliography

\end{document}
